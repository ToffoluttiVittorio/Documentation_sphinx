%% Generated by Sphinx.
\def\sphinxdocclass{report}
\documentclass[letterpaper,10pt,french]{sphinxmanual}
\ifdefined\pdfpxdimen
   \let\sphinxpxdimen\pdfpxdimen\else\newdimen\sphinxpxdimen
\fi \sphinxpxdimen=.75bp\relax
\ifdefined\pdfimageresolution
    \pdfimageresolution= \numexpr \dimexpr1in\relax/\sphinxpxdimen\relax
\fi
%% let collapsible pdf bookmarks panel have high depth per default
\PassOptionsToPackage{bookmarksdepth=5}{hyperref}

\PassOptionsToPackage{booktabs}{sphinx}
\PassOptionsToPackage{colorrows}{sphinx}

\PassOptionsToPackage{warn}{textcomp}
\usepackage[utf8]{inputenc}
\ifdefined\DeclareUnicodeCharacter
% support both utf8 and utf8x syntaxes
  \ifdefined\DeclareUnicodeCharacterAsOptional
    \def\sphinxDUC#1{\DeclareUnicodeCharacter{"#1}}
  \else
    \let\sphinxDUC\DeclareUnicodeCharacter
  \fi
  \sphinxDUC{00A0}{\nobreakspace}
  \sphinxDUC{2500}{\sphinxunichar{2500}}
  \sphinxDUC{2502}{\sphinxunichar{2502}}
  \sphinxDUC{2514}{\sphinxunichar{2514}}
  \sphinxDUC{251C}{\sphinxunichar{251C}}
  \sphinxDUC{2572}{\textbackslash}
\fi
\usepackage{cmap}
\usepackage[T1]{fontenc}
\usepackage{amsmath,amssymb,amstext}
\usepackage{babel}



\usepackage{tgtermes}
\usepackage{tgheros}
\renewcommand{\ttdefault}{txtt}



\usepackage[Sonny]{fncychap}
\ChNameVar{\Large\normalfont\sffamily}
\ChTitleVar{\Large\normalfont\sffamily}
\usepackage{sphinx}

\fvset{fontsize=auto}
\usepackage{geometry}


% Include hyperref last.
\usepackage{hyperref}
% Fix anchor placement for figures with captions.
\usepackage{hypcap}% it must be loaded after hyperref.
% Set up styles of URL: it should be placed after hyperref.
\urlstyle{same}

\addto\captionsfrench{\renewcommand{\contentsname}{Documentations:}}

\usepackage{sphinxmessages}
\setcounter{tocdepth}{1}



\title{Plateforme cartographique OLE}
\date{oct. 04, 2024}
\release{}
\author{Vittorio Toffolutti}
\newcommand{\sphinxlogo}{\vbox{}}
\renewcommand{\releasename}{}
\makeindex
\begin{document}

\ifdefined\shorthandoff
  \ifnum\catcode`\=\string=\active\shorthandoff{=}\fi
  \ifnum\catcode`\"=\active\shorthandoff{"}\fi
\fi

\pagestyle{empty}
\sphinxmaketitle
\pagestyle{plain}
\sphinxtableofcontents
\pagestyle{normal}
\phantomsection\label{\detokenize{index::doc}}


\sphinxAtStartPar
\sphinxhref{../latex/plateformecartographiqueole.pdf}{Documentation au format PDF}.

\sphinxstepscope


\chapter{Documentation utilisateur}
\label{\detokenize{doc_user:documentation-utilisateur}}\label{\detokenize{doc_user::doc}}
\sphinxAtStartPar
\sphinxhref{../latex/plateformecartographiqueole.pdf}{Documentation au format PDF}.

\sphinxstepscope


\section{Catalogue}
\label{\detokenize{doc_user/catalogue:catalogue}}\label{\detokenize{doc_user/catalogue::doc}}
\begin{sphinxShadowBox}
\sphinxstyletopictitle{Table des matières}
\begin{itemize}
\item {} 
\sphinxAtStartPar
\phantomsection\label{\detokenize{doc_user/catalogue:id1}}{\hyperref[\detokenize{doc_user/catalogue:introduction}]{\sphinxcrossref{Introduction}}}

\item {} 
\sphinxAtStartPar
\phantomsection\label{\detokenize{doc_user/catalogue:id2}}{\hyperref[\detokenize{doc_user/catalogue:la-recherche-des-donnees-dans-le-catalogue}]{\sphinxcrossref{La recherche des données dans le catalogue}}}

\item {} 
\sphinxAtStartPar
\phantomsection\label{\detokenize{doc_user/catalogue:id3}}{\hyperref[\detokenize{doc_user/catalogue:les-fonctionnalites-des-fiches-de-donnees}]{\sphinxcrossref{Les fonctionnalités des fiches de données}}}

\end{itemize}
\end{sphinxShadowBox}


\subsection{Introduction}
\label{\detokenize{doc_user/catalogue:introduction}}
\sphinxAtStartPar
Les données de ce catalogue proviennent de différents catalogue nationaux mais peuvent aussi être propre à l’Office de l’eau Réunion.
Chaque donnée est reliée à une organisation et possède une description détaillée.


\subsection{La recherche des données dans le catalogue}
\label{\detokenize{doc_user/catalogue:la-recherche-des-donnees-dans-le-catalogue}}
\sphinxAtStartPar
Il y a 3 onglets dans ce catalogue :
\begin{itemize}
\item {} 
\sphinxAtStartPar
\sphinxstylestrong{ACCUEIL} pour afficher les dernières données postées.

\end{itemize}

\noindent{\hspace*{\fill}\sphinxincludegraphics[width=600\sphinxpxdimen]{{catalogue_accueil}.png}\hspace*{\fill}}
\begin{itemize}
\item {} 
\sphinxAtStartPar
\sphinxstylestrong{DONNEES} pour afficher toutes les données.

\end{itemize}

\noindent{\hspace*{\fill}\sphinxincludegraphics[width=600\sphinxpxdimen]{{catalogue_donnees}.png}\hspace*{\fill}}
\begin{itemize}
\item {} 
\sphinxAtStartPar
\sphinxstylestrong{ORGANISATION} pour afficher toutes les organisations qui possèdent des données.

\end{itemize}

\noindent{\hspace*{\fill}\sphinxincludegraphics[width=600\sphinxpxdimen]{{catalogue_orga}.png}\hspace*{\fill}}

\sphinxAtStartPar
Vous pouvez filtrer vos recherches en fonction :
\begin{itemize}
\item {} 
\sphinxAtStartPar
de la date de publication

\item {} 
\sphinxAtStartPar
du type de données

\item {} 
\sphinxAtStartPar
du format

\item {} 
\sphinxAtStartPar
de l’organisation qui l’a publiée

\item {} 
\sphinxAtStartPar
des mots clés associés

\item {} 
\sphinxAtStartPar
ou encore du type de licence si elle est renseignée

\end{itemize}

\noindent{\hspace*{\fill}\sphinxincludegraphics[width=600\sphinxpxdimen]{{filter_options}.png}\hspace*{\fill}}


\subsection{Les fonctionnalités des fiches de données}
\label{\detokenize{doc_user/catalogue:les-fonctionnalites-des-fiches-de-donnees}}
\sphinxAtStartPar
Lorsque vous cliquez sur une donnée, la page de description de cette donnée s’affiche.


\subsubsection{Description de la donnée}
\label{\detokenize{doc_user/catalogue:description-de-la-donnee}}
\sphinxAtStartPar
Le haut de la page est dédiée à la description de cette donnée.
Il y’a :
\begin{itemize}
\item {} 
\sphinxAtStartPar
un titre

\item {} 
\sphinxAtStartPar
une description

\item {} 
\sphinxAtStartPar
la dernière date de la mise à jour,

\item {} 
\sphinxAtStartPar
son point de contact

\item {} 
\sphinxAtStartPar
le catalogue dont elle provient

\item {} 
\sphinxAtStartPar
les mots clés associées

\item {} 
\sphinxAtStartPar
un pourcentage à titre indicatif de la qualité de cette donnée

\item {} 
\sphinxAtStartPar
et d’autre informations plus technique

\end{itemize}

\noindent{\hspace*{\fill}\sphinxincludegraphics[width=600\sphinxpxdimen]{{fiche_info}.png}\hspace*{\fill}}


\subsubsection{Prévisualisation de la donnée}
\label{\detokenize{doc_user/catalogue:previsualisation-de-la-donnee}}
\sphinxAtStartPar
Une interface de prévisualisation est aussi accessible si vous descendez la page.
Cette interface permet de :
\begin{itemize}
\item {} 
\sphinxAtStartPar
prévisualiser la donnée

\end{itemize}

\noindent{\hspace*{\fill}\sphinxincludegraphics[width=600\sphinxpxdimen]{{fiche_previsu}.png}\hspace*{\fill}}
\begin{itemize}
\item {} 
\sphinxAtStartPar
visualiser le tableau attributaire

\end{itemize}

\noindent{\hspace*{\fill}\sphinxincludegraphics[width=600\sphinxpxdimen]{{fiche_table}.png}\hspace*{\fill}}
\begin{itemize}
\item {} 
\sphinxAtStartPar
faire différents graphiques en fonction des attributs

\end{itemize}

\noindent{\hspace*{\fill}\sphinxincludegraphics[width=600\sphinxpxdimen]{{fiche_graphe}.png}\hspace*{\fill}}

\begin{sphinxadmonition}{note}{Note:}
\sphinxAtStartPar
La couche de donnée, le tableau ou encore le graphique peuvent ne pas s’afficher car la donnée est mal configurée coté serveur.
\end{sphinxadmonition}


\subsubsection{Téléchargement de la donnée}
\label{\detokenize{doc_user/catalogue:telechargement-de-la-donnee}}
\sphinxAtStartPar
Vous pouvez aussi télécharger la donnée sous différents formats :

\noindent{\hspace*{\fill}\sphinxincludegraphics[width=600\sphinxpxdimen]{{fiche_tele}.png}\hspace*{\fill}}

\sphinxAtStartPar
Mais aussi avoir accées à d’autre liens et URL, ainsi qu’aux flux OGC disponibles :

\noindent{\hspace*{\fill}\sphinxincludegraphics[width=600\sphinxpxdimen]{{fiche_liens}.png}\hspace*{\fill}}

\begin{sphinxadmonition}{note}{Note:}
\sphinxAtStartPar
Ces liens sont dépéndant de la qualité de la donnée et de son intégration, ils peuvent ne pas fonctionner.
\end{sphinxadmonition}

\sphinxAtStartPar
Vous pouvez aussi visualiser la donnée dans une interface cartographique en cliquant ici et cela vous fera apparaître le {\hyperref[\detokenize{doc_user/visualiseur:id1}]{\sphinxcrossref{\DUrole{std,std-ref}{visualiseur}}}}.

\noindent{\hspace*{\fill}\sphinxincludegraphics[width=600\sphinxpxdimen]{{fiche_carto}.png}\hspace*{\fill}}

\sphinxstepscope


\section{Visualiseur}
\label{\detokenize{doc_user/visualiseur:visualiseur}}\label{\detokenize{doc_user/visualiseur::doc}}\phantomsection\label{\detokenize{doc_user/visualiseur:id1}}
\begin{sphinxShadowBox}
\sphinxstyletopictitle{Table des matières}
\begin{itemize}
\item {} 
\sphinxAtStartPar
\phantomsection\label{\detokenize{doc_user/visualiseur:id2}}{\hyperref[\detokenize{doc_user/visualiseur:introduction}]{\sphinxcrossref{Introduction}}}

\item {} 
\sphinxAtStartPar
\phantomsection\label{\detokenize{doc_user/visualiseur:id3}}{\hyperref[\detokenize{doc_user/visualiseur:la-gestion-des-couches}]{\sphinxcrossref{La gestion des couches}}}

\item {} 
\sphinxAtStartPar
\phantomsection\label{\detokenize{doc_user/visualiseur:id4}}{\hyperref[\detokenize{doc_user/visualiseur:les-fonctionnalites-techniques}]{\sphinxcrossref{Les fonctionnalités techniques}}}

\end{itemize}
\end{sphinxShadowBox}


\subsection{Introduction}
\label{\detokenize{doc_user/visualiseur:introduction}}
\sphinxAtStartPar
Le module cartographique de cette plateforme permet de présenter des couches de données géographique dans un environnement technique.
Cette interface permet de représenter plusieurs couches géographique mais ne peut pas se substituer à l’utilisation complète d’un outil SIG bureautique type QGIS.

\sphinxAtStartPar
L’interface se présente comme ceci :

\noindent{\hspace*{\fill}\sphinxincludegraphics[width=600\sphinxpxdimen]{{visu_nbr}.png}\hspace*{\fill}}
\begin{itemize}
\item {} 
\sphinxAtStartPar
1 : l’arborescence des couches

\item {} 
\sphinxAtStartPar
2 : recherche d’un lieu

\item {} 
\sphinxAtStartPar
3 : les fonctionnalités

\item {} 
\sphinxAtStartPar
4 : les outils de navigation

\item {} 
\sphinxAtStartPar
5 : les fonds de plans

\end{itemize}

\begin{sphinxadmonition}{note}{Note:}
\sphinxAtStartPar
La donnée peut ne pas s’afficher si elle n’est pas disponible ou alors dans le mauvais référentiel de coordonnée.
\end{sphinxadmonition}


\subsection{La gestion des couches}
\label{\detokenize{doc_user/visualiseur:la-gestion-des-couches}}
\sphinxAtStartPar
Si vous cliquez sur 1, l’arborescence des couches va apparaître et vous pourrez :
\begin{itemize}
\item {} 
\sphinxAtStartPar
rendre visible ou non la couche

\item {} 
\sphinxAtStartPar
modifier l’ordre des couches

\item {} 
\sphinxAtStartPar
modifier l’opacité en pourcentage

\end{itemize}

\noindent{\hspace*{\fill}\sphinxincludegraphics[width=600\sphinxpxdimen]{{visu_couches_details}.png}\hspace*{\fill}}

\sphinxAtStartPar
Dans cet onglet, vous pouvez, à l’aide de ces 3 boutons :

\noindent{\hspace*{\fill}\sphinxincludegraphics[width=300\sphinxpxdimen]{{visu_couches_button}.png}\hspace*{\fill}}
\begin{itemize}
\item {} 
\sphinxAtStartPar
ajouter des données, ce qui ouvrira cette onglet :

\end{itemize}

\noindent{\hspace*{\fill}\sphinxincludegraphics[width=600\sphinxpxdimen]{{visu_cat}.png}\hspace*{\fill}}

\sphinxAtStartPar
Dans cet onglet vous pouvez choisir le catalogue, chercher par mots clés puis ajouter la donnée
\begin{itemize}
\item {} 
\sphinxAtStartPar
ajouter des groupes pour vos données

\item {} 
\sphinxAtStartPar
créer des annotations :

\end{itemize}

\noindent{\hspace*{\fill}\sphinxincludegraphics[width=600\sphinxpxdimen]{{visu_annotation}.png}\hspace*{\fill}}

\sphinxAtStartPar
Lorsque vous cliquez sur une couche, plusieurs fonctions apparaissent :

\noindent{\hspace*{\fill}\sphinxincludegraphics[width=500\sphinxpxdimen]{{visu_couches_barre}.png}\hspace*{\fill}}
\begin{itemize}
\item {} 
\sphinxAtStartPar
zoomer sur la couche

\item {} 
\sphinxAtStartPar
gérer les réglages de la couche :

\end{itemize}

\noindent{\hspace*{\fill}\sphinxincludegraphics[width=600\sphinxpxdimen]{{visu_couches_reglages}.png}\hspace*{\fill}}

\sphinxAtStartPar
Dans ces réglages vous pouvez modifier, les informations, l’affichage, filtrer les champs, le style et les informations attributaires.
\begin{itemize}
\item {} 
\sphinxAtStartPar
filtrer les couches

\item {} 
\sphinxAtStartPar
ouvrir la table attributaire

\item {} 
\sphinxAtStartPar
supprimer la couche

\item {} 
\sphinxAtStartPar
créer un widget

\item {} 
\sphinxAtStartPar
exporter la couche

\item {} 
\sphinxAtStartPar
voir les métadonnées

\end{itemize}

\begin{sphinxadmonition}{note}{Note:}
\sphinxAtStartPar
Les options sont dépendantes de la donnée, elle peuvent ne pas être toutes disponible en fonction de la donnée.
\end{sphinxadmonition}

\sphinxAtStartPar
Pour les fonds de plans, vous cliquez sur l’imagette en bas à gauche et vous pourrez en changer :

\noindent{\hspace*{\fill}\sphinxincludegraphics[width=600\sphinxpxdimen]{{visu_fonds}.png}\hspace*{\fill}}


\subsection{Les fonctionnalités techniques}
\label{\detokenize{doc_user/visualiseur:les-fonctionnalites-techniques}}
\sphinxAtStartPar
Pour ce qui est des différentes fonctionnalités :

\noindent{\hspace*{\fill}\sphinxincludegraphics[width=50\sphinxpxdimen]{{visu_fct}.png}\hspace*{\fill}}

\sphinxAtStartPar
Dans l’ordre, vous pouvez :
\begin{itemize}
\item {} 
\sphinxAtStartPar
imprimer une réalisation :

\end{itemize}

\noindent{\hspace*{\fill}\sphinxincludegraphics[width=600\sphinxpxdimen]{{visu_print}.png}\hspace*{\fill}}

\sphinxAtStartPar
Choisir le titre, le format et si la légende apparaît ou non

\begin{sphinxadmonition}{note}{Note:}
\sphinxAtStartPar
Ne marche pas pour l’instant.
\end{sphinxadmonition}
\begin{itemize}
\item {} 
\sphinxAtStartPar
importer des données

\item {} 
\sphinxAtStartPar
exporter la carte au format WMC

\item {} 
\sphinxAtStartPar
ajouter des données

\item {} 
\sphinxAtStartPar
charger des cartes déjà enrigstrer

\item {} 
\sphinxAtStartPar
mesurer des distances

\item {} 
\sphinxAtStartPar
enrigstrer la carte :

\end{itemize}

\noindent{\hspace*{\fill}\sphinxincludegraphics[width=600\sphinxpxdimen]{{visu_download}.png}\hspace*{\fill}}

\sphinxAtStartPar
Vous pourrez choisir une imagette, le titre, vous pouvez aussi, en cliquant sur le crayon, définir un texte qui sera visible à l’ouverture de la carte.
Pour définir des droits de lecture et d’édition, vous devez sélectionner un groupe et spécifier si il à les droits de lecture ou d’écriture.
L’enregistrement ira dans la page {\hyperref[\detokenize{doc_user/error:application}]{\sphinxcrossref{\DUrole{std,std-ref}{Application}}}}.
\begin{itemize}
\item {} 
\sphinxAtStartPar
voir les réglages

\item {} 
\sphinxAtStartPar
partager la réalisation

\item {} 
\sphinxAtStartPar
voir la documentation

\item {} 
\sphinxAtStartPar
faire le tutoriel

\item {} 
\sphinxAtStartPar
effacer la session

\end{itemize}

\sphinxstepscope


\section{Application \sphinxhyphen{} Cartothèque}
\label{\detokenize{doc_user/application:application-cartotheque}}\label{\detokenize{doc_user/application::doc}}\phantomsection\label{\detokenize{doc_user/application:application}}
\begin{sphinxShadowBox}
\sphinxstyletopictitle{Table des matières}
\begin{itemize}
\item {} 
\sphinxAtStartPar
\phantomsection\label{\detokenize{doc_user/application:id1}}{\hyperref[\detokenize{doc_user/application:introduction}]{\sphinxcrossref{Introduction}}}

\item {} 
\sphinxAtStartPar
\phantomsection\label{\detokenize{doc_user/application:id2}}{\hyperref[\detokenize{doc_user/application:dashboard}]{\sphinxcrossref{Dashboard}}}

\item {} 
\sphinxAtStartPar
\phantomsection\label{\detokenize{doc_user/application:id3}}{\hyperref[\detokenize{doc_user/application:geostory}]{\sphinxcrossref{GeoStory}}}

\end{itemize}
\end{sphinxShadowBox}


\subsection{Introduction}
\label{\detokenize{doc_user/application:introduction}}
\sphinxAtStartPar
La page Application sert de cartotèque en lien avec le {\hyperref[\detokenize{doc_user/visualiseur:id1}]{\sphinxcrossref{\DUrole{std,std-ref}{visualiseur}}}}. Dans cette cartothèque, 4 types de representation sont possibles :
la carte simple avec le visualiseur, le tableau de bord, la GeoStory et le contexte réservé aux administrateurs.

\noindent{\hspace*{\fill}\sphinxincludegraphics[width=600\sphinxpxdimen]{{app_global}.png}\hspace*{\fill}}


\subsection{Dashboard}
\label{\detokenize{doc_user/application:dashboard}}
\noindent{\hspace*{\fill}\sphinxincludegraphics[width=600\sphinxpxdimen]{{app_dashboard}.png}\hspace*{\fill}}

\sphinxAtStartPar
Vous pouvez ajouter différents widget en fonctions des données du catalogue, un tutoriel vous guide directement lorsque vous créer un dashboard.

\begin{sphinxadmonition}{note}{Note:}
\sphinxAtStartPar
Les widgets sont dépéndant de la configuration de la donnée, ils peuvent ne pas être disponible.
\end{sphinxadmonition}

\sphinxAtStartPar
Voici le lien de la documentation officiel pour aller dans le détail :

\sphinxAtStartPar
\sphinxhref{https://docs.mapstore.geosolutionsgroup.com/en/v2024.01.02/user-guide/exploring-dashboards/}{Documentation Mapstore Dashboard}


\subsection{GeoStory}
\label{\detokenize{doc_user/application:geostory}}
\noindent{\hspace*{\fill}\sphinxincludegraphics[width=600\sphinxpxdimen]{{app_geostory}.png}\hspace*{\fill}}

\sphinxAtStartPar
Avec les GeoStories, vous pouvez créer des documents textes en y intégrant des cartes intéractives. La gestions des composant se fait sur la gauche
de l’interface qui sont : les titres, les bannières, les paragraphes, les sections immersives, les geocarrousel, les sections multimedia et les pages web.
Un tutoriel vous guide directement lorsque vous créez une GeoStory.

\sphinxAtStartPar
Voici le lien de la documentation officiel pour aller dans le détail :

\sphinxAtStartPar
\sphinxhref{https://docs.mapstore.geosolutionsgroup.com/en/v2024.01.02/user-guide/exploring-stories/}{Documentation Mapstore GeoStory}

\sphinxstepscope


\section{Import de données}
\label{\detokenize{doc_user/import:import-de-donnees}}\label{\detokenize{doc_user/import::doc}}
\begin{sphinxShadowBox}
\sphinxstyletopictitle{Table des matières}
\begin{itemize}
\item {} 
\sphinxAtStartPar
\phantomsection\label{\detokenize{doc_user/import:id1}}{\hyperref[\detokenize{doc_user/import:introduction}]{\sphinxcrossref{Introduction}}}

\item {} 
\sphinxAtStartPar
\phantomsection\label{\detokenize{doc_user/import:id2}}{\hyperref[\detokenize{doc_user/import:integration-de-shapefile}]{\sphinxcrossref{Intégration de shapefile}}}

\item {} 
\sphinxAtStartPar
\phantomsection\label{\detokenize{doc_user/import:id3}}{\hyperref[\detokenize{doc_user/import:integration-de-csv}]{\sphinxcrossref{Intégration de CSV}}}

\item {} 
\sphinxAtStartPar
\phantomsection\label{\detokenize{doc_user/import:id4}}{\hyperref[\detokenize{doc_user/import:processus-d-integration}]{\sphinxcrossref{Processus d’intégration}}}

\end{itemize}
\end{sphinxShadowBox}


\subsection{Introduction}
\label{\detokenize{doc_user/import:introduction}}
\sphinxAtStartPar
La page Import permet d’intégrer des données de manière simplifié dans le catalogue.

\noindent{\hspace*{\fill}\sphinxincludegraphics[width=600\sphinxpxdimen]{{import}.png}\hspace*{\fill}}

\sphinxAtStartPar
Deux format sont acceptés, le shapefile en zip et le CSV, à une limite de 500 Mo. Vous pouvez ajouter votre donnée, cliquer sur le bouton « J’ai le droit de publier cette donnée »
puis passer à l’étape suivante.


\subsection{Intégration de shapefile}
\label{\detokenize{doc_user/import:integration-de-shapefile}}
\sphinxAtStartPar
La particularité d’un shapefile est la projection :

\noindent{\hspace*{\fill}\sphinxincludegraphics[width=700\sphinxpxdimen]{{import_proj}.png}\hspace*{\fill}}

\sphinxAtStartPar
Pour bien renseigner la donnée, assurez vous que le carré orange qui represente l’emprise de votre donnée est au bon endroit et qu’une projection est renseigné.
De même pour l’encodage, si votre exemple d’objet possède des carractère illisible, vous pouvez changer l’encodage.

\begin{sphinxadmonition}{note}{Note:}
\sphinxAtStartPar
Si pour une donnée, aucune projection n’est valide, veuillez le faire remonter au service informatique.
\end{sphinxadmonition}


\subsection{Intégration de CSV}
\label{\detokenize{doc_user/import:integration-de-csv}}
\sphinxAtStartPar
La particularité d’un CSV est la geométrie :

\noindent{\hspace*{\fill}\sphinxincludegraphics[width=700\sphinxpxdimen]{{import_csv}.png}\hspace*{\fill}}

\sphinxAtStartPar
Pour bien renseigner la donnée, vous pouvez choisir le séparateur de colonne, de texte et aussi renseigner une geométrie ou non. Pour ajouter une geométrie,
il faut obligatoirement un champ latitude et longitude dans le bon format comme sur la photo ci\sphinxhyphen{}dessus.


\subsection{Processus d’intégration}
\label{\detokenize{doc_user/import:processus-d-integration}}
\sphinxAtStartPar
Vous pouvez ensuite ajouter un titre et une description :

\noindent{\hspace*{\fill}\sphinxincludegraphics[width=600\sphinxpxdimen]{{import_shape_titre}.png}\hspace*{\fill}}

\sphinxAtStartPar
Puis ajouter des mots clés en cliquant une fois sur le carré blanc, ce qui fera apparaître les différents mots clés.

\noindent{\hspace*{\fill}\sphinxincludegraphics[width=600\sphinxpxdimen]{{import_shape_keyword}.png}\hspace*{\fill}}

\sphinxAtStartPar
Ensuite vient la date de création, elle se renseigne automatiquement mais vous pouvez la changer si la donnée est antérieur.

\noindent{\hspace*{\fill}\sphinxincludegraphics[width=600\sphinxpxdimen]{{import_shape_time}.png}\hspace*{\fill}}

\sphinxAtStartPar
En dernier, il faut décrire le processus de création de la donnée :

\noindent{\hspace*{\fill}\sphinxincludegraphics[width=600\sphinxpxdimen]{{import_shape_processus}.png}\hspace*{\fill}}

\sphinxAtStartPar
Et vous avez un récapitulatif de votre intégration, cliquez sur « publier » pour intégrer la donnée dans le catalogue.

\noindent{\hspace*{\fill}\sphinxincludegraphics[width=600\sphinxpxdimen]{{import_shape_pub}.png}\hspace*{\fill}}

\sphinxstepscope


\section{Erreurs fréquentes}
\label{\detokenize{doc_user/error:erreurs-frequentes}}\label{\detokenize{doc_user/error::doc}}\phantomsection\label{\detokenize{doc_user/error:application}}
\begin{sphinxShadowBox}
\sphinxstyletopictitle{Table des matières}
\begin{itemize}
\item {} 
\sphinxAtStartPar
\phantomsection\label{\detokenize{doc_user/error:id1}}{\hyperref[\detokenize{doc_user/error:la-couche-ne-charge-pas-dans-le-visualiseur}]{\sphinxcrossref{La couche ne charge pas dans le visualiseur}}}

\item {} 
\sphinxAtStartPar
\phantomsection\label{\detokenize{doc_user/error:id2}}{\hyperref[\detokenize{doc_user/error:pas-de-table-attributaire}]{\sphinxcrossref{Pas de table attributaire}}}

\end{itemize}
\end{sphinxShadowBox}


\subsection{La couche ne charge pas dans le visualiseur}
\label{\detokenize{doc_user/error:la-couche-ne-charge-pas-dans-le-visualiseur}}

\subsection{Pas de table attributaire}
\label{\detokenize{doc_user/error:pas-de-table-attributaire}}
\sphinxstepscope


\chapter{Documentation administrateur}
\label{\detokenize{doc_admin:documentation-administrateur}}\label{\detokenize{doc_admin::doc}}
\sphinxAtStartPar
\sphinxhref{../latex/plateformecartographiqueole.pdf}{Documentation au format PDF}.

\sphinxstepscope


\section{Catalogue \sphinxhyphen{} GeoNetwork}
\label{\detokenize{doc_admin/catalogue:catalogue-geonetwork}}\label{\detokenize{doc_admin/catalogue::doc}}
\begin{sphinxShadowBox}
\sphinxstyletopictitle{Table des matières}
\begin{itemize}
\item {} 
\sphinxAtStartPar
\phantomsection\label{\detokenize{doc_admin/catalogue:id1}}{\hyperref[\detokenize{doc_admin/catalogue:introduction}]{\sphinxcrossref{Introduction}}}

\item {} 
\sphinxAtStartPar
\phantomsection\label{\detokenize{doc_admin/catalogue:id2}}{\hyperref[\detokenize{doc_admin/catalogue:gestion-des-fiches-de-metadonnees}]{\sphinxcrossref{Gestion des fiches de métadonnées}}}

\item {} 
\sphinxAtStartPar
\phantomsection\label{\detokenize{doc_admin/catalogue:id3}}{\hyperref[\detokenize{doc_admin/catalogue:gerer-les-droits-d-acces-aux-fiches-de-metadonnees}]{\sphinxcrossref{Gérer les droits d’accès aux fiches de métadonnées}}}

\item {} 
\sphinxAtStartPar
\phantomsection\label{\detokenize{doc_admin/catalogue:id4}}{\hyperref[\detokenize{doc_admin/catalogue:administration}]{\sphinxcrossref{Administration}}}

\end{itemize}
\end{sphinxShadowBox}


\subsection{Introduction}
\label{\detokenize{doc_admin/catalogue:introduction}}
\sphinxAtStartPar
La technologie utilisé par le catalogue est GeoNetwork, cette documentation à pour but de se repérer rapidement dans l’interface mais n’a pas
pour vocation de remplacer la documentation officiel :
\sphinxurl{https://docs.geonetwork-opensource.org/4.2/user-guide/}

\sphinxAtStartPar
Le GeoNetwork est utilisé comme catalogue CSW (Catalogue Service for the Web) ce qui permet de référencer les métadonnées couplées aux flux de données.

\sphinxAtStartPar
La page principale se compose de 4 composants : la recherche de données, la visualisation, les fiches de métadonnées et l’administration

\noindent{\hspace*{\fill}\sphinxincludegraphics[width=700\sphinxpxdimen]{{cat_barre}.png}\hspace*{\fill}}

\sphinxAtStartPar
La recherche de données est la même que dans le catalogue mais avec l’interface basique de GeoNetwork.
La visualisation renvoie sur le visualisateur qui est MapStore.


\subsection{Gestion des fiches de métadonnées}
\label{\detokenize{doc_admin/catalogue:gestion-des-fiches-de-metadonnees}}
\sphinxAtStartPar
Dans l’onglet « Contribuer » puis « Accueil édition » :

\noindent{\hspace*{\fill}\sphinxincludegraphics[width=700\sphinxpxdimen]{{cat_meta}.png}\hspace*{\fill}}

\sphinxAtStartPar
Cette section fournit une liste des fiches avec les fonctionnalités associées, vous pouvez éditer les fiches, les supprimer,
gérer les annuaires (inutile pour geOrchestra), faire de l’édition en série et gérer les droits d’accès.

\noindent{\hspace*{\fill}\sphinxincludegraphics[width=700\sphinxpxdimen]{{cat_fiche}.png}\hspace*{\fill}}

\sphinxAtStartPar
Dans l’interface d’édition d’une fiche, vous pouvez changez toutes les informations à gauche de l’écran, et ajouter des éléments à droite.
Les ajouts peuvent être des images, des liens ou des ressources qui correspondent à des liens de parentés, des flux OGC ou d’autre.


\subsection{Gérer les droits d’accès aux fiches de métadonnées}
\label{\detokenize{doc_admin/catalogue:gerer-les-droits-d-acces-aux-fiches-de-metadonnees}}\phantomsection\label{\detokenize{doc_admin/catalogue:privileges}}
\sphinxAtStartPar
Vous pouvez restraindre l’accès aux fiches de métadonnée, les fiches sont automatiquement visible pour toutes les organisation de l’infrastructure.
Mais si vous aller dans la fiche de métadonnée que vous voulez modifier, allez dans »Gérer la fiche » puis « Privilèges » et vous pourrez modifier les
différents droit en fonction des organisations :

\noindent{\hspace*{\fill}\sphinxincludegraphics[width=700\sphinxpxdimen]{{cat_gerer}.png}\hspace*{\fill}}

\sphinxAtStartPar
Vous pouvez modifier l’accès à la consultation simple ou encore, la visualisation, le téléchargement, l’édition ou la notification en fonction des organismes.

\noindent{\hspace*{\fill}\sphinxincludegraphics[width=700\sphinxpxdimen]{{cat_privileges}.png}\hspace*{\fill}}


\subsection{Administration}
\label{\detokenize{doc_admin/catalogue:administration}}
\begin{sphinxShadowBox}
\begin{itemize}
\item {} 
\sphinxAtStartPar
\phantomsection\label{\detokenize{doc_admin/catalogue:id5}}{\hyperref[\detokenize{doc_admin/catalogue:metadonnees-et-modeles}]{\sphinxcrossref{Métadonnées et modèles}}}

\item {} 
\sphinxAtStartPar
\phantomsection\label{\detokenize{doc_admin/catalogue:id6}}{\hyperref[\detokenize{doc_admin/catalogue:utilisateur-et-groupe}]{\sphinxcrossref{Utilisateur et groupe}}}

\item {} 
\sphinxAtStartPar
\phantomsection\label{\detokenize{doc_admin/catalogue:id7}}{\hyperref[\detokenize{doc_admin/catalogue:moissonnage}]{\sphinxcrossref{Moissonnage}}}

\item {} 
\sphinxAtStartPar
\phantomsection\label{\detokenize{doc_admin/catalogue:id8}}{\hyperref[\detokenize{doc_admin/catalogue:statistique-et-statut}]{\sphinxcrossref{Statistique et statut}}}

\item {} 
\sphinxAtStartPar
\phantomsection\label{\detokenize{doc_admin/catalogue:id9}}{\hyperref[\detokenize{doc_admin/catalogue:rapports}]{\sphinxcrossref{Rapports}}}

\item {} 
\sphinxAtStartPar
\phantomsection\label{\detokenize{doc_admin/catalogue:id10}}{\hyperref[\detokenize{doc_admin/catalogue:thesaurus}]{\sphinxcrossref{Thésaurus}}}

\item {} 
\sphinxAtStartPar
\phantomsection\label{\detokenize{doc_admin/catalogue:id11}}{\hyperref[\detokenize{doc_admin/catalogue:parametres}]{\sphinxcrossref{Paramètres}}}

\item {} 
\sphinxAtStartPar
\phantomsection\label{\detokenize{doc_admin/catalogue:id12}}{\hyperref[\detokenize{doc_admin/catalogue:outils}]{\sphinxcrossref{Outils}}}

\end{itemize}
\end{sphinxShadowBox}

\sphinxAtStartPar
Pour ce qui est de l’administration, elle est divisé en 8 catégories :

\noindent{\hspace*{\fill}\sphinxincludegraphics[width=700\sphinxpxdimen]{{cat_admin_parties}.png}\hspace*{\fill}}


\subsubsection{Métadonnées et modèles}
\label{\detokenize{doc_admin/catalogue:metadonnees-et-modeles}}\begin{quote}

\sphinxAtStartPar
La page « Métadonnées et modèle » sert à définir les modèles de fiches de métadonnées à utiliser :

\noindent{\hspace*{\fill}\sphinxincludegraphics[width=700\sphinxpxdimen]{{cat_modele}.png}\hspace*{\fill}}
\end{quote}

\sphinxAtStartPar
Les modèles de fiche de métadonnées sont gérées automatiquement par le module d’import de geOrchestra.


\subsubsection{Utilisateur et groupe}
\label{\detokenize{doc_admin/catalogue:utilisateur-et-groupe}}\begin{quote}

\noindent{\hspace*{\fill}\sphinxincludegraphics[width=700\sphinxpxdimen]{{cat_user}.png}\hspace*{\fill}}
\end{quote}

\sphinxAtStartPar
Les utilisateurs et les organisation sont gérés dans la page {\hyperref[\detokenize{doc_admin/utilisateurs:utilisateur}]{\sphinxcrossref{\DUrole{std,std-ref}{Utilisateur}}}}


\subsubsection{Moissonnage}
\label{\detokenize{doc_admin/catalogue:moissonnage}}\begin{quote}

\noindent{\hspace*{\fill}\sphinxincludegraphics[width=700\sphinxpxdimen]{{cat_moisson}.png}\hspace*{\fill}}
\end{quote}

\sphinxAtStartPar
Le moissonnage est très utile car il permet de référencer les fiches de métadonnées d’un autre catalogue sur le GeoNetwork interne.
Il faut connaître la technologie du catalogue que l’on veut référencer, renseigné l’url puis les différentes filtres que l’on veut appliquer.
Il est aussi possible de plannifier le moissonnage.

\sphinxAtStartPar
Les moissonnage sont différents en fonction de la technologie du catalogue cible.
Voici la documentation officiel pour chaque technologie :

\sphinxAtStartPar
\sphinxurl{https://docs.geonetwork-opensource.org/4.2/user-guide/harvesting/}


\subsubsection{Statistique et statut}
\label{\detokenize{doc_admin/catalogue:statistique-et-statut}}\begin{quote}

\noindent{\hspace*{\fill}\sphinxincludegraphics[width=700\sphinxpxdimen]{{cat_stats}.png}\hspace*{\fill}}
\end{quote}

\sphinxAtStartPar
Cette section permet de connaître l’état du système très rapidement. L’analyse des liens scanne tous les liens des métadonnées, le versionnement permet de connaître l’état
d’une métadonnée précise.


\subsubsection{Rapports}
\label{\detokenize{doc_admin/catalogue:rapports}}\begin{quote}

\noindent{\hspace*{\fill}\sphinxincludegraphics[width=700\sphinxpxdimen]{{cat_rapport}.png}\hspace*{\fill}}
\end{quote}

\sphinxAtStartPar
La partie rapport permet de créer des rapports très rapidement :
\begin{itemize}
\item {} 
\sphinxAtStartPar
sur la mise à jour des fiches

\item {} 
\sphinxAtStartPar
sur les fiches stockées en interne

\item {} 
\sphinxAtStartPar
sur l’ajout de fichier dans les fiches

\item {} 
\sphinxAtStartPar
sur l’historique des fiches

\item {} 
\sphinxAtStartPar
sur les accès utilisateurs

\end{itemize}


\subsubsection{Thésaurus}
\label{\detokenize{doc_admin/catalogue:thesaurus}}\begin{quote}

\noindent{\hspace*{\fill}\sphinxincludegraphics[width=700\sphinxpxdimen]{{cat_thes}.png}\hspace*{\fill}}
\end{quote}

\sphinxAtStartPar
Le thésaurus est le dictionnaire à mots clés, il définit les mots clés que vous pouvez utiliser pour vos métadonnées.


\subsubsection{Paramètres}
\label{\detokenize{doc_admin/catalogue:parametres}}\begin{quote}

\noindent{\hspace*{\fill}\sphinxincludegraphics[width=700\sphinxpxdimen]{{cat_param}.png}\hspace*{\fill}}
\end{quote}

\sphinxAtStartPar
Dans cet onglet se trouve les paramètres pour la configuration système dont voici la documentation en details :

\sphinxAtStartPar
\sphinxurl{https://docs.geonetwork-opensource.org/4.2/fr/administrator-guide/configuring-the-catalog/system-configuration/}

\sphinxAtStartPar
Sur cette partie se trouve aussi les paramètre pour changer l’interface utilisateur, changer le style, ajouter des logos, gérer les différents catalogues moissonnés,
gérer les différentes langues, activer et tester le CSW, ajouter des serveurs cartographiques type GeoServer et ajouter des pages statiques.


\subsubsection{Outils}
\label{\detokenize{doc_admin/catalogue:outils}}\begin{quote}

\noindent{\hspace*{\fill}\sphinxincludegraphics[width=700\sphinxpxdimen]{{cat_outil}.png}\hspace*{\fill}}
\end{quote}

\sphinxAtStartPar
Cette partie permet d’inéragir avec les index d’elasticsearch qui est le moteur de recherche derrière GeoNetwork. Cela permet de relancer l’indexation des données si
l’on pense que des problèmes ont été réglé. ll ne faut globalement pas cliquer sur ces boutons.

\sphinxstepscope


\section{MapStore}
\label{\detokenize{doc_admin/visualiseur:mapstore}}\label{\detokenize{doc_admin/visualiseur::doc}}
\begin{sphinxShadowBox}
\sphinxstyletopictitle{Table des matières}
\begin{itemize}
\item {} 
\sphinxAtStartPar
\phantomsection\label{\detokenize{doc_admin/visualiseur:id1}}{\hyperref[\detokenize{doc_admin/visualiseur:introduction}]{\sphinxcrossref{Introduction}}}

\item {} 
\sphinxAtStartPar
\phantomsection\label{\detokenize{doc_admin/visualiseur:id2}}{\hyperref[\detokenize{doc_admin/visualiseur:les-contextes}]{\sphinxcrossref{Les contextes}}}

\end{itemize}
\end{sphinxShadowBox}


\subsection{Introduction}
\label{\detokenize{doc_admin/visualiseur:introduction}}
\sphinxAtStartPar
La partie administrateur du module Mapstore ajoute peu de fonctionnalités, la seule fonction en plus est la création de contextes.


\subsection{Les contextes}
\label{\detokenize{doc_admin/visualiseur:les-contextes}}
\sphinxAtStartPar
Les contextes permettent de créer des cartes en choisissant l’interface finale pour par exemple rendre la carte plus abordable et moins technique.
Par exemple avec ce contexte qui ne presente que le bouton accueil, télécharger et importer :

\noindent{\hspace*{\fill}\sphinxincludegraphics[width=600\sphinxpxdimen]{{mapstore_contexte}.png}\hspace*{\fill}}

\sphinxAtStartPar
Un tutoriel est automatiquement lancé lorsque vous créer un contexte et vous guide pas à pas dans la création
Il faut commencer par choisir un titre, une description, ajouter les données que l’on veut afficher,
et choisir les fonctions :

\noindent{\hspace*{\fill}\sphinxincludegraphics[width=600\sphinxpxdimen]{{mapstore_plugins}.png}\hspace*{\fill}}

\sphinxAtStartPar
Les fonctions à choisir sont explicite et facile à comprendre.
Enfin il reste à enregister le contexte pour le rendre disponible aux groupes que l’on veut.

\sphinxstepscope


\section{Services \sphinxhyphen{} GeoServer}
\label{\detokenize{doc_admin/services:services-geoserver}}\label{\detokenize{doc_admin/services::doc}}
\begin{sphinxShadowBox}
\sphinxstyletopictitle{Table des matières}
\begin{itemize}
\item {} 
\sphinxAtStartPar
\phantomsection\label{\detokenize{doc_admin/services:id1}}{\hyperref[\detokenize{doc_admin/services:introduction}]{\sphinxcrossref{Introduction}}}

\item {} 
\sphinxAtStartPar
\phantomsection\label{\detokenize{doc_admin/services:id2}}{\hyperref[\detokenize{doc_admin/services:les-donnees-stockees-en-interne}]{\sphinxcrossref{Les données stockées en interne}}}

\item {} 
\sphinxAtStartPar
\phantomsection\label{\detokenize{doc_admin/services:id3}}{\hyperref[\detokenize{doc_admin/services:la-diffusion-des-donnees-avec-les-flux-ogc}]{\sphinxcrossref{La diffusion des données avec les flux OGC}}}

\item {} 
\sphinxAtStartPar
\phantomsection\label{\detokenize{doc_admin/services:id4}}{\hyperref[\detokenize{doc_admin/services:la-restriction-d-acces-aux-donnees}]{\sphinxcrossref{La restriction d’accès aux données}}}

\end{itemize}
\end{sphinxShadowBox}


\subsection{Introduction}
\label{\detokenize{doc_admin/services:introduction}}
\sphinxAtStartPar
Cette page est l’interface de GeoServer, le GeoServer est la technologie qui permet de diffuser les données stockées en interne.
Voici la documentation officiel :
\sphinxurl{https://docs.geoserver.org/}

\sphinxAtStartPar
Vous n’avez normalement pas à intervenir dans cette page, à part pour des changements sur la configuration des différents flux.


\subsection{Les données stockées en interne}
\label{\detokenize{doc_admin/services:les-donnees-stockees-en-interne}}
\sphinxAtStartPar
GeoServer est directement connecté à une base de données PostGIS et diffuse les données internes :

\noindent{\hspace*{\fill}\sphinxincludegraphics[width=700\sphinxpxdimen]{{geos_interface}.png}\hspace*{\fill}}

\sphinxAtStartPar
Les données sont organisées en « espaces de travail » qui prend le nom de l’organisation à qui appartient la donnée, puis est reliée à un entrepot, qui est l’emplacement
dans la base de données de là où est stocké la donnée. Cette organisation et ce stockage se fait automatiquement avec le module d’importation de geOrchestra.


\subsection{La diffusion des données avec les flux OGC}
\label{\detokenize{doc_admin/services:la-diffusion-des-donnees-avec-les-flux-ogc}}
\sphinxAtStartPar
Lorsqu’une donnée est intégrée dans geOrchestra via le module d’intégration, deux types de services sont créés : un flux WMS (Web Map Service)
et un flux WFS (Web Feature Service).

\sphinxAtStartPar
WMS (Web Map Service) : Ce service permet de représenter la donnée sous forme de cartes raster (images générées à partir des données géospatiales).
Les couches WMS sont souvent utilisées pour l’affichage dans des visualiseurs cartographiques, car elles sont légères et rapides à charger.

\sphinxAtStartPar
WFS (Web Feature Service) : Ce service permet d’accéder aux données vectorielles, offrant la possibilité de requêter et de manipuler directement
les entités géospatiales (points, lignes, polygones). Le WFS est essentiel pour effectuer des requêtes sur les objets géospatiaux et obtenir des informations
précises sur ces entités.

\sphinxAtStartPar
Ces services sont conformes aux normes européennes et permettent une interopérabilité entre différents modules et systèmes. Lorsqu’une
donnée apparaît sur le visualiseur, elle est généralement issue du flux WMS pour des raisons de performance, car les données raster sont plus
rapides et moins gourmandes en ressources. Toutefois, le flux WFS est crucial pour permettre des interactions plus détaillées, telles que des requêtes sur les entités.
Vous pouvez configurer ces flux dans GeoServer en accédant aux paramètres du service, par exemple pour définir les autorisations ou activer/désactiver la
transformation du système de coordonnées de référence (CRS). Cela vous permet de contrôler précisément comment les données sont diffusées et utilisées au sein de la plateforme.


\subsection{La restriction d’accès aux données}
\label{\detokenize{doc_admin/services:la-restriction-d-acces-aux-donnees}}
\sphinxAtStartPar
La manipulation des droits se fait normalement dans l’onglet {\hyperref[\detokenize{doc_admin/utilisateurs:utilisateur}]{\sphinxcrossref{\DUrole{std,std-ref}{Utilisateur}}}}. La seule chose qui ne peut pas se faire dans la page {\hyperref[\detokenize{doc_admin/utilisateurs:utilisateur}]{\sphinxcrossref{\DUrole{std,std-ref}{Utilisateur}}}}
est la restriction d’accès aux données, qui se fait, pour les métadonnées dans l’onglet {\hyperref[\detokenize{doc_admin/catalogue:privileges}]{\sphinxcrossref{\DUrole{std,std-ref}{privilèges}}}}

\sphinxAtStartPar
Par défaut, toutes les données et les ressources dans GeoServer sont accessibles à tous les utilisateurs.
Pour gérer l’accès, des restrictions spécifiques peuvent être appliquées par la suite.

\noindent{\hspace*{\fill}\sphinxincludegraphics[width=700\sphinxpxdimen]{{geos_secu}.png}\hspace*{\fill}}

\sphinxAtStartPar
\sphinxstylestrong{Définir l’espace de travail} : Spécifiez l’espace de travail concerné. Dans cet exemple, nous utilisons l’espace de travail « ole »,
qui contient les données intégrées par l’Office de l’eau.

\sphinxAtStartPar
\sphinxstylestrong{Cibler les données} : Indiquez les données que vous souhaitez restreindre. Pour cibler toutes les données, vous pouvez utiliser le symbole « * ».

\sphinxAtStartPar
\sphinxstylestrong{Type d’accès} : Sélectionnez le type d’accès à restreindre. Dans cet exemple, nous choisissons l’accès en lecture.

\sphinxAtStartPar
\sphinxstylestrong{Définir les rôles} : Précisez les rôles qui auront accès à cette sécurité. Ici, nous incluons les rôles « SASPE » et « OREBA ».

\sphinxAtStartPar
Avec cet exemple, seulement les utilisateurs qui possèdent le rôles « SASPE » et/ou « OREBA » peuvent visualiser les flux des données de l’espace de travail « ole »
qui correspond aux données de l’Office de l’eau Réunion.

\sphinxAtStartPar
Ce qui en resulte par cette interface et les règles suivantes :
\begin{itemize}
\item {} 
\sphinxAtStartPar
toutes les données sont lisible par tous les groupes, mais par dessus vient s’ajouter :

\item {} 
\sphinxAtStartPar
les données de l’Office de l’eau Réunion ne sont lisible que par les utilisateurs qui sont dans les groupes OREBA et/ou SASPE

\end{itemize}

\noindent{\hspace*{\fill}\sphinxincludegraphics[width=700\sphinxpxdimen]{{geos_result}.png}\hspace*{\fill}}

\sphinxstepscope


\section{Utilsateur \sphinxhyphen{} console admin}
\label{\detokenize{doc_admin/utilisateurs:utilsateur-console-admin}}\label{\detokenize{doc_admin/utilisateurs::doc}}\phantomsection\label{\detokenize{doc_admin/utilisateurs:utilisateur}}
\begin{sphinxShadowBox}
\sphinxstyletopictitle{Table des matières}
\begin{itemize}
\item {} 
\sphinxAtStartPar
\phantomsection\label{\detokenize{doc_admin/utilisateurs:id1}}{\hyperref[\detokenize{doc_admin/utilisateurs:introduction}]{\sphinxcrossref{Introduction}}}

\item {} 
\sphinxAtStartPar
\phantomsection\label{\detokenize{doc_admin/utilisateurs:id2}}{\hyperref[\detokenize{doc_admin/utilisateurs:utilisateurs}]{\sphinxcrossref{Utilisateurs}}}

\item {} 
\sphinxAtStartPar
\phantomsection\label{\detokenize{doc_admin/utilisateurs:id3}}{\hyperref[\detokenize{doc_admin/utilisateurs:organismes}]{\sphinxcrossref{Organismes}}}

\item {} 
\sphinxAtStartPar
\phantomsection\label{\detokenize{doc_admin/utilisateurs:id4}}{\hyperref[\detokenize{doc_admin/utilisateurs:roles}]{\sphinxcrossref{Rôles}}}

\item {} 
\sphinxAtStartPar
\phantomsection\label{\detokenize{doc_admin/utilisateurs:id5}}{\hyperref[\detokenize{doc_admin/utilisateurs:autres}]{\sphinxcrossref{Autres :}}}

\end{itemize}
\end{sphinxShadowBox}


\subsection{Introduction}
\label{\detokenize{doc_admin/utilisateurs:introduction}}
\sphinxAtStartPar
La console d’admin sert à gérer les utilisateurs, les droits, et voir les activités des utilisateurs.
Chaque utilisateur est reliée à une organisation, et les accès sont gérés par des rôles qui sont prédéfinis.
Vous pouvez ajouter, modifier ou supprimer des rôles en fonction des utilisateurs.

\sphinxAtStartPar
La première page est le « Tableau de bord » avec le récapitulatif des actions passées, les utilisateurs en attente de validation,
et qui s’est connecté sur la journée.

\noindent{\hspace*{\fill}\sphinxincludegraphics[width=700\sphinxpxdimen]{{user_dashboard}.png}\hspace*{\fill}}


\subsection{Utilisateurs}
\label{\detokenize{doc_admin/utilisateurs:utilisateurs}}
\sphinxAtStartPar
Cette section permet de voir la liste des utilisateurs et leurs informations :

\noindent{\hspace*{\fill}\sphinxincludegraphics[width=700\sphinxpxdimen]{{user_user}.png}\hspace*{\fill}}

\sphinxAtStartPar
En cliquant sur un utilisateur vous pourrez modifier ses caractéristiques :

\noindent{\hspace*{\fill}\sphinxincludegraphics[width=700\sphinxpxdimen]{{user_user_user}.png}\hspace*{\fill}}


\subsection{Organismes}
\label{\detokenize{doc_admin/utilisateurs:organismes}}
\sphinxAtStartPar
Les utilisateurs sont obligatoirement rattachés à une organisation :

\noindent{\hspace*{\fill}\sphinxincludegraphics[width=700\sphinxpxdimen]{{user_orga}.png}\hspace*{\fill}}

\sphinxAtStartPar
Si vous cliquez sur une organisation, vous pouvez modifier ses informations ainsi que ses membres :

\noindent{\hspace*{\fill}\sphinxincludegraphics[width=700\sphinxpxdimen]{{user_orga_orga}.png}\hspace*{\fill}}


\subsection{Rôles}
\label{\detokenize{doc_admin/utilisateurs:roles}}
\sphinxAtStartPar
Les rôles permettent de regrouper les utilisateurs et de leur donner des accès et droits :

\noindent{\hspace*{\fill}\sphinxincludegraphics[width=700\sphinxpxdimen]{{user_role}.png}\hspace*{\fill}}

\sphinxAtStartPar
Certain rôles définissent des accès particulier et il est possible de créer des groupes en plus pour regrouper des utilisateurs entre eux.

\sphinxAtStartPar
Les rôles principaux sont :
\begin{itemize}
\item {} 
\sphinxAtStartPar
\sphinxstylestrong{SUPERUSER} : accès à la console d’admin

\item {} 
\sphinxAtStartPar
\sphinxstylestrong{ADMINISTRATOR} : permet d’accéder au module admin de GeoServer

\item {} 
\sphinxAtStartPar
\sphinxstylestrong{GN\_ADMIN} : permet d’accéder au GeoNetwork qui est le module admin du catalogue

\item {} 
\sphinxAtStartPar
\sphinxstylestrong{GN\_EDITOR} : permet d’éditer les fiches dans GeoNetwork

\item {} 
\sphinxAtStartPar
\sphinxstylestrong{GN\_REVIEWER} : permet de publier des données à la main dans GeoNetwork

\item {} 
\sphinxAtStartPar
\sphinxstylestrong{MAPSTORE\_ADMIN} : permet d’accéder au module admin de Mapstore, donc à la création de contextes

\item {} 
\sphinxAtStartPar
\sphinxstylestrong{USER} : permet de se log dans geOrchestra et d’enregister des cartes, dashboards et GeoStories dans Mapstore

\item {} 
\sphinxAtStartPar
\sphinxstylestrong{REFERENT} : permet de modifier les informations de son organisme

\item {} 
\sphinxAtStartPar
\sphinxstylestrong{IMPORT} : donne accès au module d’import de données dans geOrchestra

\end{itemize}

\sphinxAtStartPar
On peut très bien ajouter des rôles, par exemple les rôles OREBA et SASPE, il faut ajouter le rôle OREBA et SASPE au utilisateurs qui appartiennent à ces services.
Puis si l’on veut partager des cartes dans Mapstore et ne les rendre visible ou éditable seulement par un service, il faudra spécifier le groupe en question.

\sphinxAtStartPar
Ou encore créer des groupe pour restraindre l’accès à certaines données avec GeoServer.


\subsection{Autres :}
\label{\detokenize{doc_admin/utilisateurs:autres}}\begin{itemize}
\item {} 
\sphinxAtStartPar
\sphinxstylestrong{Délégation} : sert à donner, à un utilisateur, le droit de promouvoir un autre utilisateur avec des rôles spécifiques

\item {} 
\sphinxAtStartPar
\sphinxstylestrong{Statistique} : permet de voir le nombre de requêtes par jour, et les couches les plus consultées

\item {} 
\sphinxAtStartPar
\sphinxstylestrong{Journaux} : permet d’accèder à l’historique des actions de la console d’admin

\end{itemize}

\sphinxstepscope


\section{Application}
\label{\detokenize{doc_admin/analytics:application}}\label{\detokenize{doc_admin/analytics::doc}}
\sphinxAtStartPar
Le module analytics permet d’analyser les flux OGC donc les données issues des différents service web OGC de GeoServer :

\noindent{\hspace*{\fill}\sphinxincludegraphics[width=700\sphinxpxdimen]{{ana}.png}\hspace*{\fill}}

\sphinxAtStartPar
L’interface permet de connaître :
\begin{itemize}
\item {} 
\sphinxAtStartPar
le service web, le titre de la couche et la requête

\item {} 
\sphinxAtStartPar
l’utilisateur et le nombre de requêtes

\item {} 
\sphinxAtStartPar
l’organisation et le nombre de requêtes

\end{itemize}

\sphinxstepscope


\chapter{Documentation d’installation}
\label{\detokenize{doc_instal:documentation-d-installation}}\label{\detokenize{doc_instal::doc}}
\sphinxAtStartPar
\sphinxhref{../latex/plateformecartographiqueole.pdf}{Documentation au format PDF}.

\sphinxstepscope


\section{Installation}
\label{\detokenize{doc_instal/installation:installation}}\label{\detokenize{doc_instal/installation::doc}}
\begin{sphinxShadowBox}
\sphinxstyletopictitle{Table des matières}
\begin{itemize}
\item {} 
\sphinxAtStartPar
\phantomsection\label{\detokenize{doc_instal/installation:id1}}{\hyperref[\detokenize{doc_instal/installation:introduction}]{\sphinxcrossref{Introduction}}}

\item {} 
\sphinxAtStartPar
\phantomsection\label{\detokenize{doc_instal/installation:id2}}{\hyperref[\detokenize{doc_instal/installation:ansible}]{\sphinxcrossref{Ansible}}}

\item {} 
\sphinxAtStartPar
\phantomsection\label{\detokenize{doc_instal/installation:id3}}{\hyperref[\detokenize{doc_instal/installation:serveur-web}]{\sphinxcrossref{Serveur web}}}

\item {} 
\sphinxAtStartPar
\phantomsection\label{\detokenize{doc_instal/installation:id4}}{\hyperref[\detokenize{doc_instal/installation:script-de-personnalisation}]{\sphinxcrossref{Script de personnalisation}}}

\end{itemize}
\end{sphinxShadowBox}


\subsection{Introduction}
\label{\detokenize{doc_instal/installation:introduction}}
\sphinxAtStartPar
Georchestra est une IDG qui intègre plusieurs modules et donc plusieurs technologies, il y’a plusieurs façon d’installer cette infrastructure
\begin{itemize}
\item {} 
\sphinxAtStartPar
par docker

\item {} 
\sphinxAtStartPar
par Ansible

\item {} 
\sphinxAtStartPar
à la main

\end{itemize}

\sphinxAtStartPar
Le choix pour l’Office de l’eau Réunion à été Ansible qui permet d’installer des paquets Debians rapidement et automatiquement.

\sphinxAtStartPar
Le lien pour le github de georchestra est le suivant : \sphinxurl{https://github.com/georchestra}


\subsection{Ansible}
\label{\detokenize{doc_instal/installation:ansible}}
\sphinxAtStartPar
Prérequis :
\begin{itemize}
\item {} 
\sphinxAtStartPar
Debian Bookworm (12.x) VM

\item {} 
\sphinxAtStartPar
Ansible : sudo apt install ansible

\end{itemize}

\begin{sphinxVerbatim}[commandchars=\\\{\}]
sudo\PYG{+w}{ }apt\PYG{+w}{ }install\PYG{+w}{ }ansible
\end{sphinxVerbatim}
\begin{itemize}
\item {} 
\sphinxAtStartPar
Java 17 :

\end{itemize}

\begin{sphinxVerbatim}[commandchars=\\\{\}]
sudo\PYG{+w}{ }apt\PYG{+w}{ }install\PYG{+w}{ }openjdk\PYGZhy{}17\PYGZhy{}jdk
\end{sphinxVerbatim}
\begin{itemize}
\item {} 
\sphinxAtStartPar
Clone the source, le code est issue du repo « ansible » de georchestra :

\end{itemize}

\begin{sphinxVerbatim}[commandchars=\\\{\}]
sudo\PYG{+w}{ }git\PYG{+w}{ }clone\PYG{+w}{ }https://github.com/ToffoluttiVittorio/ansible.git
\end{sphinxVerbatim}
\begin{itemize}
\item {} 
\sphinxAtStartPar
Installer les rôles de GeoNetwork :

\end{itemize}

\begin{sphinxVerbatim}[commandchars=\\\{\}]
sudo\PYG{+w}{ }ansible\PYGZhy{}galaxy\PYG{+w}{ }install\PYG{+w}{ }\PYGZhy{}r\PYG{+w}{ }requirements.yaml
sudo\PYG{+w}{ }chmod\PYG{+w}{ }\PYGZhy{}777\PYG{+w}{ }chemin/vers/ansible/roles/
\end{sphinxVerbatim}
\begin{itemize}
\item {} 
\sphinxAtStartPar
Ajouter les clés manquantes :

\end{itemize}

\begin{sphinxVerbatim}[commandchars=\\\{\}]
sudo\PYG{+w}{ }apt\PYGZhy{}key\PYG{+w}{ }adv\PYG{+w}{ }\PYGZhy{}\PYGZhy{}keyserver\PYG{+w}{ }keyserver.ubuntu.com\PYG{+w}{ }\PYGZhy{}\PYGZhy{}recv\PYGZhy{}keys\PYG{+w}{ }0E98404D386FA1D9
sudo\PYG{+w}{ }apt\PYGZhy{}key\PYG{+w}{ }adv\PYG{+w}{ }\PYGZhy{}\PYGZhy{}keyserver\PYG{+w}{ }keyserver.ubuntu.com\PYG{+w}{ }\PYGZhy{}\PYGZhy{}recv\PYGZhy{}keys\PYG{+w}{ }6ED0E7B82643E131
sudo\PYG{+w}{ }apt\PYGZhy{}key\PYG{+w}{ }adv\PYG{+w}{ }\PYGZhy{}\PYGZhy{}keyserver\PYG{+w}{ }keyserver.ubuntu.com\PYG{+w}{ }\PYGZhy{}\PYGZhy{}recv\PYGZhy{}keys\PYG{+w}{ }605C66F00D6C9793
\end{sphinxVerbatim}
\begin{itemize}
\item {} 
\sphinxAtStartPar
Run the playbook for ansible :

\end{itemize}

\begin{sphinxVerbatim}[commandchars=\\\{\}]
sudo\PYG{+w}{ }ansible\PYGZhy{}playbook\PYG{+w}{ }playbooks/georchestra.yml
\end{sphinxVerbatim}

\sphinxAtStartPar
L’installation de l’infrastructure de geOrchestra et faite, il reste à installer un serveur de mail et les scripts de personnalisation pour avoir
l’application fonctionnel et complète pour l’Office de l’eau Réunion.


\subsection{Serveur web}
\label{\detokenize{doc_instal/installation:serveur-web}}
\sphinxAtStartPar
Pour le serveur web, pour l’instant un postfix est installé :

\begin{sphinxVerbatim}[commandchars=\\\{\}]
sudo\PYG{+w}{ }apt\PYG{+w}{ }install\PYG{+w}{ }postfix
sudo\PYG{+w}{ }systemctl\PYG{+w}{ }start\PYG{+w}{ }postfix.service
\end{sphinxVerbatim}

\sphinxAtStartPar
avec cette configuration dans le fichier /etc/postfix/main.cf :

\begin{sphinxVerbatim}[commandchars=\\\{\}]
\PYG{n+nv}{smtpd\PYGZus{}relay\PYGZus{}restrictions}\PYG{+w}{ }\PYG{o}{=}\PYG{+w}{ }permit\PYGZus{}mynetworks\PYG{+w}{ }permit\PYGZus{}sasl\PYGZus{}authenticated\PYG{+w}{ }defer\PYGZus{}unauth\PYGZus{}destination
\PYG{n+nv}{myhostname}\PYG{+w}{ }\PYG{o}{=}\PYG{+w}{ }Ansible\PYGZhy{}42.myguest.virtualbox.org
\PYG{n+nv}{alias\PYGZus{}maps}\PYG{+w}{ }\PYG{o}{=}\PYG{+w}{ }hash:/etc/aliases
\PYG{n+nv}{alias\PYGZus{}database}\PYG{+w}{ }\PYG{o}{=}\PYG{+w}{ }hash:/etc/aliases
\PYG{n+nv}{mydestination}\PYG{+w}{ }\PYG{o}{=}\PYG{+w}{ }\PYG{n+nv}{\PYGZdl{}myhostname},\PYG{+w}{ }localhost,\PYG{+w}{ }localhost.\PYG{n+nv}{\PYGZdl{}mydomain},\PYG{+w}{ }mail.\PYG{n+nv}{\PYGZdl{}mydomain},\PYG{+w}{ }www.\PYG{n+nv}{\PYGZdl{}mydomain},\PYG{+w}{ }localhost,\PYG{+w}{ }\PYG{n+nv}{\PYGZdl{}mydomain}
\PYG{n+nv}{relayhost}\PYG{+w}{ }\PYG{o}{=}
\PYG{n+nv}{mynetworks}\PYG{+w}{ }\PYG{o}{=}\PYG{+w}{ }\PYG{l+m}{127}.0.0.0/8\PYG{+w}{ }\PYG{o}{[}::ffff:127.0.0.0\PYG{o}{]}/104\PYG{+w}{ }\PYG{o}{[}::1\PYG{o}{]}/128
\PYG{n+nv}{mailbox\PYGZus{}size\PYGZus{}limit}\PYG{+w}{ }\PYG{o}{=}\PYG{+w}{ }\PYG{l+m}{0}
\PYG{n+nv}{recipient\PYGZus{}delimiter}\PYG{+w}{ }\PYG{o}{=}\PYG{+w}{ }+
\PYG{n+nv}{inet\PYGZus{}interfaces}\PYG{+w}{ }\PYG{o}{=}\PYG{+w}{ }all
\PYG{n+nv}{inet\PYGZus{}protocols}\PYG{+w}{ }\PYG{o}{=}\PYG{+w}{ }all
\end{sphinxVerbatim}


\subsection{Script de personnalisation}
\label{\detokenize{doc_instal/installation:script-de-personnalisation}}
\sphinxAtStartPar
Les scripts de personnalisation servent à ajouter les spécifications pour l’Office de l’eau Réunion sans directement changer le code d’installation.

\sphinxAtStartPar
Il y’a trois script bash qui modifient les logos, couleurs et référentiel de coordonée dans le dossier « Configuration » :

\begin{sphinxVerbatim}[commandchars=\\\{\}]
\PYG{c+ch}{\PYGZsh{}!/bin/bash}

\PYG{c+c1}{\PYGZsh{} Mise à jour du fichier de propriétés pour le changement de langue}
\PYG{n+nb}{echo}\PYG{+w}{ }\PYG{l+s+s2}{\PYGZdq{}Remplacement de \PYGZsq{}language=en\PYGZsq{} par \PYGZsq{}language=fr\PYGZsq{} dans le fichier de propriétés...\PYGZdq{}}
sed\PYG{+w}{ }\PYGZhy{}i\PYG{+w}{ }\PYG{l+s+s1}{\PYGZsq{}s/language=en/language=fr/\PYGZsq{}}\PYG{+w}{ }/etc/georchestra/default.properties
\PYG{n+nb}{echo}\PYG{+w}{ }\PYG{l+s+s2}{\PYGZdq{}Mise à jour du fichier de propriétés terminée.\PYGZdq{}}

\PYG{c+c1}{\PYGZsh{} Mise à jour du fichier de propriétés pour le changement d\PYGZsq{}URL du logo}
\PYG{n+nb}{echo}\PYG{+w}{ }\PYG{l+s+s2}{\PYGZdq{}Remplacement de l\PYGZsq{}URL du logo dans le fichier de propriétés...\PYGZdq{}}
sed\PYG{+w}{ }\PYGZhy{}i\PYG{+w}{ }\PYG{l+s+s1}{\PYGZsq{}s|logoUrl=https://www.georchestra.org/public/georchestra\PYGZhy{}logo.svg|logoUrl=https://raw.githubusercontent.com/ToffoluttiVittorio/ansible/master/Configuration/georchestra\PYGZhy{}logo.svg|\PYGZsq{}}\PYG{+w}{ }/etc/georchestra/default.properties
\PYG{n+nb}{echo}\PYG{+w}{ }\PYG{l+s+s2}{\PYGZdq{}Mise à jour de l\PYGZsq{}URL du logo terminée.\PYGZdq{}}

\PYG{c+c1}{\PYGZsh{} Remplacement de l\PYGZsq{}URL de la feuille de style commentée dans le fichier de propriétés}
\PYG{n+nb}{echo}\PYG{+w}{ }\PYG{l+s+s2}{\PYGZdq{}Remplacement de l\PYGZsq{}URL de la feuille de style commentée dans le fichier de propriétés...\PYGZdq{}}
sed\PYG{+w}{ }\PYGZhy{}i\PYG{+w}{ }\PYG{l+s+s1}{\PYGZsq{}s|\PYGZsh{} georchestraStylesheet=http://my\PYGZhy{}domain\PYGZhy{}name/stylesheet.css|georchestraStylesheet=./stylesheet.css|\PYGZsq{}}\PYG{+w}{ }/etc/georchestra/default.properties
\PYG{n+nb}{echo}\PYG{+w}{ }\PYG{l+s+s2}{\PYGZdq{}Mise à jour de l\PYGZsq{}URL de la feuille de style terminée.\PYGZdq{}}

\PYG{c+c1}{\PYGZsh{} Activation des analytics dans le fichier de propriétés}
\PYG{n+nb}{echo}\PYG{+w}{ }\PYG{l+s+s2}{\PYGZdq{}Activation des analytics dans le fichier de propriétés...\PYGZdq{}}
sed\PYG{+w}{ }\PYGZhy{}i\PYG{+w}{ }\PYG{l+s+s1}{\PYGZsq{}s/analyticsEnabled=false/analyticsEnabled=true/\PYGZsq{}}\PYG{+w}{ }/etc/georchestra/default.properties
\PYG{n+nb}{echo}\PYG{+w}{ }\PYG{l+s+s2}{\PYGZdq{}Activation des analytics terminée.\PYGZdq{}}

\PYG{c+c1}{\PYGZsh{} Mise à jour de la timezone dans le fichier de propriétés}
\PYG{n+nb}{echo}\PYG{+w}{ }\PYG{l+s+s2}{\PYGZdq{}Remplacement de la timezone dans le fichier de propriétés...\PYGZdq{}}
sed\PYG{+w}{ }\PYGZhy{}i\PYG{+w}{ }\PYG{l+s+s1}{\PYGZsq{}s|\PYGZsh{}localTimezone=Europe/Paris|localTimezone=Indian/Reunion|\PYGZsq{}}\PYG{+w}{ }/etc/georchestra/analytics/analytics.properties
\PYG{n+nb}{echo}\PYG{+w}{ }\PYG{l+s+s2}{\PYGZdq{}Mise à jour de la timezone terminée.\PYGZdq{}}

\PYG{c+c1}{\PYGZsh{} Traduction des valeurs de orgTypeValues dans le fichier de propriétés}
\PYG{n+nb}{echo}\PYG{+w}{ }\PYG{l+s+s2}{\PYGZdq{}Remplacement des valeurs de orgTypeValues par leur traduction en français...\PYGZdq{}}
sed\PYG{+w}{ }\PYGZhy{}i\PYG{+w}{ }\PYG{l+s+s1}{\PYGZsq{}s/orgTypeValues=Association,Company,NGO,Individual,Other/orgTypeValues=Association,Entreprise,ONG,Individu,Autre/\PYGZsq{}}\PYG{+w}{ }/etc/georchestra/console/console.properties
\PYG{n+nb}{echo}\PYG{+w}{ }\PYG{l+s+s2}{\PYGZdq{}Traduction des valeurs de orgTypeValues terminée.\PYGZdq{}}
\end{sphinxVerbatim}

\begin{sphinxVerbatim}[commandchars=\\\{\}]
\PYG{c+ch}{\PYGZsh{}!/bin/bash}

\PYG{c+c1}{\PYGZsh{} Vérifier si la nouvelle projection existe déjà dans le fichier JSON et ajouter si elle n\PYGZsq{}existe pas}
\PYG{n+nb}{echo}\PYG{+w}{ }\PYG{l+s+s2}{\PYGZdq{}Vérification et ajout de la nouvelle entrée à la liste \PYGZsq{}projections\PYGZsq{} dans le fichier JSON...\PYGZdq{}}
\PYG{k}{if}\PYG{+w}{ }!\PYG{+w}{ }grep\PYG{+w}{ }\PYGZhy{}q\PYG{+w}{ }\PYG{l+s+s1}{\PYGZsq{}\PYGZdq{}value\PYGZdq{}: \PYGZdq{}EPSG:2975\PYGZdq{}\PYGZsq{}}\PYG{+w}{ }/etc/georchestra/datafeeder/frontend\PYGZhy{}config.json\PYG{p}{;}\PYG{+w}{ }\PYG{k}{then}
\PYG{+w}{   }sed\PYG{+w}{ }\PYGZhy{}i\PYG{+w}{ }\PYG{l+s+s1}{\PYGZsq{}/\PYGZdq{}projections\PYGZdq{}: \PYGZbs{}[/a \PYGZbs{}}
\PYG{l+s+s1}{   \PYGZob{}\PYGZbs{}}
\PYG{l+s+s1}{      \PYGZdq{}label\PYGZdq{}: \PYGZdq{}RGR92 / UTM zone 40S\PYGZdq{},\PYGZbs{}}
\PYG{l+s+s1}{      \PYGZdq{}value\PYGZdq{}: \PYGZdq{}EPSG:2975\PYGZdq{}\PYGZbs{}}
\PYG{l+s+s1}{   \PYGZcb{},\PYGZsq{}}\PYG{+w}{ }/etc/georchestra/datafeeder/frontend\PYGZhy{}config.json
\PYG{+w}{   }\PYG{n+nb}{echo}\PYG{+w}{ }\PYG{l+s+s2}{\PYGZdq{}Nouvelle entrée ajoutée à la liste \PYGZsq{}projections\PYGZsq{}.\PYGZdq{}}
\PYG{k}{else}
\PYG{+w}{   }\PYG{n+nb}{echo}\PYG{+w}{ }\PYG{l+s+s2}{\PYGZdq{}La projection \PYGZsq{}EPSG:2975\PYGZsq{} existe déjà dans la liste \PYGZsq{}projections\PYGZsq{}.\PYGZdq{}}
\PYG{k}{fi}

\PYG{n+nb}{echo}\PYG{+w}{ }\PYG{l+s+s2}{\PYGZdq{}Mise à jour du fichier JSON terminée.\PYGZdq{}}

\PYG{c+c1}{\PYGZsh{} Remplacement des valeurs dans le fichier XML}
\PYG{n+nb}{echo}\PYG{+w}{ }\PYG{l+s+s2}{\PYGZdq{}Remplacement de \PYGZsq{}codeListValue=\PYGZbs{}\PYGZdq{}eng\PYGZbs{}\PYGZdq{}\PYGZsq{} par \PYGZsq{}codeListValue=\PYGZbs{}\PYGZdq{}fre\PYGZbs{}\PYGZdq{}\PYGZsq{} dans le fichier XML de datafeeder\PYGZdq{}}
sed\PYG{+w}{ }\PYGZhy{}i\PYG{+w}{ }\PYG{l+s+s1}{\PYGZsq{}s/codeListValue=\PYGZdq{}eng\PYGZdq{}/codeListValue=\PYGZdq{}fre\PYGZdq{}/g\PYGZsq{}}\PYG{+w}{ }/etc/georchestra/datafeeder/metadata\PYGZus{}template.xml
\PYG{n+nb}{echo}\PYG{+w}{ }\PYG{l+s+s2}{\PYGZdq{}Remplacement dans le fichier XML terminé.\PYGZdq{}}

\PYG{c+c1}{\PYGZsh{} Suppression du fichier header\PYGZus{}bg.web et copie du fichier header\PYGZus{}bg.webp}
\PYG{n+nb}{echo}\PYG{+w}{ }\PYG{l+s+s2}{\PYGZdq{}Suppression du fichier header\PYGZus{}bg.web et copie du fichier header\PYGZus{}bg.webp...\PYGZdq{}}
rm\PYG{+w}{ }\PYGZhy{}f\PYG{+w}{ }/etc/georchestra/datahub/assets/img/header\PYGZus{}bg.web
cp\PYG{+w}{ }header\PYGZus{}bg.webp\PYG{+w}{ }/etc/georchestra/datahub/assets/img/
\PYG{n+nb}{echo}\PYG{+w}{ }\PYG{l+s+s2}{\PYGZdq{}Fichier header\PYGZus{}bg.web remplacé par header\PYGZus{}bg.webp.\PYGZdq{}}

\PYG{c+c1}{\PYGZsh{} Remplacement dans le fichier TOML pour les langues}
\PYG{n+nb}{echo}\PYG{+w}{ }\PYG{l+s+s2}{\PYGZdq{}Remplacement de \PYGZsq{}\PYGZsh{} languages = [\PYGZsq{}en\PYGZsq{}, \PYGZsq{}fr\PYGZsq{}, \PYGZsq{}de\PYGZsq{}]\PYGZsq{} par \PYGZsq{}languages = [\PYGZsq{}en\PYGZsq{}, \PYGZsq{}fr\PYGZsq{}, \PYGZsq{}de\PYGZsq{}]\PYGZsq{} dans le fichier TOML...\PYGZdq{}}
sed\PYG{+w}{ }\PYGZhy{}i\PYG{+w}{ }\PYG{l+s+s2}{\PYGZdq{}s/\PYGZsh{} languages = \PYGZbs{}[\PYGZsq{}en\PYGZsq{}, \PYGZsq{}fr\PYGZsq{}, \PYGZsq{}de\PYGZsq{}\PYGZbs{}]/languages = \PYGZbs{}[\PYGZsq{}en\PYGZsq{}, \PYGZsq{}fr\PYGZsq{}, \PYGZsq{}de\PYGZsq{}\PYGZbs{}]/\PYGZdq{}}\PYG{+w}{ }/etc/georchestra/datahub/conf/default.toml
\PYG{n+nb}{echo}\PYG{+w}{ }\PYG{l+s+s2}{\PYGZdq{}Remplacement dans le fichier TOML terminé.\PYGZdq{}}

\PYG{c+c1}{\PYGZsh{} Remplacement de la couleur primaire dans le fichier TOML}
\PYG{n+nb}{echo}\PYG{+w}{ }\PYG{l+s+s2}{\PYGZdq{}Remplacement de \PYGZsq{}primary\PYGZus{}color = \PYGZbs{}\PYGZdq{}\PYGZsh{}85127e\PYGZbs{}\PYGZdq{}\PYGZsq{} par \PYGZsq{}primary\PYGZus{}color = \PYGZbs{}\PYGZdq{}\PYGZsh{}0a397f\PYGZbs{}\PYGZdq{}\PYGZsq{} dans le fichier TOML...\PYGZdq{}}
sed\PYG{+w}{ }\PYGZhy{}i\PYG{+w}{ }\PYG{l+s+s1}{\PYGZsq{}s/primary\PYGZus{}color = \PYGZdq{}\PYGZsh{}85127e\PYGZdq{}/primary\PYGZus{}color = \PYGZdq{}\PYGZsh{}0a397f\PYGZdq{}/\PYGZsq{}}\PYG{+w}{ }/etc/georchestra/datahub/conf/default.toml
\PYG{n+nb}{echo}\PYG{+w}{ }\PYG{l+s+s2}{\PYGZdq{}Remplacement de la couleur primaire dans le fichier TOML terminé.\PYGZdq{}}

\PYG{c+c1}{\PYGZsh{} Remplacement de la couleur secondaire dans le fichier TOML}
\PYG{n+nb}{echo}\PYG{+w}{ }\PYG{l+s+s2}{\PYGZdq{}Remplacement de \PYGZsq{}secondary\PYGZus{}color = \PYGZbs{}\PYGZdq{}\PYGZsh{}1b1f3b\PYGZbs{}\PYGZdq{}\PYGZsq{} par \PYGZsq{}secondary\PYGZus{}color = \PYGZbs{}\PYGZdq{}\PYGZsh{}225ea8\PYGZbs{}\PYGZdq{}\PYGZsq{} dans le fichier TOML...\PYGZdq{}}
sed\PYG{+w}{ }\PYGZhy{}i\PYG{+w}{ }\PYG{l+s+s1}{\PYGZsq{}s/secondary\PYGZus{}color = \PYGZdq{}\PYGZsh{}1b1f3b\PYGZdq{}/secondary\PYGZus{}color = \PYGZdq{}\PYGZsh{}225ea8\PYGZdq{}/\PYGZsq{}}\PYG{+w}{ }/etc/georchestra/datahub/conf/default.toml
\PYG{n+nb}{echo}\PYG{+w}{ }\PYG{l+s+s2}{\PYGZdq{}Remplacement de la couleur secondaire dans le fichier TOML terminé.\PYGZdq{}}

\PYG{c+c1}{\PYGZsh{} Suppression du commentaire et activation de la ligne dans le fichier TOML}
\PYG{n+nb}{echo}\PYG{+w}{ }\PYG{l+s+s2}{\PYGZdq{}Remplacement de \PYGZsq{}\PYGZsh{} enabled = true\PYGZsq{} par \PYGZsq{}enabled = true\PYGZsq{} pour activer le \PYGZpc{} de qualité de métadonnée\PYGZdq{}}
sed\PYG{+w}{ }\PYGZhy{}i\PYG{+w}{ }\PYG{l+s+s1}{\PYGZsq{}s/\PYGZsh{} enabled = true/enabled = true/\PYGZsq{}}\PYG{+w}{ }/etc/georchestra/datahub/conf/default.toml
\PYG{n+nb}{echo}\PYG{+w}{ }\PYG{l+s+s2}{\PYGZdq{}Activation de la ligne dans le fichier TOML terminée.\PYGZdq{}}

\PYG{c+c1}{\PYGZsh{} Suppression des sections \PYGZsq{}en\PYGZsq{} et \PYGZsq{}it\PYGZsq{} dans le fichier JSON}
\PYG{c+c1}{\PYGZsh{}echo \PYGZdq{}Suppression des sections \PYGZsq{}en\PYGZsq{} et \PYGZsq{}it\PYGZsq{} dans le fichier JSON...\PYGZdq{}}
\PYG{c+c1}{\PYGZsh{}sed \PYGZhy{}i \PYGZsq{}/\PYGZdq{}en\PYGZdq{}: \PYGZob{}/,/\PYGZcb{},/d\PYGZsq{} /etc/georchestra/mapstore/configs/localConfig.json}
\PYG{c+c1}{\PYGZsh{}sed \PYGZhy{}i \PYGZsq{}/\PYGZdq{}it\PYGZdq{}: \PYGZob{}/,/\PYGZcb{},/d\PYGZsq{} /etc/georchestra/mapstore/configs/localConfig.json}
\PYG{c+c1}{\PYGZsh{}echo \PYGZdq{}Suppression des sections terminée.\PYGZdq{}}

\PYG{c+c1}{\PYGZsh{} Vérifier si la nouvelle projection existe déjà dans la section \PYGZsq{}projectionDefs\PYGZsq{} et ajouter si elle n\PYGZsq{}existe pas}
\PYG{n+nb}{echo}\PYG{+w}{ }\PYG{l+s+s2}{\PYGZdq{}Vérification et ajout de la nouvelle projection à la section \PYGZsq{}projectionDefs\PYGZsq{}...\PYGZdq{}}
\PYG{k}{if}\PYG{+w}{ }!\PYG{+w}{ }grep\PYG{+w}{ }\PYGZhy{}q\PYG{+w}{ }\PYG{l+s+s1}{\PYGZsq{}\PYGZdq{}code\PYGZdq{}: \PYGZdq{}EPSG:2975\PYGZdq{}\PYGZsq{}}\PYG{+w}{ }/etc/georchestra/mapstore/configs/localConfig.json\PYG{p}{;}\PYG{+w}{ }\PYG{k}{then}
\PYG{+w}{   }sed\PYG{+w}{ }\PYGZhy{}i\PYG{+w}{ }\PYG{l+s+s1}{\PYGZsq{}/\PYGZdq{}projectionDefs\PYGZdq{}: \PYGZbs{}[/a \PYGZbs{}}
\PYG{l+s+s1}{      \PYGZdq{}code\PYGZdq{}: \PYGZdq{}EPSG:2975\PYGZdq{},\PYGZbs{}}
\PYG{l+s+s1}{      \PYGZdq{}def\PYGZdq{}: \PYGZdq{}+proj=lcc +lat\PYGZus{}1=48.5 +lat\PYGZus{}2=49.5 +lat\PYGZus{}0=48.0 +lon\PYGZus{}0=\PYGZhy{}123.0 +x\PYGZus{}0=1000000 +y\PYGZus{}0=0 +ellps=GRS80 +towgs84=0,0,0,0,0,0,0 +units=m +no\PYGZus{}defs\PYGZdq{},\PYGZbs{}}
\PYG{l+s+s1}{      \PYGZdq{}extent\PYGZdq{}: [\PYGZhy{}600000, 1500000, 1200000, 5000000],\PYGZbs{}}
\PYG{l+s+s1}{      \PYGZdq{}worldExtent\PYGZdq{}: [\PYGZhy{}130, 24, \PYGZhy{}66, 49]\PYGZbs{}}
\PYG{l+s+s1}{   \PYGZcb{},\PYGZob{}\PYGZsq{}}\PYG{+w}{ }/etc/georchestra/mapstore/configs/localConfig.json
\PYG{+w}{   }\PYG{n+nb}{echo}\PYG{+w}{ }\PYG{l+s+s2}{\PYGZdq{}Nouvelle projection ajoutée à la section \PYGZsq{}projectionDefs\PYGZsq{}.\PYGZdq{}}
\PYG{k}{else}
\PYG{+w}{   }\PYG{n+nb}{echo}\PYG{+w}{ }\PYG{l+s+s2}{\PYGZdq{}La projection \PYGZsq{}EPSG:2975\PYGZsq{} existe déjà dans la section \PYGZsq{}projectionDefs\PYGZsq{}.\PYGZdq{}}
\PYG{k}{fi}

\PYG{n+nb}{echo}\PYG{+w}{ }\PYG{l+s+s2}{\PYGZdq{}Mise à jour du fichier terminé.\PYGZdq{}}
\end{sphinxVerbatim}

\begin{sphinxVerbatim}[commandchars=\\\{\}]
\PYG{c+ch}{\PYGZsh{}!/bin/bash}

\PYG{c+c1}{\PYGZsh{} Copier le fichier stylesheet.css dans les répertoires de destination}
\PYG{n+nb}{echo}\PYG{+w}{ }\PYG{l+s+s2}{\PYGZdq{}Copie du fichier stylesheet.css dans les répertoires de destination...\PYGZdq{}}

\PYG{c+c1}{\PYGZsh{} Répertoires de destination}
\PYG{n+nv}{DESTINATIONS}\PYG{o}{=}\PYG{o}{(}
\PYG{l+s+s2}{\PYGZdq{}/var/www/georchestra/htdocs/datahub/\PYGZdq{}}
\PYG{l+s+s2}{\PYGZdq{}/srv/tomcat/georchestra/webapps/analytics/\PYGZdq{}}
\PYG{l+s+s2}{\PYGZdq{}/srv/tomcat/proxycas/webapps/cas/WEB\PYGZhy{}INF/classes/static/\PYGZdq{}}
\PYG{l+s+s2}{\PYGZdq{}/srv/tomcat/georchestra/webapps/console/account/\PYGZdq{}}
\PYG{o}{)}

\PYG{c+c1}{\PYGZsh{} Boucle pour copier le fichier dans chaque répertoire}
\PYG{k}{for}\PYG{+w}{ }DEST\PYG{+w}{ }\PYG{k}{in}\PYG{+w}{ }\PYG{l+s+s2}{\PYGZdq{}}\PYG{l+s+si}{\PYGZdl{}\PYGZob{}}\PYG{n+nv}{DESTINATIONS}\PYG{p}{[@]}\PYG{l+s+si}{\PYGZcb{}}\PYG{l+s+s2}{\PYGZdq{}}\PYG{p}{;}\PYG{+w}{ }\PYG{k}{do}
cp\PYG{+w}{ }./stylesheet.css\PYG{+w}{ }\PYG{l+s+s2}{\PYGZdq{}}\PYG{n+nv}{\PYGZdl{}DEST}\PYG{l+s+s2}{\PYGZdq{}}
\PYG{n+nb}{echo}\PYG{+w}{ }\PYG{l+s+s2}{\PYGZdq{}}\PYG{l+s+s2}{Fichier stylesheet.css copié avec succès dans }\PYG{n+nv}{\PYGZdl{}DEST}\PYG{l+s+s2}{.}\PYG{l+s+s2}{\PYGZdq{}}
\PYG{k}{done}

\PYG{c+c1}{\PYGZsh{} Remplacement des couleurs dans le fichier CSS}
\PYG{n+nb}{echo}\PYG{+w}{ }\PYG{l+s+s2}{\PYGZdq{}Remplacement des couleurs dans le fichier cas.css\PYGZdq{}}
\PYG{c+c1}{\PYGZsh{} Remplacer \PYGZsh{}540069 par \PYGZsh{}0a397f}
sed\PYG{+w}{ }\PYGZhy{}i\PYG{+w}{ }\PYG{l+s+s1}{\PYGZsq{}s/\PYGZsh{}540069/\PYGZsh{}0a397f/g\PYGZsq{}}\PYG{+w}{ }\PYG{l+s+s2}{\PYGZdq{}/srv/tomcat/proxycas/webapps/cas/WEB\PYGZhy{}INF/classes/static/themes/georchestra/css/cas.css\PYGZdq{}}
\PYG{c+c1}{\PYGZsh{} Remplacer \PYGZsh{}720e9e par \PYGZsh{}0a397f}
sed\PYG{+w}{ }\PYGZhy{}i\PYG{+w}{ }\PYG{l+s+s1}{\PYGZsq{}s/\PYGZsh{}720e9e/\PYGZsh{}0a397f/g\PYGZsq{}}\PYG{+w}{ }\PYG{l+s+s2}{\PYGZdq{}/srv/tomcat/proxycas/webapps/cas/WEB\PYGZhy{}INF/classes/static/themes/georchestra/css/cas.css\PYGZdq{}}
\PYG{c+c1}{\PYGZsh{} Remplacer \PYGZsh{}845490 par \PYGZsh{}225ea8}
sed\PYG{+w}{ }\PYGZhy{}i\PYG{+w}{ }\PYG{l+s+s1}{\PYGZsq{}s/\PYGZsh{}845490/\PYGZsh{}225ea8/g\PYGZsq{}}\PYG{+w}{ }\PYG{l+s+s2}{\PYGZdq{}/srv/tomcat/proxycas/webapps/cas/WEB\PYGZhy{}INF/classes/static/themes/georchestra/css/cas.css\PYGZdq{}}
\PYG{n+nb}{echo}\PYG{+w}{ }\PYG{l+s+s2}{\PYGZdq{}Remplacement des couleurs terminé.\PYGZdq{}}

\PYG{c+c1}{\PYGZsh{} Remplacement des valeurs de langue dans le fichier JSP}
\PYG{c+c1}{\PYGZsh{}echo \PYGZdq{}Remplacement des valeurs de langue dans le fichier JSP...\PYGZdq{}}

\PYG{c+c1}{\PYGZsh{} Remplacer lang = forcedLang par lang = \PYGZdq{}fr\PYGZdq{}}
\PYG{c+c1}{\PYGZsh{}sed \PYGZhy{}i \PYGZsq{}s/lang = forcedLang/lang = \PYGZdq{}fr\PYGZdq{}/g\PYGZsq{} \PYGZdq{}/srv/tomcat/georchestra/webapps/analytics/WEB\PYGZhy{}INF/jsp/index.jsp\PYGZdq{}}

\PYG{c+c1}{\PYGZsh{} Remplacer lang = detectedLanguage par lang = \PYGZdq{}fr\PYGZdq{}}
\PYG{c+c1}{\PYGZsh{}sed \PYGZhy{}i \PYGZsq{}s/lang = detectedLanguage/lang = \PYGZdq{}fr\PYGZdq{}/g\PYGZsq{} \PYGZdq{}/srv/tomcat/georchestra/webapps/analytics/WEB\PYGZhy{}INF/jsp/index.jsp\PYGZdq{}}

\PYG{c+c1}{\PYGZsh{}echo \PYGZdq{}Remplacement des valeurs de langue terminé.\PYGZdq{}}

\PYG{c+c1}{\PYGZsh{} Changement de couleurs dans le css de mapstore}
\PYG{n+nb}{echo}\PYG{+w}{ }\PYG{l+s+s2}{\PYGZdq{}Changement de couleurs dans le css de mapstore\PYGZdq{}}
sed\PYG{+w}{ }\PYGZhy{}i\PYG{+w}{ }\PYG{l+s+s1}{\PYGZsq{}s/\PYGZsh{}85127e/\PYGZsh{}0a397f/g\PYGZsq{}}\PYG{+w}{ }/srv/tomcat/georchestra/webapps/mapstore/dist/themes/default.css

\PYG{n+nb}{echo}\PYG{+w}{ }\PYG{l+s+s2}{\PYGZdq{}Changement de couleurs dans le css de mapstore terminé.\PYGZdq{}}


\PYG{c+c1}{\PYGZsh{} Changement du header de datahub}
\PYG{n+nb}{echo}\PYG{+w}{ }\PYG{l+s+s2}{\PYGZdq{}Changement du header de datahub\PYGZdq{}}
\PYG{c+c1}{\PYGZsh{} Chemin vers votre fichier HTML}
\PYG{n+nv}{file}\PYG{o}{=}\PYG{l+s+s2}{\PYGZdq{}/var/www/georchestra/htdocs/datahub/index.html\PYGZdq{}}

\PYG{c+c1}{\PYGZsh{} Attributs à vérifier}
\PYG{n+nv}{attr\PYGZus{}check}\PYG{o}{=}\PYG{l+s+s2}{\PYGZdq{}lang=\PYGZsq{}fr\PYGZsq{} stylesheet=\PYGZsq{}./stylesheet.css\PYGZsq{} logo\PYGZhy{}url=\PYGZsq{}./georchestra\PYGZhy{}logo.svg\PYGZsq{}\PYGZdq{}}

\PYG{c+c1}{\PYGZsh{} Vérifier si la balise \PYGZlt{}geor\PYGZhy{}header\PYGZgt{} avec les attributs existe déjà}
\PYG{k}{if}\PYG{+w}{ }grep\PYG{+w}{ }\PYGZhy{}q\PYG{+w}{ }\PYG{l+s+s2}{\PYGZdq{}}\PYG{l+s+s2}{\PYGZlt{}geor\PYGZhy{}header.*}\PYG{n+nv}{\PYGZdl{}attr\PYGZus{}check}\PYG{l+s+s2}{.*\PYGZgt{}}\PYG{l+s+s2}{\PYGZdq{}}\PYG{+w}{ }\PYG{l+s+s2}{\PYGZdq{}}\PYG{n+nv}{\PYGZdl{}file}\PYG{l+s+s2}{\PYGZdq{}}\PYG{p}{;}\PYG{+w}{ }\PYG{k}{then}
\PYG{n+nb}{echo}\PYG{+w}{ }\PYG{l+s+s2}{\PYGZdq{}Les attributs existent déjà dans la balise \PYGZlt{}geor\PYGZhy{}header\PYGZgt{}.\PYGZdq{}}
\PYG{k}{else}
\PYG{n+nb}{echo}\PYG{+w}{ }\PYG{l+s+s2}{\PYGZdq{}Les attributs n\PYGZsq{}existent pas. Ajout en cours...\PYGZdq{}}
\PYG{c+c1}{\PYGZsh{} Commande sed pour ajouter les attributs}
sed\PYG{+w}{ }\PYGZhy{}i\PYG{+w}{ }\PYG{l+s+s2}{\PYGZdq{}s/\PYGZlt{}geor\PYGZhy{}header active\PYGZhy{}app=\PYGZsq{}datahub\PYGZsq{} legacy\PYGZhy{}header=\PYGZsq{}false\PYGZsq{} legacy\PYGZhy{}url=\PYGZsq{}\PYGZbs{}/header\PYGZbs{}/\PYGZsq{} style=\PYGZsq{}height:90px\PYGZsq{}\PYGZgt{}/\PYGZlt{}geor\PYGZhy{}header active\PYGZhy{}app=\PYGZsq{}datahub\PYGZsq{} legacy\PYGZhy{}header=\PYGZsq{}false\PYGZsq{} legacy\PYGZhy{}url=\PYGZsq{}\PYGZbs{}/header\PYGZbs{}/\PYGZsq{} lang=\PYGZsq{}fr\PYGZsq{} stylesheet=\PYGZsq{}.\PYGZbs{}/stylesheet.css\PYGZsq{} logo\PYGZhy{}url=\PYGZsq{}.\PYGZbs{}/georchestra\PYGZhy{}logo.svg\PYGZsq{} style=\PYGZsq{}height:90px\PYGZsq{}\PYGZgt{}/g\PYGZdq{}}\PYG{+w}{ }\PYG{l+s+s2}{\PYGZdq{}}\PYG{n+nv}{\PYGZdl{}file}\PYG{l+s+s2}{\PYGZdq{}}
\PYG{n+nb}{echo}\PYG{+w}{ }\PYG{l+s+s2}{\PYGZdq{}Les attributs ont été ajoutés.\PYGZdq{}}
\PYG{k}{fi}

\PYG{c+c1}{\PYGZsh{}Ajout du logo pour le header de datahub}
\PYG{n+nb}{echo}\PYG{+w}{ }\PYG{l+s+s2}{\PYGZdq{}Ajout du logo pour le header de datahub\PYGZdq{}}
cp\PYG{+w}{ }./georchestra\PYGZhy{}logo.svg\PYG{+w}{ }/var/www/georchestra/htdocs/datahub/
\PYG{n+nb}{echo}\PYG{+w}{ }\PYG{l+s+s2}{\PYGZdq{}Ajout du logo pour le header de mapstore terminé\PYGZdq{}}

\PYG{c+c1}{\PYGZsh{}Changement du favicon}
\PYG{n+nb}{echo}\PYG{+w}{ }\PYG{l+s+s2}{\PYGZdq{}Remplacement du favicon\PYGZdq{}}
rm\PYG{+w}{ }/var/www/georchestra/htdocs/favicon.ico
cp\PYG{+w}{ }./favicon.ico\PYG{+w}{ }/var/www/georchestra/htdocs/favicon.ico
\PYG{n+nb}{echo}\PYG{+w}{ }\PYG{l+s+s2}{\PYGZdq{}Ramplacement du favicon\PYGZdq{}}

\PYG{c+c1}{\PYGZsh{}Changement des couleurs de mapstore}
\PYG{n+nb}{echo}\PYG{+w}{ }\PYG{l+s+s2}{\PYGZdq{}Changement des couleurs pour mapstore\PYGZdq{}}
sed\PYG{+w}{ }\PYGZhy{}i\PYG{+w}{ }\PYG{l+s+s1}{\PYGZsq{}s/\PYGZsh{}6f0f69/\PYGZsh{}0a397f/g\PYGZsq{}}\PYG{+w}{ }/srv/tomcat/georchestra/webapps/mapstore/dist/themes/default.css
sed\PYG{+w}{ }\PYGZhy{}i\PYG{+w}{ }\PYG{l+s+s1}{\PYGZsq{}s/\PYGZsh{}ed76e5/\PYGZsh{}0a397f/g\PYGZsq{}}\PYG{+w}{ }/srv/tomcat/georchestra/webapps/mapstore/dist/themes/default.css
sed\PYG{+w}{ }\PYGZhy{}i\PYG{+w}{ }\PYG{l+s+s1}{\PYGZsq{}s/\PYGZsh{}df1ed3/\PYGZsh{}0a397f/g\PYGZsq{}}\PYG{+w}{ }/srv/tomcat/georchestra/webapps/mapstore/dist/themes/default.css
sed\PYG{+w}{ }\PYGZhy{}i\PYG{+w}{ }\PYG{l+s+s1}{\PYGZsq{}s/\PYGZsh{}708/\PYGZsh{}0a397f/g\PYGZsq{}}\PYG{+w}{ }/srv/tomcat/georchestra/webapps/mapstore/dist/themes/default.css
sed\PYG{+w}{ }\PYGZhy{}i\PYG{+w}{ }\PYG{l+s+s1}{\PYGZsq{}s/\PYGZsh{}d97fff/\PYGZsh{}0a397f/g\PYGZsq{}}\PYG{+w}{ }/srv/tomcat/georchestra/webapps/mapstore/dist/themes/default.css
sed\PYG{+w}{ }\PYGZhy{}i\PYG{+w}{ }\PYG{l+s+s1}{\PYGZsq{}s/\PYGZsh{}6e296a/\PYGZsh{}0a397f/g\PYGZsq{}}\PYG{+w}{ }/srv/tomcat/georchestra/webapps/mapstore/dist/themes/default.css
sed\PYG{+w}{ }\PYGZhy{}i\PYG{+w}{ }\PYG{l+s+s1}{\PYGZsq{}s/\PYGZsh{}800080/\PYGZsh{}0a397f/g\PYGZsq{}}\PYG{+w}{ }/srv/tomcat/georchestra/webapps/mapstore/dist/themes/default.css
sed\PYG{+w}{ }\PYGZhy{}i\PYG{+w}{ }\PYG{l+s+s1}{\PYGZsq{}s/\PYGZsh{}b218a9/\PYGZsh{}0a397f/g\PYGZsq{}}\PYG{+w}{ }/srv/tomcat/georchestra/webapps/mapstore/dist/themes/default.css
sed\PYG{+w}{ }\PYGZhy{}i\PYG{+w}{ }\PYG{l+s+s1}{\PYGZsq{}s/\PYGZsh{}610/\PYGZsh{}0a397f/g\PYGZsq{}}\PYG{+w}{ }/srv/tomcat/georchestra/webapps/mapstore/dist/themes/default.css
sed\PYG{+w}{ }\PYGZhy{}i\PYG{+w}{ }\PYG{l+s+s1}{\PYGZsq{}s/\PYGZsh{}d5c/\PYGZsh{}0a397f/g\PYGZsq{}}\PYG{+w}{ }/srv/tomcat/georchestra/webapps/mapstore/dist/themes/default.css
sed\PYG{+w}{ }\PYGZhy{}i\PYG{+w}{ }\PYG{l+s+s1}{\PYGZsq{}s/\PYGZsh{}8e1387/\PYGZsh{}0a397f/g\PYGZsq{}}\PYG{+w}{ }/srv/tomcat/georchestra/webapps/mapstore/dist/themes/default.css
sed\PYG{+w}{ }\PYGZhy{}i\PYG{+w}{ }\PYG{l+s+s1}{\PYGZsq{}s/\PYGZsh{}7c1175/\PYGZsh{}0a397f/g\PYGZsq{}}\PYG{+w}{ }/srv/tomcat/georchestra/webapps/mapstore/dist/themes/default.css
sed\PYG{+w}{ }\PYGZhy{}i\PYG{+w}{ }\PYG{l+s+s1}{\PYGZsq{}s/\PYGZsh{}42093e/\PYGZsh{}0a397f/g\PYGZsq{}}\PYG{+w}{ }/srv/tomcat/georchestra/webapps/mapstore/dist/themes/default.css
sed\PYG{+w}{ }\PYGZhy{}i\PYG{+w}{ }\PYG{l+s+s1}{\PYGZsq{}s/\PYGZsh{}150314/\PYGZsh{}0a397f/g\PYGZsq{}}\PYG{+w}{ }/srv/tomcat/georchestra/webapps/mapstore/dist/themes/default.css
sed\PYG{+w}{ }\PYGZhy{}i\PYG{+w}{ }\PYG{l+s+s1}{\PYGZsq{}s/\PYGZsh{}390836/\PYGZsh{}0a397f/g\PYGZsq{}}\PYG{+w}{ }/srv/tomcat/georchestra/webapps/mapstore/dist/themes/default.css
sed\PYG{+w}{ }\PYGZhy{}i\PYG{+w}{ }\PYG{l+s+s1}{\PYGZsq{}s/\PYGZsh{}4f0b46/\PYGZsh{}0a397f/g\PYGZsq{}}\PYG{+w}{ }/srv/tomcat/georchestra/webapps/mapstore/dist/themes/default.css
sed\PYG{+w}{ }\PYGZhy{}i\PYG{+w}{ }\PYG{l+s+s1}{\PYGZsq{}s/\PYGZsh{}680c63/\PYGZsh{}0a397f/g\PYGZsq{}}\PYG{+w}{ }/srv/tomcat/georchestra/webapps/mapstore/dist/themes/default.css
sed\PYG{+w}{ }\PYGZhy{}i\PYG{+w}{ }\PYG{l+s+s1}{\PYGZsq{}s/\PYGZsh{}73106d/\PYGZsh{}0a397f/g\PYGZsq{}}\PYG{+w}{ }/srv/tomcat/georchestra/webapps/mapstore/dist/themes/default.css
sed\PYG{+w}{ }\PYGZhy{}i\PYG{+w}{ }\PYG{l+s+s1}{\PYGZsq{}s/\PYGZsh{}73106d/\PYGZsh{}0a397f/g\PYGZsq{}}\PYG{+w}{ }/srv/tomcat/georchestra/webapps/mapstore/dist/themes/default.css
\PYG{n+nb}{echo}\PYG{+w}{ }\PYG{l+s+s2}{\PYGZdq{}Changement des couleur terminés\PYGZdq{}}

\PYG{c+c1}{\PYGZsh{} Copie du favicon.png dans le repertoire de geonetwork}
\PYG{n+nb}{echo}\PYG{+w}{ }\PYG{l+s+s2}{\PYGZdq{}Copie du favicon.png dans le repertoire de geonetwork\PYGZdq{}}
cp\PYG{+w}{ }./favicon.png\PYG{+w}{ }/srv/data/geonetwork/data/resources/images/logos/
\PYG{n+nb}{echo}\PYG{+w}{ }\PYG{l+s+s2}{\PYGZdq{}Copie du favicon.png dans le repertoire de geonetwork terminé\PYGZdq{}}

\PYG{c+c1}{\PYGZsh{} Vérification et ajout des redirections}
\PYG{n+nb}{echo}\PYG{+w}{ }\PYG{l+s+s2}{\PYGZdq{}Vérification des redirections\PYGZdq{}}

\PYG{c+c1}{\PYGZsh{} Vérifiez si le pattern existe déjà dans le fichier}
\PYG{k}{if}\PYG{+w}{ }!\PYG{+w}{ }grep\PYG{+w}{ }\PYGZhy{}q\PYG{+w}{ }\PYG{l+s+s1}{\PYGZsq{}Redirect the stylesheet\PYGZsq{}}\PYG{+w}{ }/etc/nginx/sites\PYGZhy{}available/georchestra\PYG{p}{;}\PYG{+w}{ }\PYG{k}{then}
\PYG{+w}{   }\PYG{c+c1}{\PYGZsh{} Ajouter les redirections juste avant la dernière occurrence du pattern spécifique}
\PYG{+w}{   }sed\PYG{+w}{ }\PYGZhy{}i\PYG{+w}{ }\PYG{l+s+s1}{\PYGZsq{}/\PYGZsh{} redirect default to datahub/i \PYGZbs{}}
\PYG{l+s+s1}{      \PYGZsh{} Redirect the stylesheet.css url of geoserver to something known\PYGZbs{}}
\PYG{l+s+s1}{      location /geoserver/web/wicket/bookmarkable/stylesheet.css \PYGZob{}\PYGZbs{}}
\PYG{l+s+s1}{            alias /etc/georchestra/stylesheet.css;\PYGZbs{}}
\PYG{l+s+s1}{      \PYGZcb{}\PYGZbs{}}
\PYG{l+s+s1}{      \PYGZsh{} Same for another url\PYGZbs{}}
\PYG{l+s+s1}{      location /geoserver/web/stylesheet.css \PYGZob{}\PYGZbs{}}
\PYG{l+s+s1}{            alias /etc/georchestra/stylesheet.css;\PYGZbs{}}
\PYG{l+s+s1}{      \PYGZcb{}\PYGZbs{}}
\PYG{l+s+s1}{      \PYGZsh{} Redirect the stylesheet.css url of geonetwork to something known\PYGZbs{}}
\PYG{l+s+s1}{      location \PYGZti{} \PYGZca{}/geonetwork/.*/.*/stylesheet\PYGZbs{}\PYGZbs{}.css\PYGZdl{} \PYGZob{}\PYGZbs{}}
\PYG{l+s+s1}{            alias /etc/georchestra/stylesheet.css;\PYGZbs{}}
\PYG{l+s+s1}{      \PYGZcb{}\PYGZbs{}}
\PYG{l+s+s1}{      \PYGZsh{} Redirect the stylesheet.css of the console admin account url to something known\PYGZbs{}}
\PYG{l+s+s1}{      location /console/account/stylesheet.css \PYGZob{}\PYGZbs{}}
\PYG{l+s+s1}{            alias /etc/georchestra/stylesheet.css;\PYGZbs{}}
\PYG{l+s+s1}{      \PYGZcb{}\PYGZbs{}}
\PYG{l+s+s1}{      \PYGZsh{} Redirect the stylesheet.css of the console admin manager url to something known\PYGZbs{}}
\PYG{l+s+s1}{      location /console/manager/stylesheet.css \PYGZob{}\PYGZbs{}}
\PYG{l+s+s1}{            alias /etc/georchestra/stylesheet.css;\PYGZbs{}}
\PYG{l+s+s1}{      \PYGZcb{}\PYGZbs{}}
\PYG{l+s+s1}{      \PYGZsq{}}\PYG{+w}{ }/etc/nginx/sites\PYGZhy{}available/georchestra

\PYG{+w}{   }\PYG{n+nb}{echo}\PYG{+w}{ }\PYG{l+s+s2}{\PYGZdq{}Les redirections ont été ajoutées\PYGZdq{}}
\PYG{k}{else}
\PYG{+w}{   }\PYG{n+nb}{echo}\PYG{+w}{ }\PYG{l+s+s2}{\PYGZdq{}Les redirections ont déjà été ajoutées\PYGZdq{}}
\PYG{k}{fi}
\PYG{n+nb}{echo}\PYG{+w}{ }\PYG{l+s+s2}{\PYGZdq{}Mise à jour des redirections terminée.\PYGZdq{}}
\end{sphinxVerbatim}

\sphinxstepscope


\section{Configuration}
\label{\detokenize{doc_instal/configuration:configuration}}\label{\detokenize{doc_instal/configuration::doc}}
\begin{sphinxShadowBox}
\sphinxstyletopictitle{Table des matières}
\begin{itemize}
\item {} 
\sphinxAtStartPar
\phantomsection\label{\detokenize{doc_instal/configuration:id1}}{\hyperref[\detokenize{doc_instal/configuration:introduction}]{\sphinxcrossref{Introduction}}}

\item {} 
\sphinxAtStartPar
\phantomsection\label{\detokenize{doc_instal/configuration:id2}}{\hyperref[\detokenize{doc_instal/configuration:localisation-des-differents-repertoires}]{\sphinxcrossref{Localisation des différents répertoires}}}

\item {} 
\sphinxAtStartPar
\phantomsection\label{\detokenize{doc_instal/configuration:id3}}{\hyperref[\detokenize{doc_instal/configuration:fichiers-de-configuration-du-datadir}]{\sphinxcrossref{Fichiers de configuration du datadir}}}

\item {} 
\sphinxAtStartPar
\phantomsection\label{\detokenize{doc_instal/configuration:id4}}{\hyperref[\detokenize{doc_instal/configuration:versionnement-des-modules}]{\sphinxcrossref{Versionnement des modules}}}

\item {} 
\sphinxAtStartPar
\phantomsection\label{\detokenize{doc_instal/configuration:id5}}{\hyperref[\detokenize{doc_instal/configuration:base-de-donnee}]{\sphinxcrossref{Base de donnée}}}

\item {} 
\sphinxAtStartPar
\phantomsection\label{\detokenize{doc_instal/configuration:id6}}{\hyperref[\detokenize{doc_instal/configuration:relancer-l-infrastructure}]{\sphinxcrossref{Relancer l’infrastructure}}}

\end{itemize}
\end{sphinxShadowBox}


\subsection{Introduction}
\label{\detokenize{doc_instal/configuration:introduction}}
\sphinxAtStartPar
Le code étant très dense et compilé, il faudra comprendre la structure et les fichiers de configuration plutôt que le code en profondeur.


\subsection{Localisation des différents répertoires}
\label{\detokenize{doc_instal/configuration:localisation-des-differents-repertoires}}
\sphinxAtStartPar
Les logs des différents modules sont dans : \sphinxcode{\sphinxupquote{/srv/log/}}

\sphinxAtStartPar
Les binaires et le code source sont divisée en trois :
\sphinxhyphen{} \sphinxcode{\sphinxupquote{/srv/tomcat/georchestra/webapps}} pour les modules analytics, console, geonetwork, geowebcache, header, import, mapstore
\sphinxhyphen{} \sphinxcode{\sphinxupquote{/srv/tomcat/geoserver/webapps}} pour le module geoserver
\sphinxhyphen{} \sphinxcode{\sphinxupquote{/srv/tomcat/proxycas/webapps}} pour les modules cas et ROOT

\sphinxAtStartPar
Les données de geonetwork et geoserver sont dans le repertoire : \sphinxcode{\sphinxupquote{/srv/data/}}

\sphinxAtStartPar
Les pages web statiques sont dans : \sphinxcode{\sphinxupquote{/var/www/georchestra/htdocs/}}

\sphinxAtStartPar
Le module nginx est lui dans : \sphinxcode{\sphinxupquote{/etc/nginx/}}

\sphinxAtStartPar
Le dossier de configuration se trouve dans : \sphinxcode{\sphinxupquote{/etc/georchestra/}}


\subsection{Fichiers de configuration du datadir}
\label{\detokenize{doc_instal/configuration:fichiers-de-configuration-du-datadir}}
\sphinxAtStartPar
GeOrchestra possèdent un « datadir » qui est un repertoire de fichiers de configuration qui sert à modifier rapidement certaines configurations.
Il se situe dans : \sphinxcode{\sphinxupquote{/etc/georchestra}}
Il faut ensuite naviguer dans les différents répertoies pour modifier la configuration de rchaque module.

\sphinxAtStartPar
Les paramètres généraux peuvent être modfiée dans le fichier \sphinxcode{\sphinxupquote{default.properties}} où il est possible de modifier :
\sphinxhyphen{} le logo
\sphinxhyphen{} le style du header
\sphinxhyphen{} les paramètre de postgresql
\sphinxhyphen{} les paramètre du ldap
\sphinxhyphen{} les paramètres du rabittmq
\sphinxhyphen{} les paramètres SMTP

\sphinxAtStartPar
Ensuite il faut naviguer dans les différents sous\sphinxhyphen{}répertoire pour modifier spécifiquement les configs, voici le lien
de la documentation qui explique plus en détails : \sphinxurl{https://github.com/georchestra/datadir}


\subsection{Versionnement des modules}
\label{\detokenize{doc_instal/configuration:versionnement-des-modules}}
\sphinxAtStartPar
Le versionnement s’effectue dans le fichier \sphinxcode{\sphinxupquote{../playbooks/georchestra.yml}} qui est le fichier qui va donner les versions et les modules à installer
lors du lancement de l’installation.

\sphinxAtStartPar
Ce fichier sert à tout configurer, les versions, les chemins, les ports, les modules …

\sphinxAtStartPar
Il est très simple à lire et comprendre :

\begin{sphinxVerbatim}[commandchars=\\\{\}]
\PYGZhy{}\PYGZhy{}\PYGZhy{}
\PYGZhy{}\PYG{+w}{ }name:\PYG{+w}{ }georchestra\PYG{+w}{ }deployment
hosts:\PYG{+w}{ }localhost
\PYG{c+c1}{\PYGZsh{} note: above host must match the content of the \PYGZdq{}hosts\PYGZdq{} file}
become:\PYG{+w}{ }\PYG{n+nb}{true}
roles:
\PYG{+w}{   }\PYGZhy{}\PYG{+w}{ }\PYG{o}{\PYGZob{}}\PYG{+w}{ }role:\PYG{+w}{ }georchestra,\PYG{+w}{ }tags:\PYG{+w}{ }georchestra\PYG{+w}{ }\PYG{o}{\PYGZcb{}}
\PYG{+w}{   }\PYGZhy{}\PYG{+w}{ }\PYG{o}{\PYGZob{}}\PYG{+w}{ }role:\PYG{+w}{ }elastic.elasticsearch,\PYG{+w}{ }tags:\PYG{+w}{ }es\PYG{+w}{ }\PYG{o}{\PYGZcb{}}
\PYG{+w}{   }\PYGZhy{}\PYG{+w}{ }\PYG{o}{\PYGZob{}}\PYG{+w}{ }role:\PYG{+w}{ }geerlingguy.kibana,\PYG{+w}{ }tags:\PYG{+w}{ }kibana\PYG{+w}{ }\PYG{o}{\PYGZcb{}}

vars:

\PYG{+w}{   }georchestra\PYGZus{}versions:
\PYG{+w}{      }\PYG{c+c1}{\PYGZsh{} master version}
\PYG{+w}{      }\PYG{c+c1}{\PYGZsh{} datadir: 24.0 \PYGZsh{} or, see https://github.com/georchestra/datadir/branches}
\PYG{+w}{      }\PYG{c+c1}{\PYGZsh{} debian\PYGZus{}repository\PYGZus{}url: deb [signed\PYGZhy{}by=/etc/apt/keyrings/packages.georchestra.org.gpg] https://packages.georchestra.org/debian master main \PYGZsh{} or 24.0.x}
\PYG{+w}{      }\PYG{c+c1}{\PYGZsh{} georchestra\PYGZus{}repository: 24.0.x \PYGZsh{} see https://github.com/georchestra/georchestra/branches}
\PYG{+w}{      }\PYG{c+c1}{\PYGZsh{} geonetwork\PYGZus{}datadir: gn4.2.7 \PYGZsh{} see https://github.com/georchestra/geonetwork\PYGZus{}minimal\PYGZus{}datadir/branches}
\PYG{+w}{      }\PYG{c+c1}{\PYGZsh{} geoserver\PYGZus{}datadir: 2.25.0 \PYGZsh{} https://github.com/georchestra/geoserver\PYGZus{}minimal\PYGZus{}datadir/branches}
\PYG{+w}{      }\PYG{c+c1}{\PYGZsh{} 24.0.x}

\PYG{+w}{      }datadir:\PYG{+w}{ }\PYG{l+s+s2}{\PYGZdq{}24.0\PYGZdq{}}
\PYG{+w}{      }debian\PYGZus{}repository\PYGZus{}url:\PYG{+w}{ }\PYG{l+s+s2}{\PYGZdq{}deb [signed\PYGZhy{}by=/etc/apt/keyrings/packages.georchestra.org.gpg] https://packages.georchestra.org/debian 24.0.x main\PYGZdq{}}
\PYG{+w}{      }georchestra\PYGZus{}repository:\PYG{+w}{ }\PYG{l+s+s2}{\PYGZdq{}24.0.x\PYGZdq{}}
\PYG{+w}{      }geonetwork\PYGZus{}datadir:\PYG{+w}{ }\PYG{l+s+s2}{\PYGZdq{}gn4.2.7\PYGZdq{}}
\PYG{+w}{      }geoserver\PYGZus{}datadir:\PYG{+w}{ }\PYG{l+s+s2}{\PYGZdq{}24.0\PYGZdq{}}


\PYG{+w}{   }java\PYGZus{}version:\PYG{+w}{ }java\PYGZhy{}17\PYGZhy{}openjdk\PYGZhy{}amd64
\PYG{+w}{   }tomcat\PYGZus{}version:\PYG{+w}{ }\PYG{l+m}{9}
\PYG{+w}{   }kibana\PYGZus{}server\PYGZus{}host:\PYG{+w}{ }\PYG{l+m}{127}.0.0.1
\PYG{+w}{   }es\PYGZus{}version:\PYG{+w}{ }\PYG{l+m}{7}.17.22
\PYG{+w}{   }es\PYGZus{}data\PYGZus{}dirs:
\PYG{+w}{      }\PYGZhy{}\PYG{+w}{ }/srv/elasticsearch/data
\PYG{+w}{   }es\PYGZus{}log\PYGZus{}dir:\PYG{+w}{ }/srv/elasticsearch/logs
\PYG{+w}{   }es\PYGZus{}config:
\PYG{+w}{      }cluster.name:\PYG{+w}{ }\PYG{l+s+s2}{\PYGZdq{}\PYGZob{}\PYGZob{} georchestra.fqdn \PYGZcb{}\PYGZcb{}\PYGZdq{}}
\PYG{+w}{      }bootstrap.memory\PYGZus{}lock:\PYG{+w}{ }\PYG{n+nb}{true}
\PYG{+w}{   }es\PYGZus{}heap\PYGZus{}size:\PYG{+w}{ }1g
\PYG{+w}{   }cadastrapp:
\PYG{+w}{      }enabled:\PYG{+w}{ }\PYG{n+nb}{false}
\PYG{+w}{      }db:
\PYG{+w}{      }name:\PYG{+w}{ }georchestra
\PYG{+w}{      }user:\PYG{+w}{ }georchestra
\PYG{+w}{      }schema:\PYG{+w}{ }cadastrapp
\PYG{+w}{      }pass:\PYG{+w}{ }georchestra
\PYG{+w}{      }qgisdb:
\PYG{+w}{      }host:\PYG{+w}{ }localhost
\PYG{+w}{      }port:\PYG{+w}{ }\PYG{l+m}{5432}
\PYG{+w}{      }name:\PYG{+w}{ }georchestra
\PYG{+w}{      }user:\PYG{+w}{ }georchestra
\PYG{+w}{      }pass:\PYG{+w}{ }georchestra
\PYG{+w}{      }schema:\PYG{+w}{ }qadastre
\PYG{+w}{      }gitrepo:\PYG{+w}{ }https://github.com/georchestra/cadastrapp
\PYG{+w}{      }gitversion:\PYG{+w}{ }master
\PYG{+w}{      }debsrc:
\PYG{+w}{      }path:\PYG{+w}{ }/data/src/georchestra/cadastrapp/cadastrapp/target/
\PYG{+w}{      }pkg:\PYG{+w}{ }georchestra\PYGZhy{}cadastrapp\PYGZus{}99.master.202108020909\PYGZti{}80b14a6\PYGZhy{}1\PYGZus{}all.deb
\PYG{+w}{      }host:\PYG{+w}{ }build.fluela
\PYG{+w}{      }workdir:\PYG{+w}{ }/tmp/cadastrapp/tmp
\PYG{+w}{   }\PYG{c+c1}{\PYGZsh{} Set here your Github token, which should at least have the \PYGZsq{}actions\PYGZsq{} scope}
\PYG{+w}{   }github\PYGZus{}action\PYGZus{}token:\PYG{+w}{ }secret
\PYG{+w}{   }\PYG{c+c1}{\PYGZsh{} if deploying an ms2 artifact from gh}
\PYG{+w}{   }\PYG{c+c1}{\PYGZsh{} mapstore: \PYGZob{}}
\PYG{+w}{   }\PYG{c+c1}{\PYGZsh{}  enabled: True,}
\PYG{+w}{   }\PYG{c+c1}{\PYGZsh{}  repo: georchestra/mapstore2\PYGZhy{}georchestra,}
\PYG{+w}{   }\PYG{c+c1}{\PYGZsh{}  artifact\PYGZus{}id: 119135632,}
\PYG{+w}{   }\PYG{c+c1}{\PYGZsh{}  artifact\PYGZus{}sha256: b2803ecc76a3768fdc5e358f23b5c5ce10b02ddc \PYGZsh{}git commit hash}
\PYG{+w}{   }\PYG{c+c1}{\PYGZsh{} \PYGZcb{}}
\PYG{+w}{   }openldap:
\PYG{+w}{      }topdc:\PYG{+w}{ }georchestra
\PYG{+w}{      }basedn:\PYG{+w}{ }\PYG{n+nv}{dc}\PYG{o}{=}georchestra,dc\PYG{o}{=}org\PYG{+w}{ }\PYG{c+c1}{\PYGZsh{} has to be in the form dc=\PYGZob{}\PYGZob{} topdc \PYGZcb{}\PYGZcb{},dc=xx}
\PYG{+w}{      }rootdn:\PYG{+w}{ }\PYG{n+nv}{cn}\PYG{o}{=}admin,dc\PYG{o}{=}georchestra,dc\PYG{o}{=}org
\PYG{+w}{      }rootpw:\PYG{+w}{ }secret
\PYG{+w}{      }gitrepo:\PYG{+w}{ }https://raw.github.com/georchestra/georchestra
\PYG{+w}{      }ldifs:
\PYG{+w}{      }\PYGZhy{}\PYG{+w}{ }bootstrap
\PYG{+w}{      }\PYGZhy{}\PYG{+w}{ }docker\PYGZhy{}root/georchestraSchema
\PYG{+w}{      }\PYGZhy{}\PYG{+w}{ }docker\PYGZhy{}root/etc/ldap.dist/modules/groupofmembers
\PYG{+w}{      }\PYGZhy{}\PYG{+w}{ }docker\PYGZhy{}root/etc/ldap.dist/modules/openssh
\PYG{+w}{      }\PYGZhy{}\PYG{+w}{ }docker\PYGZhy{}root/memberof
\PYG{+w}{      }\PYGZhy{}\PYG{+w}{ }docker\PYGZhy{}root/lastbind
\PYG{+w}{      }\PYGZhy{}\PYG{+w}{ }root
\PYG{+w}{      }\PYGZhy{}\PYG{+w}{ }docker\PYGZhy{}root/georchestra
\PYG{+w}{      }gitversion:\PYG{+w}{ }\PYG{l+s+s2}{\PYGZdq{}\PYGZob{}\PYGZob{} georchestra\PYGZus{}versions.georchestra\PYGZus{}repository \PYGZcb{}\PYGZcb{}\PYGZdq{}}

\PYG{+w}{   }georchestra:
\PYG{+w}{      }fqdn:\PYG{+w}{ }georchestra.ole.re
\PYG{+w}{      }max\PYGZus{}body\PYGZus{}size:\PYG{+w}{ }100M
\PYG{+w}{      }ign\PYGZus{}api\PYGZus{}key:\PYG{+w}{ }luvs4p9c4yq5ewfwqcqgm83f\PYG{+w}{ }\PYG{c+c1}{\PYGZsh{} invalid key only used in sviewer}
\PYG{+w}{      }db:
\PYG{+w}{      }name:\PYG{+w}{ }georchestra
\PYG{+w}{      }user:\PYG{+w}{ }georchestra
\PYG{+w}{      }pass:\PYG{+w}{ }georchestra
\PYG{+w}{      }datadir:
\PYG{+w}{      }path:\PYG{+w}{ }/etc/georchestra
\PYG{+w}{      }gitrepo:\PYG{+w}{ }https://github.com/georchestra/datadir
\PYG{+w}{      }gitversion:\PYG{+w}{ }\PYG{l+s+s2}{\PYGZdq{}\PYGZob{}\PYGZob{} georchestra\PYGZus{}versions.datadir \PYGZcb{}\PYGZcb{}\PYGZdq{}}
\PYG{+w}{      }debian:
\PYG{+w}{      }repo:\PYG{+w}{ }\PYG{l+s+s2}{\PYGZdq{}\PYGZob{}\PYGZob{} georchestra\PYGZus{}versions.debian\PYGZus{}repository\PYGZus{}url \PYGZcb{}\PYGZcb{}\PYGZdq{}}
\PYG{+w}{      }key:\PYG{+w}{ }https://packages.georchestra.org/debian/landry\PYGZpc{}40georchestra.org.gpg.pubkey
\PYG{+w}{   }geonetwork:
\PYG{+w}{      }db:
\PYG{+w}{      }schema:\PYG{+w}{ }geonetwork
\PYG{+w}{      }datadir:
\PYG{+w}{      }path:\PYG{+w}{ }/srv/data/geonetwork/
\PYG{+w}{      }gitrepo:\PYG{+w}{ }https://github.com/georchestra/geonetwork\PYGZus{}minimal\PYGZus{}datadir
\PYG{+w}{      }gitversion:\PYG{+w}{ }\PYG{l+s+s2}{\PYGZdq{}\PYGZob{}\PYGZob{} georchestra\PYGZus{}versions.geonetwork\PYGZus{}datadir \PYGZcb{}\PYGZcb{}\PYGZdq{}}
\PYG{+w}{   }geoserver:
\PYG{+w}{      }privileged:
\PYG{+w}{      }user:\PYG{+w}{ }geoserver\PYGZus{}privileged\PYGZus{}user
\PYG{+w}{      }pass:\PYG{+w}{ }gerlsSnFd6SmM
\PYG{+w}{      }datadir:
\PYG{+w}{      }path:\PYG{+w}{ }/srv/data/geoserver/
\PYG{+w}{      }gitrepo:\PYG{+w}{ }https://github.com/georchestra/geoserver\PYGZus{}minimal\PYGZus{}datadir
\PYG{+w}{      }gitversion:\PYG{+w}{ }\PYG{l+s+s2}{\PYGZdq{}\PYGZob{}\PYGZob{} georchestra\PYGZus{}versions.geoserver\PYGZus{}datadir \PYGZcb{}\PYGZcb{}\PYGZdq{}}
\PYG{+w}{      }wms\PYGZus{}srslist:
\PYG{+w}{      }\PYGZhy{}\PYG{+w}{ }\PYG{l+m}{2154}
\PYG{+w}{      }\PYGZhy{}\PYG{+w}{ }\PYG{l+m}{3857}
\PYG{+w}{      }\PYGZhy{}\PYG{+w}{ }\PYG{l+m}{3942}
\PYG{+w}{      }\PYGZhy{}\PYG{+w}{ }\PYG{l+m}{3943}
\PYG{+w}{      }\PYGZhy{}\PYG{+w}{ }\PYG{l+m}{3944}
\PYG{+w}{      }\PYGZhy{}\PYG{+w}{ }\PYG{l+m}{3945}
\PYG{+w}{      }\PYGZhy{}\PYG{+w}{ }\PYG{l+m}{3946}
\PYG{+w}{      }\PYGZhy{}\PYG{+w}{ }\PYG{l+m}{3947}
\PYG{+w}{      }\PYGZhy{}\PYG{+w}{ }\PYG{l+m}{3948}
\PYG{+w}{      }\PYGZhy{}\PYG{+w}{ }\PYG{l+m}{3949}
\PYG{+w}{      }\PYGZhy{}\PYG{+w}{ }\PYG{l+m}{3950}
\PYG{+w}{      }\PYGZhy{}\PYG{+w}{ }\PYG{l+m}{4171}
\PYG{+w}{      }\PYGZhy{}\PYG{+w}{ }\PYG{l+m}{4258}
\PYG{+w}{      }\PYGZhy{}\PYG{+w}{ }\PYG{l+m}{4326}
\PYG{+w}{      }\PYGZhy{}\PYG{+w}{ }\PYG{l+m}{23030}
\PYG{+w}{      }\PYGZhy{}\PYG{+w}{ }\PYG{l+m}{23031}
\PYG{+w}{      }\PYGZhy{}\PYG{+w}{ }\PYG{l+m}{23032}
\PYG{+w}{      }\PYGZhy{}\PYG{+w}{ }\PYG{l+m}{32630}
\PYG{+w}{      }\PYGZhy{}\PYG{+w}{ }\PYG{l+m}{32631}
\PYG{+w}{      }\PYGZhy{}\PYG{+w}{ }\PYG{l+m}{32632}
\PYG{+w}{      }\PYGZhy{}\PYG{+w}{ }\PYG{l+m}{4171}
\PYG{+w}{      }\PYGZhy{}\PYG{+w}{ }\PYG{l+m}{4271}
\PYG{+w}{      }\PYGZhy{}\PYG{+w}{ }\PYG{l+m}{3758}
\PYG{+w}{   }geowebcache\PYGZus{}datadir:\PYG{+w}{ }/srv/data/geowebcache/
\PYG{+w}{   }tomcat\PYGZus{}keystore\PYGZus{}pass:\PYG{+w}{ }tomcatkstp
\PYG{+w}{   }tomcat\PYGZus{}basedir:\PYG{+w}{ }/srv/tomcat
\PYG{+w}{   }system\PYGZus{}locale:\PYG{+w}{ }en\PYGZus{}US.UTF\PYGZhy{}8
\PYG{+w}{   }logs\PYGZus{}basedir:\PYG{+w}{ }/srv/log
\PYG{+w}{   }force\PYGZus{}https:\PYG{+w}{ }\PYG{n+nb}{true}\PYG{+w}{ }\PYG{c+c1}{\PYGZsh{} set to false if running behind a reverse proxy that does SSL}
\PYG{+w}{   }\PYG{c+c1}{\PYGZsh{} if running behind a reverse proxy, uncomment/fill so that you get the real client ip in accesslogs}
\PYG{+w}{   }\PYG{c+c1}{\PYGZsh{}reverse\PYGZus{}proxy\PYGZus{}real\PYGZus{}ip: 10.0.0.1}
\PYG{+w}{   }\PYG{c+c1}{\PYGZsh{}reverse\PYGZus{}proxy\PYGZus{}real\PYGZus{}ip\PYGZus{}header: X\PYGZhy{}Forwarded\PYGZhy{}For}
\PYG{+w}{   }console\PYGZus{}adminemail:\PYG{+w}{ }admin@example.org
\PYG{+w}{   }console\PYGZus{}captcha:
\PYG{+w}{      }privateKey:\PYG{+w}{ }\PYG{l+s+s2}{\PYGZdq{}\PYGZdq{}}
\PYG{+w}{      }publicKey:\PYG{+w}{ }\PYG{l+s+s2}{\PYGZdq{}\PYGZdq{}}
\PYG{+w}{   }tomcat\PYGZus{}instances:
\PYG{+w}{      }proxycas:
\PYG{+w}{      }port:\PYG{+w}{ }\PYG{l+m}{8180}
\PYG{+w}{      }control\PYGZus{}port:\PYG{+w}{ }\PYG{l+m}{8105}
\PYG{+w}{      }xms:\PYG{+w}{ }256m
\PYG{+w}{      }xmx:\PYG{+w}{ }512m
\PYG{+w}{      }georchestra:
\PYG{+w}{      }port:\PYG{+w}{ }\PYG{l+m}{8280}
\PYG{+w}{      }control\PYGZus{}port:\PYG{+w}{ }\PYG{l+m}{8205}
\PYG{+w}{      }xms:\PYG{+w}{ }1G
\PYG{+w}{      }xmx:\PYG{+w}{ }2G
\PYG{+w}{      }geoserver:
\PYG{+w}{      }port:\PYG{+w}{ }\PYG{l+m}{8380}
\PYG{+w}{      }control\PYGZus{}port:\PYG{+w}{ }\PYG{l+m}{8305}
\PYG{+w}{      }xms:\PYG{+w}{ }1G
\PYG{+w}{      }xmx:\PYG{+w}{ }1G
\PYG{+w}{   }georchestra\PYGZus{}wars:
\PYG{+w}{      }analytics:
\PYG{+w}{      }pkg:\PYG{+w}{ }georchestra\PYGZhy{}analytics
\PYG{+w}{      }tomcat:\PYG{+w}{ }georchestra
\PYG{+w}{      }enabled:\PYG{+w}{ }\PYG{n+nb}{true}
\PYG{+w}{      }cas:
\PYG{+w}{      }pkg:\PYG{+w}{ }georchestra\PYGZhy{}cas
\PYG{+w}{      }tomcat:\PYG{+w}{ }proxycas
\PYG{+w}{      }enabled:\PYG{+w}{ }\PYG{n+nb}{true}
\PYG{+w}{      }geonetwork:
\PYG{+w}{      }pkg:\PYG{+w}{ }georchestra\PYGZhy{}geonetwork
\PYG{+w}{      }tomcat:\PYG{+w}{ }georchestra
\PYG{+w}{      }enabled:\PYG{+w}{ }\PYG{n+nb}{true}
\PYG{+w}{      }\PYG{c+c1}{\PYGZsh{} mapstore: \PYGZsh{} using a github action artifact}
\PYG{+w}{      }\PYG{c+c1}{\PYGZsh{}   url: https://api.github.com/repos/\PYGZob{}\PYGZob{} mapstore.repo \PYGZcb{}\PYGZcb{}/actions/artifacts/\PYGZob{}\PYGZob{} mapstore.artifact\PYGZus{}id \PYGZcb{}\PYGZcb{}/zip}
\PYG{+w}{      }\PYG{c+c1}{\PYGZsh{}   tomcat: georchestra}
\PYG{+w}{      }\PYG{c+c1}{\PYGZsh{}   artifact\PYGZus{}sha256: \PYGZdq{}\PYGZob{}\PYGZob{} mapstore.artifact\PYGZus{}sha256 \PYGZcb{}\PYGZcb{}\PYGZdq{}}
\PYG{+w}{      }\PYG{c+c1}{\PYGZsh{}   enabled: \PYGZdq{}\PYGZob{}\PYGZob{} mapstore.enabled \PYGZcb{}\PYGZcb{}\PYGZdq{}}
\PYG{+w}{      }mapstore:\PYG{+w}{ }\PYG{c+c1}{\PYGZsh{} using the package from packages.georchestra.org}
\PYG{+w}{      }pkg:\PYG{+w}{ }georchestra\PYGZhy{}mapstore
\PYG{+w}{      }tomcat:\PYG{+w}{ }georchestra
\PYG{+w}{      }enabled:\PYG{+w}{ }\PYG{n+nb}{true}
\PYG{+w}{      }geoserver:
\PYG{+w}{      }pkg:\PYG{+w}{ }georchestra\PYGZhy{}geoserver
\PYG{+w}{      }tomcat:\PYG{+w}{ }geoserver
\PYG{+w}{      }enabled:\PYG{+w}{ }\PYG{n+nb}{true}
\PYG{+w}{      }geowebcache:
\PYG{+w}{      }pkg:\PYG{+w}{ }georchestra\PYGZhy{}geowebcache
\PYG{+w}{      }tomcat:\PYG{+w}{ }georchestra
\PYG{+w}{      }enabled:\PYG{+w}{ }\PYG{n+nb}{true}
\PYG{+w}{      }import:
\PYG{+w}{      }pkg:\PYG{+w}{ }georchestra\PYGZhy{}datafeeder\PYGZhy{}ui
\PYG{+w}{      }tomcat:\PYG{+w}{ }georchestra
\PYG{+w}{      }enabled:\PYG{+w}{ }\PYG{n+nb}{true}
\PYG{+w}{      }header:
\PYG{+w}{      }pkg:\PYG{+w}{ }georchestra\PYGZhy{}header
\PYG{+w}{      }tomcat:\PYG{+w}{ }georchestra
\PYG{+w}{      }enabled:\PYG{+w}{ }\PYG{n+nb}{true}
\PYG{+w}{      }console:
\PYG{+w}{      }pkg:\PYG{+w}{ }georchestra\PYGZhy{}console
\PYG{+w}{      }tomcat:\PYG{+w}{ }georchestra
\PYG{+w}{      }enabled:\PYG{+w}{ }\PYG{n+nb}{true}
\PYG{+w}{      }cadastrapp:
\PYG{+w}{      }pkg:\PYG{+w}{ }georchestra\PYGZhy{}cadastrapp
\PYG{+w}{      }tomcat:\PYG{+w}{ }georchestra
\PYG{+w}{      }enabled:\PYG{+w}{ }\PYG{n+nb}{false}
\PYG{+w}{      }ROOT:
\PYG{+w}{      }pkg:\PYG{+w}{ }georchestra\PYGZhy{}security\PYGZhy{}proxy
\PYG{+w}{      }tomcat:\PYG{+w}{ }proxycas
\PYG{+w}{      }enabled:\PYG{+w}{ }\PYG{n+nb}{true}
\PYG{+w}{   }datafeeder:
\PYG{+w}{      }enabled:\PYG{+w}{ }\PYG{n+nb}{true}
\PYG{+w}{      }port:\PYG{+w}{ }\PYG{l+m}{8480}
\PYG{+w}{   }\PYG{c+c1}{\PYGZsh{} not yet, doesnt work standalone ?}
\PYG{+w}{   }\PYG{c+c1}{\PYGZsh{}    cas:}
\PYG{+w}{   }\PYG{c+c1}{\PYGZsh{}      pkg: georchestra\PYGZhy{}cas}
\PYG{+w}{   }\PYG{c+c1}{\PYGZsh{}      enabled: true}
\PYG{+w}{   }\PYG{c+c1}{\PYGZsh{}      port: 8980}
\PYG{+w}{   }gn\PYGZus{}cloud\PYGZus{}searching:
\PYG{+w}{      }enabled:\PYG{+w}{ }\PYG{n+nb}{true}
\PYG{+w}{      }port:\PYG{+w}{ }\PYG{l+m}{8580}
\PYG{+w}{      }url:\PYG{+w}{ }https://packages.georchestra.org/bot/wars/geonetwork\PYGZhy{}microservices/searching.jar
\PYG{+w}{   }gn\PYGZus{}ogc\PYGZus{}api\PYGZus{}records:
\PYG{+w}{      }enabled:\PYG{+w}{ }\PYG{n+nb}{true}
\PYG{+w}{      }port:\PYG{+w}{ }\PYG{l+m}{8880}
\PYG{+w}{      }url:\PYG{+w}{ }https://packages.georchestra.org/bot/wars/geonetwork\PYGZhy{}microservices/gn\PYGZhy{}ogc\PYGZhy{}api\PYGZhy{}records.jar
\PYG{+w}{   }datahub:
\PYG{+w}{      }enabled:\PYG{+w}{ }\PYG{n+nb}{true}
\PYG{+w}{      }url:\PYG{+w}{ }https://packages.georchestra.org/bot/datahub/datahub.zip
\PYG{+w}{      }default\PYGZus{}api\PYGZus{}url:\PYG{+w}{ }/geonetwork/srv/api\PYG{+w}{ }\PYG{c+c1}{\PYGZsh{} could be set to any other GeoNetwork catalogue, even remote if CORS allows it}
\PYG{+w}{   }mviewer:
\PYG{+w}{      }enabled:\PYG{+w}{ }\PYG{n+nb}{false}
\PYG{+w}{      }port:\PYG{+w}{ }\PYG{l+m}{8680}
\PYG{+w}{      }gitrepo:\PYG{+w}{ }https://github.com/mviewer/mviewer
\PYG{+w}{      }gitversion:\PYG{+w}{ }master
\PYG{+w}{   }mviewerstudio:
\PYG{+w}{      }enabled:\PYG{+w}{ }\PYG{n+nb}{false}
\PYG{+w}{      }port:\PYG{+w}{ }\PYG{l+m}{8780}
\PYG{+w}{      }gitrepo:\PYG{+w}{ }https://github.com/mviewer/mviewerstudio
\PYG{+w}{      }gitversion:\PYG{+w}{ }master
\PYG{+w}{   }gateway:
\PYG{+w}{      }enabled:\PYG{+w}{ }\PYG{n+nb}{false}
\PYG{+w}{      }port:\PYG{+w}{ }\PYG{l+m}{8980}
tasks:
\PYG{+w}{   }\PYGZhy{}\PYG{+w}{ }name:\PYG{+w}{ }reconfigure\PYG{+w}{ }Kibana\PYG{+w}{ }after\PYG{+w}{ }geerlingguy.kibana
\PYG{+w}{      }copy:
\PYG{+w}{      }src:\PYG{+w}{ }resources/kibana.yml
\PYG{+w}{      }dest:\PYG{+w}{ }/etc/kibana/kibana.yml
\PYG{+w}{      }owner:\PYG{+w}{ }root
\PYG{+w}{      }group:\PYG{+w}{ }root
\PYG{+w}{      }mode:\PYG{+w}{ }\PYG{l+s+s2}{\PYGZdq{}0644\PYGZdq{}}
\PYG{+w}{      }notify:\PYG{+w}{ }restart\PYG{+w}{ }kibana

handlers:
\PYG{+w}{   }\PYGZhy{}\PYG{+w}{ }name:\PYG{+w}{ }restart\PYG{+w}{ }kibana
\PYG{+w}{      }service:\PYG{+w}{ }\PYG{n+nv}{name}\PYG{o}{=}kibana\PYG{+w}{ }\PYG{n+nv}{state}\PYG{o}{=}restarted
\end{sphinxVerbatim}


\subsection{Base de donnée}
\label{\detokenize{doc_instal/configuration:base-de-donnee}}
\sphinxAtStartPar
La base de donnée est accessible avec psql :

\begin{sphinxVerbatim}[commandchars=\\\{\}]
psql\PYG{+w}{ }\PYGZhy{}U\PYG{+w}{ }georchestra\PYG{+w}{ }\PYGZhy{}h\PYG{+w}{ }localhost
\end{sphinxVerbatim}

\sphinxAtStartPar
Elle stocke les données dans différents schémas. Il n’est pas nécéssaire de l’utiliser.


\subsection{Relancer l’infrastructure}
\label{\detokenize{doc_instal/configuration:relancer-l-infrastructure}}
\sphinxAtStartPar
Pour relancer l’infrastructure, il faut relancer les 3 tomcats et potentiellement nginx :
\begin{itemize}
\item {} 
\sphinxAtStartPar
sudo systemctl restart \sphinxhref{mailto:tomcat@georchestra.service}{tomcat@georchestra.service}

\item {} 
\sphinxAtStartPar
sudo systemctl restart \sphinxhref{mailto:tomcat@geoserver.service}{tomcat@geoserver.service}

\item {} 
\sphinxAtStartPar
sudo systemctl restart \sphinxhref{mailto:tomcat@proxycas.service}{tomcat@proxycas.service}

\item {} 
\sphinxAtStartPar
sudo systemctl restart nginx

\end{itemize}

\sphinxstepscope


\section{Mise à jour}
\label{\detokenize{doc_instal/maj:mise-a-jour}}\label{\detokenize{doc_instal/maj::doc}}
\begin{sphinxShadowBox}
\sphinxstyletopictitle{Table des matières}
\begin{itemize}
\item {} 
\sphinxAtStartPar
\phantomsection\label{\detokenize{doc_instal/maj:id1}}{\hyperref[\detokenize{doc_instal/maj:introduction}]{\sphinxcrossref{Introduction}}}

\item {} 
\sphinxAtStartPar
\phantomsection\label{\detokenize{doc_instal/maj:id2}}{\hyperref[\detokenize{doc_instal/maj:paquets-debians}]{\sphinxcrossref{Paquets debians}}}

\end{itemize}
\end{sphinxShadowBox}


\subsection{Introduction}
\label{\detokenize{doc_instal/maj:introduction}}
\sphinxAtStartPar
La version actuelle de geOrchestra est la version 24, les versions sont supporté pendant 1 an avec des patchs mineurs qui ne demande pas de
configuration supplémentaire et peuvent être installées avec les paquets debians directement.

\sphinxAtStartPar
Pour ce qui est de l’installation de versions majeurs, elle se font en modifiant le fichier \sphinxcode{\sphinxupquote{georchestra.yml}},
il faudra relancer toute l’installation et potentiellement faire des ajustements.


\subsection{Paquets debians}
\label{\detokenize{doc_instal/maj:paquets-debians}}
\sphinxAtStartPar
Voici la liste des paquets debians installé par georchestra :

\noindent{\hspace*{\fill}\sphinxincludegraphics[width=700\sphinxpxdimen]{{debian_paquet}.png}\hspace*{\fill}}



\renewcommand{\indexname}{Index}
\printindex
\end{document}