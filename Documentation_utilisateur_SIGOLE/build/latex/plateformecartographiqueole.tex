%% Generated by Sphinx.
\def\sphinxdocclass{report}
\documentclass[letterpaper,10pt,french]{sphinxmanual}
\ifdefined\pdfpxdimen
   \let\sphinxpxdimen\pdfpxdimen\else\newdimen\sphinxpxdimen
\fi \sphinxpxdimen=.75bp\relax
\ifdefined\pdfimageresolution
    \pdfimageresolution= \numexpr \dimexpr1in\relax/\sphinxpxdimen\relax
\fi
%% let collapsible pdf bookmarks panel have high depth per default
\PassOptionsToPackage{bookmarksdepth=5}{hyperref}

\PassOptionsToPackage{booktabs}{sphinx}
\PassOptionsToPackage{colorrows}{sphinx}

\PassOptionsToPackage{warn}{textcomp}
\usepackage[utf8]{inputenc}
\ifdefined\DeclareUnicodeCharacter
% support both utf8 and utf8x syntaxes
  \ifdefined\DeclareUnicodeCharacterAsOptional
    \def\sphinxDUC#1{\DeclareUnicodeCharacter{"#1}}
  \else
    \let\sphinxDUC\DeclareUnicodeCharacter
  \fi
  \sphinxDUC{00A0}{\nobreakspace}
  \sphinxDUC{2500}{\sphinxunichar{2500}}
  \sphinxDUC{2502}{\sphinxunichar{2502}}
  \sphinxDUC{2514}{\sphinxunichar{2514}}
  \sphinxDUC{251C}{\sphinxunichar{251C}}
  \sphinxDUC{2572}{\textbackslash}
\fi
\usepackage{cmap}
\usepackage[T1]{fontenc}
\usepackage{amsmath,amssymb,amstext}
\usepackage{babel}



\usepackage{tgtermes}
\usepackage{tgheros}
\renewcommand{\ttdefault}{txtt}



\usepackage[Sonny]{fncychap}
\ChNameVar{\Large\normalfont\sffamily}
\ChTitleVar{\Large\normalfont\sffamily}
\usepackage{sphinx}

\fvset{fontsize=auto}
\usepackage{geometry}


% Include hyperref last.
\usepackage{hyperref}
% Fix anchor placement for figures with captions.
\usepackage{hypcap}% it must be loaded after hyperref.
% Set up styles of URL: it should be placed after hyperref.
\urlstyle{same}

\addto\captionsfrench{\renewcommand{\contentsname}{Documentation:}}

\usepackage{sphinxmessages}
\setcounter{tocdepth}{1}



\title{Plateforme cartographique OLE}
\date{nov. 06, 2024}
\release{}
\author{Vittorio Toffolutti}
\newcommand{\sphinxlogo}{\vbox{}}
\renewcommand{\releasename}{}
\makeindex
\begin{document}

\ifdefined\shorthandoff
  \ifnum\catcode`\=\string=\active\shorthandoff{=}\fi
  \ifnum\catcode`\"=\active\shorthandoff{"}\fi
\fi

\pagestyle{empty}
\sphinxmaketitle
\pagestyle{plain}
\sphinxtableofcontents
\pagestyle{normal}
\phantomsection\label{\detokenize{index::doc}}


\sphinxAtStartPar
\sphinxhref{plateformecartographiqueole.pdf}{Documentation au format PDF}.

\sphinxstepscope


\chapter{Documentation utilisateur}
\label{\detokenize{doc_user:documentation-utilisateur}}\label{\detokenize{doc_user::doc}}
\sphinxAtStartPar
\sphinxhref{plateformecartographiqueole.pdf}{Documentation au format PDF}.

\sphinxstepscope


\section{Le catalogue}
\label{\detokenize{doc_user/catalogue:le-catalogue}}\label{\detokenize{doc_user/catalogue::doc}}
\begin{sphinxShadowBox}
\sphinxstyletopictitle{Table des matières}
\begin{itemize}
\item {} 
\sphinxAtStartPar
\phantomsection\label{\detokenize{doc_user/catalogue:id2}}{\hyperref[\detokenize{doc_user/catalogue:introduction}]{\sphinxcrossref{Introduction}}}

\item {} 
\sphinxAtStartPar
\phantomsection\label{\detokenize{doc_user/catalogue:id3}}{\hyperref[\detokenize{doc_user/catalogue:la-recherche-de-donnee-dans-le-catalogue}]{\sphinxcrossref{La recherche de donnée dans le catalogue}}}

\item {} 
\sphinxAtStartPar
\phantomsection\label{\detokenize{doc_user/catalogue:id4}}{\hyperref[\detokenize{doc_user/catalogue:les-fonctionnalites-des-fiches-de-donnees}]{\sphinxcrossref{Les fonctionnalités des fiches de données}}}

\end{itemize}
\end{sphinxShadowBox}


\subsection{Introduction}
\label{\detokenize{doc_user/catalogue:introduction}}
\sphinxAtStartPar
Les données de ce catalogue proviennent de différents catalogues nationaux mais peuvent aussi être propre à l’Office de l’eau Réunion.
Chaque donnée est reliée à une organisation et possède une description détaillée.


\subsection{La recherche de donnée dans le catalogue}
\label{\detokenize{doc_user/catalogue:la-recherche-de-donnee-dans-le-catalogue}}
\sphinxAtStartPar
Il y a 3 onglets dans ce catalogue :
\begin{itemize}
\item {} 
\sphinxAtStartPar
\sphinxstylestrong{ACCUEIL} pour afficher les dernières données postées

\end{itemize}

\noindent{\hspace*{\fill}\sphinxincludegraphics[width=600\sphinxpxdimen]{{catalogue_accueil}.png}\hspace*{\fill}}

\sphinxAtStartPar
 
\begin{itemize}
\item {} 
\sphinxAtStartPar
\sphinxstylestrong{DONNEES} pour afficher toutes les données

\end{itemize}

\noindent{\hspace*{\fill}\sphinxincludegraphics[width=600\sphinxpxdimen]{{catalogue_donnees}.png}\hspace*{\fill}}

\sphinxAtStartPar
 
\begin{itemize}
\item {} 
\sphinxAtStartPar
\sphinxstylestrong{ORGANISATION} pour afficher toutes les organisations qui possèdent des données

\end{itemize}

\noindent{\hspace*{\fill}\sphinxincludegraphics[width=600\sphinxpxdimen]{{catalogue_orga}.png}\hspace*{\fill}}

\sphinxAtStartPar
 

\sphinxAtStartPar
Vous pouvez filtrer vos recherches en fonction :
\begin{itemize}
\item {} 
\sphinxAtStartPar
de la date de publication

\item {} 
\sphinxAtStartPar
du type de données

\item {} 
\sphinxAtStartPar
du format

\item {} 
\sphinxAtStartPar
de l’organisation qui l’a publiée

\item {} 
\sphinxAtStartPar
des mots clés associés

\item {} 
\sphinxAtStartPar
ou encore du type de licence si elle est renseignée

\end{itemize}

\noindent{\hspace*{\fill}\sphinxincludegraphics[width=600\sphinxpxdimen]{{filter_options}.png}\hspace*{\fill}}

\sphinxAtStartPar
 


\subsection{Les fonctionnalités des fiches de données}
\label{\detokenize{doc_user/catalogue:les-fonctionnalites-des-fiches-de-donnees}}
\begin{sphinxShadowBox}
\sphinxstyletopictitle{Table des matières}
\begin{itemize}
\item {} 
\sphinxAtStartPar
\phantomsection\label{\detokenize{doc_user/catalogue:id5}}{\hyperref[\detokenize{doc_user/catalogue:description-de-la-donnee}]{\sphinxcrossref{Description de la donnée}}}

\item {} 
\sphinxAtStartPar
\phantomsection\label{\detokenize{doc_user/catalogue:id6}}{\hyperref[\detokenize{doc_user/catalogue:previsualisation-de-la-donnee}]{\sphinxcrossref{Prévisualisation de la donnée}}}

\item {} 
\sphinxAtStartPar
\phantomsection\label{\detokenize{doc_user/catalogue:id7}}{\hyperref[\detokenize{doc_user/catalogue:telechargement-de-la-donnee}]{\sphinxcrossref{Téléchargement de la donnée}}}

\end{itemize}
\end{sphinxShadowBox}

\sphinxAtStartPar
Lorsque vous cliquez sur une donnée, la page de description de cette donnée s’affiche.


\subsubsection{Description de la donnée}
\label{\detokenize{doc_user/catalogue:description-de-la-donnee}}
\sphinxAtStartPar
Le haut de la page est dédiée à la description de cette donnée.
Il y’a :
\begin{itemize}
\item {} 
\sphinxAtStartPar
un titre

\item {} 
\sphinxAtStartPar
une description

\item {} 
\sphinxAtStartPar
la dernière date de la mise à jour,

\item {} 
\sphinxAtStartPar
son point de contact

\item {} 
\sphinxAtStartPar
le catalogue dont elle provient

\item {} 
\sphinxAtStartPar
les mots clés associés

\item {} 
\sphinxAtStartPar
un pourcentage à titre indicatif de la qualité de cette donnée

\item {} 
\sphinxAtStartPar
et d’autre informations plus technique

\end{itemize}

\noindent{\hspace*{\fill}\sphinxincludegraphics[width=600\sphinxpxdimen]{{fiche_info}.png}\hspace*{\fill}}

\sphinxAtStartPar
 


\subsubsection{Prévisualisation de la donnée}
\label{\detokenize{doc_user/catalogue:previsualisation-de-la-donnee}}
\sphinxAtStartPar
Une interface de prévisualisation est aussi accessible si vous descendez la page.
Cette interface permet de :
\begin{itemize}
\item {} 
\sphinxAtStartPar
prévisualiser la donnée

\end{itemize}

\noindent{\hspace*{\fill}\sphinxincludegraphics[width=600\sphinxpxdimen]{{fiche_previsu}.png}\hspace*{\fill}}

\sphinxAtStartPar
 
\begin{itemize}
\item {} 
\sphinxAtStartPar
visualiser le tableau attributaire

\end{itemize}

\noindent{\hspace*{\fill}\sphinxincludegraphics[width=600\sphinxpxdimen]{{fiche_table}.png}\hspace*{\fill}}

\sphinxAtStartPar
 
\begin{itemize}
\item {} 
\sphinxAtStartPar
faire différents graphiques en fonction des attributs

\end{itemize}

\noindent{\hspace*{\fill}\sphinxincludegraphics[width=600\sphinxpxdimen]{{fiche_graphe}.png}\hspace*{\fill}}

\sphinxAtStartPar
 

\begin{sphinxadmonition}{note}{Note:}
\sphinxAtStartPar
La couche de donnée, le tableau ou encore le graphique peuvent ne pas s’afficher car la donnée est mal configurée coté serveur.
\end{sphinxadmonition}


\subsubsection{Téléchargement de la donnée}
\label{\detokenize{doc_user/catalogue:telechargement-de-la-donnee}}
\sphinxAtStartPar
Vous pouvez aussi télécharger la donnée sous différents formats :

\noindent{\hspace*{\fill}\sphinxincludegraphics[width=600\sphinxpxdimen]{{fiche_tele}.png}\hspace*{\fill}}

\sphinxAtStartPar
 

\sphinxAtStartPar
Mais aussi avoir accées à d’autres liens et URL, ainsi qu’aux flux OGC disponibles :

\noindent{\hspace*{\fill}\sphinxincludegraphics[width=600\sphinxpxdimen]{{fiche_liens}.png}\hspace*{\fill}}

\sphinxAtStartPar
 

\begin{sphinxadmonition}{note}{Note:}
\sphinxAtStartPar
Ces liens sont dépendants de la qualité de la donnée et de son intégration, ils peuvent ne pas fonctionner.
\end{sphinxadmonition}

\sphinxAtStartPar
Vous pouvez aussi visualiser la donnée dans une interface cartographique en cliquant ici et cela vous fera apparaître le {\hyperref[\detokenize{doc_user/visualiseur:visualiseur}]{\sphinxcrossref{\DUrole{std,std-ref}{visualiseur}}}}.

\noindent{\hspace*{\fill}\sphinxincludegraphics[width=600\sphinxpxdimen]{{fiche_carto}.png}\hspace*{\fill}}

\sphinxstepscope


\section{Le visualiseur}
\label{\detokenize{doc_user/visualiseur:le-visualiseur}}\label{\detokenize{doc_user/visualiseur::doc}}\phantomsection\label{\detokenize{doc_user/visualiseur:visualiseur}}
\begin{sphinxShadowBox}
\sphinxstyletopictitle{Table des matières}
\begin{itemize}
\item {} 
\sphinxAtStartPar
\phantomsection\label{\detokenize{doc_user/visualiseur:id1}}{\hyperref[\detokenize{doc_user/visualiseur:introduction}]{\sphinxcrossref{Introduction}}}

\item {} 
\sphinxAtStartPar
\phantomsection\label{\detokenize{doc_user/visualiseur:id2}}{\hyperref[\detokenize{doc_user/visualiseur:la-gestion-des-couches}]{\sphinxcrossref{La gestion des couches}}}

\item {} 
\sphinxAtStartPar
\phantomsection\label{\detokenize{doc_user/visualiseur:id3}}{\hyperref[\detokenize{doc_user/visualiseur:les-fonctionnalites-techniques}]{\sphinxcrossref{Les fonctionnalités techniques}}}

\end{itemize}
\end{sphinxShadowBox}


\subsection{Introduction}
\label{\detokenize{doc_user/visualiseur:introduction}}
\sphinxAtStartPar
Le module cartographique de cette plateforme permet de présenter des couches de données géographiques dans un environnement technique.
Cette interface permet de représenter plusieurs couches géographiques mais ne peut pas se substituer à l’utilisation complète d’un outil SIG bureautique type QGIS.

\sphinxAtStartPar
L’interface se présente comme ceci :

\noindent{\hspace*{\fill}\sphinxincludegraphics[width=600\sphinxpxdimen]{{visu_nbr}.png}\hspace*{\fill}}

\sphinxAtStartPar
 
\begin{itemize}
\item {} 
\sphinxAtStartPar
1 : l’arborescence des couches

\item {} 
\sphinxAtStartPar
2 : recherche d’un lieu

\item {} 
\sphinxAtStartPar
3 : les fonctionnalités

\item {} 
\sphinxAtStartPar
4 : les outils de navigation

\item {} 
\sphinxAtStartPar
5 : les fonds de plans

\end{itemize}

\begin{sphinxadmonition}{note}{Note:}
\sphinxAtStartPar
La donnée peut ne pas s’afficher si elle n’est pas disponible ou alors dans le mauvais référentiel de coordonnée.
\end{sphinxadmonition}


\subsection{La gestion des couches}
\label{\detokenize{doc_user/visualiseur:la-gestion-des-couches}}
\sphinxAtStartPar
Si vous cliquez sur 1, l’arborescence des couches va apparaître et vous pourrez :
\begin{itemize}
\item {} 
\sphinxAtStartPar
rendre visible ou non la couche

\item {} 
\sphinxAtStartPar
modifier l’ordre des couches

\item {} 
\sphinxAtStartPar
modifier l’opacité en pourcentage

\end{itemize}

\noindent{\hspace*{\fill}\sphinxincludegraphics[width=600\sphinxpxdimen]{{visu_couches_details}.png}\hspace*{\fill}}

\sphinxAtStartPar
 


\subsubsection{Ajouter des données dans l’interface}
\label{\detokenize{doc_user/visualiseur:ajouter-des-donnees-dans-l-interface}}
\sphinxAtStartPar
Si vous n’avez pas séléctionné de données, dans l’arborescence des couches, vous pouvez, à l’aide de ces 3 boutons :

\noindent{\hspace*{\fill}\sphinxincludegraphics[width=200\sphinxpxdimen]{{visu_couches_button}.png}\hspace*{\fill}}

\sphinxAtStartPar
 
\begin{itemize}
\item {} 
\sphinxAtStartPar
ajouter des données directement dans le visualiseur, du catalogue interne et d’autre catalogue enregistré \sphinxincludegraphics[width=30\sphinxpxdimen]{{ajout_couche}.png}:

\end{itemize}

\noindent{\hspace*{\fill}\sphinxincludegraphics[width=600\sphinxpxdimen]{{visu_cat}.png}\hspace*{\fill}}

\sphinxAtStartPar
 

\sphinxAtStartPar
Dans cet onglet vous pouvez choisir le catalogue, par défaut, le catalogue est celui de l’office de l’eau mais vous pouvez faire dérouler
la liste pour choisir un autre catalogue. Puis vous pouvez chercher par mots clés des données et les ajouter à l’interface.

\begin{sphinxadmonition}{note}{Note:}
\sphinxAtStartPar
Vous pouvez demander au service informatique de rajouter un catalogue de données géographiques dans cet onglet.
\end{sphinxadmonition}
\begin{itemize}
\item {} 
\sphinxAtStartPar
ajouter des groupes pour vos données avec ce boutton \sphinxincludegraphics[width=30\sphinxpxdimen]{{ajout_group}.png}

\item {} 
\sphinxAtStartPar
créer des annotations \sphinxincludegraphics[width=30\sphinxpxdimen]{{annotation}.png}:

\end{itemize}

\noindent{\hspace*{\fill}\sphinxincludegraphics[width=600\sphinxpxdimen]{{visu_annotation}.png}\hspace*{\fill}}

\sphinxAtStartPar
 


\subsubsection{Changer les paramètre de la couches \sphinxhyphen{} Style \sphinxhyphen{} Informations \sphinxhyphen{} Légende}
\label{\detokenize{doc_user/visualiseur:changer-les-parametre-de-la-couches-style-informations-legende}}
\sphinxAtStartPar
Lorsque vous cliquez sur une couche, plusieurs fonctions apparaissent :

\noindent{\hspace*{\fill}\sphinxincludegraphics[width=500\sphinxpxdimen]{{visu_couches_barre}.png}\hspace*{\fill}}

\sphinxAtStartPar
 

\sphinxAtStartPar
\sphinxstylestrong{Zoomer sur la couche} \sphinxincludegraphics[width=30\sphinxpxdimen]{{button_zoom}.png}

\sphinxAtStartPar
\sphinxstylestrong{Modifier les réglages de la couche} \sphinxincludegraphics[width=30\sphinxpxdimen]{{button_reglage}.png}

\noindent{\hspace*{\fill}\sphinxincludegraphics[width=500\sphinxpxdimen]{{visu_couches_reglages}.png}\hspace*{\fill}}

\sphinxAtStartPar
 

\sphinxAtStartPar
Dans ces réglages vous pouvez modifier, les informations, l’affichage, et surtout modifier le style des couches en cliquant sur la pipette \sphinxincludegraphics[width=30\sphinxpxdimen]{{visu_pinceau_blanco}.png} :

\noindent{\hspace*{\fill}\sphinxincludegraphics[width=600\sphinxpxdimen]{{visu_style_1}.png}\hspace*{\fill}}

\sphinxAtStartPar
 

\sphinxAtStartPar
Si vous ne pouvez pas modifier le style directement il faudra en définir un nouveau et le modifier, cliquez sur le pinceau \sphinxincludegraphics[width=30\sphinxpxdimen]{{button_pinceau}.png} pour définir
un nouveau style puis modifier le en cliquant sur ce boutton \sphinxincludegraphics[width=30\sphinxpxdimen]{{button_modif}.png}.

\noindent{\hspace*{\fill}\sphinxincludegraphics[width=600\sphinxpxdimen]{{visu_styyle}.png}\hspace*{\fill}}

\sphinxAtStartPar
 

\sphinxAtStartPar
Une fois dans l’interface de mofication du style, vous pouvez modifier le style actuel et ajouter d’autres règles. Les styles fonctionnent
avec des règles superposées les unes aux autres, cliquez sur cet icone pour ajouter une règle \sphinxincludegraphics[width=30\sphinxpxdimen]{{logo_rond}.png} et sur cet icone \sphinxincludegraphics[width=30\sphinxpxdimen]{{logo_ento}.png}
pour filtrer le style en fonction des attributs:

\noindent{\hspace*{\fill}\sphinxincludegraphics[width=600\sphinxpxdimen]{{regles_sup}.png}\hspace*{\fill}}

\sphinxAtStartPar
 

\sphinxAtStartPar
Par exemple vous pouvez ajouter une règle qui colore les stations de Saint\sphinxhyphen{}Denis en vert :

\noindent{\hspace*{\fill}\sphinxincludegraphics[width=600\sphinxpxdimen]{{visu_couche_sup}.png}\hspace*{\fill}}

\sphinxAtStartPar
 

\sphinxAtStartPar
Vous pouvez aussi styliser les éléments en fonction d’un attribut, il faut cliquer sur \sphinxincludegraphics[width=30\sphinxpxdimen]{{trois}.png} :

\noindent{\hspace*{\fill}\sphinxincludegraphics[width=500\sphinxpxdimen]{{classif}.png}\hspace*{\fill}}

\sphinxAtStartPar
 

\sphinxAtStartPar
Puis vous pourrez attribuer un style qui classifie les éléments en fonction d’un attribut :

\noindent{\hspace*{\fill}\sphinxincludegraphics[width=500\sphinxpxdimen]{{classif_2}.png}\hspace*{\fill}}

\sphinxAtStartPar
 

\sphinxAtStartPar
Pour enregistrer le style il faudra le valider en cliquant sur \sphinxincludegraphics[width=30\sphinxpxdimen]{{button_valid}.png}.

\sphinxAtStartPar
\sphinxstylestrong{Filtrer les éléments de la couche} \sphinxincludegraphics[width=30\sphinxpxdimen]{{button_filtre}.png}

\noindent{\hspace*{\fill}\sphinxincludegraphics[width=600\sphinxpxdimen]{{visu_filtre}.png}\hspace*{\fill}}

\sphinxAtStartPar
 

\sphinxAtStartPar
Vous pouvez filtrer sur un attribut, filtrer en dessinant une zone géographique, ou encore filtrer en fonction d’un attribut d’une autre couche.

\sphinxAtStartPar
\sphinxstylestrong{Ouvrir la table attributaire} \sphinxincludegraphics[width=30\sphinxpxdimen]{{button_table}.png}

\noindent{\hspace*{\fill}\sphinxincludegraphics[width=600\sphinxpxdimen]{{visuu_table}.png}\hspace*{\fill}}

\sphinxAtStartPar
Vous pouvez filtrer et télécharger le tableau. Il faut ensuite avoir des droits pour modifier et rajouter des éléments, ces modifications se repportent directement dans le catalogue.

\sphinxAtStartPar
 

\sphinxAtStartPar
\sphinxstylestrong{Supprimer la couche} \sphinxincludegraphics[width=30\sphinxpxdimen]{{button_bin}.png}

\sphinxAtStartPar
\sphinxstylestrong{Créer des graphiques} \sphinxincludegraphics[width=30\sphinxpxdimen]{{button_graph}.png}

\noindent{\hspace*{\fill}\sphinxincludegraphics[width=600\sphinxpxdimen]{{widgets}.png}\hspace*{\fill}}

\sphinxAtStartPar
 

\sphinxAtStartPar
Vous pouvez créer 4 types de gaphiques différents, et ensuite les ajouter sur la carte :

\noindent{\hspace*{\fill}\sphinxincludegraphics[width=600\sphinxpxdimen]{{graphiques}.png}\hspace*{\fill}}

\sphinxAtStartPar
 

\sphinxAtStartPar
\sphinxstylestrong{Exporter les données de la couche} \sphinxincludegraphics[width=30\sphinxpxdimen]{{button_down}.png}

\sphinxAtStartPar
\sphinxstylestrong{Afficher les informations de la couche} \sphinxincludegraphics[width=30\sphinxpxdimen]{{button_i}.png}

\begin{sphinxadmonition}{note}{Note:}
\sphinxAtStartPar
Les options sont dépendantes de la donnée, elle peuvent ne pas être toutes disponible en fonction de la donnée.
\end{sphinxadmonition}

\sphinxAtStartPar
Pour les fonds de plans, vous pouvez en changer en cliquant sur l’imagette en bas à gauche; :

\noindent{\hspace*{\fill}\sphinxincludegraphics[width=600\sphinxpxdimen]{{visu_fonds}.png}\hspace*{\fill}}

\sphinxAtStartPar
 


\subsection{Les fonctionnalités techniques}
\label{\detokenize{doc_user/visualiseur:les-fonctionnalites-techniques}}
\sphinxAtStartPar
Pour ce qui est des différentes fonctionnalités :

\noindent{\hspace*{\fill}\sphinxincludegraphics[width=50\sphinxpxdimen]{{visu_fct}.png}\hspace*{\fill}}

\sphinxAtStartPar
 

\sphinxAtStartPar
Dans l’ordre, vous pouvez :

\sphinxAtStartPar
\sphinxstylestrong{Imprimer} une réalisation \sphinxincludegraphics[width=30\sphinxpxdimen]{{print}.png}:

\noindent{\hspace*{\fill}\sphinxincludegraphics[width=600\sphinxpxdimen]{{visu_print}.png}\hspace*{\fill}}

\sphinxAtStartPar
 

\sphinxAtStartPar
Choisir le titre, le format et si la légende apparaît ou non

\sphinxAtStartPar
\sphinxstylestrong{Importer} des données \sphinxincludegraphics[width=30\sphinxpxdimen]{{import1}.png}

\sphinxAtStartPar
\sphinxstylestrong{Exporter} la carte au format WMC, ne peut pas être exporté puis ajouté à QGIS \sphinxincludegraphics[width=30\sphinxpxdimen]{{export}.png}

\sphinxAtStartPar
\sphinxstylestrong{Ajouter} des données à la carte \sphinxincludegraphics[width=30\sphinxpxdimen]{{ajout}.png}

\sphinxAtStartPar
\sphinxstylestrong{Charger} des cartes déjà enregistrées \sphinxincludegraphics[width=30\sphinxpxdimen]{{app}.png}

\sphinxAtStartPar
\sphinxstylestrong{Mesurer} des distances \sphinxincludegraphics[width=30\sphinxpxdimen]{{mesure}.png}

\sphinxAtStartPar
\sphinxstylestrong{Enregistrer} la carte \sphinxincludegraphics[width=30\sphinxpxdimen]{{enreg}.png} :

\noindent{\hspace*{\fill}\sphinxincludegraphics[width=600\sphinxpxdimen]{{visu_download}.png}\hspace*{\fill}}

\sphinxAtStartPar
 

\sphinxAtStartPar
Vous pouvez choisir une imagette, le titre, vous pouvez aussi, en cliquant sur le crayon, définir un texte qui sera visible à l’ouverture de la carte.
Pour définir des droits de lecture et d’édition, vous devez sélectionner un groupe et spécifier si il à les droits de lecture ou d’écriture.
L’enregistrement ira dans la page {\hyperref[\detokenize{doc_user/error:application}]{\sphinxcrossref{\DUrole{std,std-ref}{Application}}}}.

\sphinxAtStartPar
\sphinxstylestrong{Afficher} les réglages
\sphinxstylestrong{Partager} la réalisation
\sphinxstylestrong{Afficher la documentation} la documentation
\sphinxstylestrong{Faire} le tutoriel
\sphinxstylestrong{Effacer} la session

\sphinxstepscope


\section{La cartothèque \sphinxhyphen{} page Application}
\label{\detokenize{doc_user/application:la-cartotheque-page-application}}\label{\detokenize{doc_user/application::doc}}\phantomsection\label{\detokenize{doc_user/application:application}}
\begin{sphinxShadowBox}
\sphinxstyletopictitle{Table des matières}
\begin{itemize}
\item {} 
\sphinxAtStartPar
\phantomsection\label{\detokenize{doc_user/application:id1}}{\hyperref[\detokenize{doc_user/application:introduction}]{\sphinxcrossref{Introduction}}}

\item {} 
\sphinxAtStartPar
\phantomsection\label{\detokenize{doc_user/application:id2}}{\hyperref[\detokenize{doc_user/application:dashboard}]{\sphinxcrossref{Dashboard}}}

\item {} 
\sphinxAtStartPar
\phantomsection\label{\detokenize{doc_user/application:id3}}{\hyperref[\detokenize{doc_user/application:geostory}]{\sphinxcrossref{GeoStory}}}

\end{itemize}
\end{sphinxShadowBox}


\subsection{Introduction}
\label{\detokenize{doc_user/application:introduction}}
\sphinxAtStartPar
La page Application sert de cartotèque en lien avec le {\hyperref[\detokenize{doc_user/visualiseur:visualiseur}]{\sphinxcrossref{\DUrole{std,std-ref}{visualiseur}}}}. Dans cette cartothèque, 4 types de representations sont possibles :
la carte simple avec le visualiseur, le tableau de bord, la GeoStory et le contexte réservé aux administrateurs.

\noindent{\hspace*{\fill}\sphinxincludegraphics[width=600\sphinxpxdimen]{{app_global}.png}\hspace*{\fill}}

\sphinxAtStartPar
 

\sphinxAtStartPar
Dans cette cartothèque, 4 types de representations sont possibles :
la carte simple avec le visualiseur, le tableau de bord, la GeoStory et le contexte réservé aux administrateurs.

\noindent{\hspace*{\fill}\sphinxincludegraphics[width=200\sphinxpxdimen]{{buttons}.png}\hspace*{\fill}}


\subsection{Dashboard}
\label{\detokenize{doc_user/application:dashboard}}
\sphinxAtStartPar
Pour créer un dashboard, cliquez sur ce bouton \sphinxincludegraphics[width=30\sphinxpxdimen]{{dash}.png}

\noindent{\hspace*{\fill}\sphinxincludegraphics[width=600\sphinxpxdimen]{{app_dashboard}.png}\hspace*{\fill}}

\sphinxAtStartPar
 

\sphinxAtStartPar
Vous pouvez ajouter différents widgets en fonctions des données du catalogue, un tutoriel vous guide directement
lorsque vous créer un dashboard, les graphiques réalisables sont les mêmes que pour les cartes. Les widgets se connectent
aux données du serveur interne et non directement aux cartes réalisées.

\begin{sphinxadmonition}{note}{Note:}
\sphinxAtStartPar
Les widgets sont dépéndants de la configuration de la donnée, ils peuvent ne pas être disponible.
\end{sphinxadmonition}

\sphinxAtStartPar
Voici le lien de la documentation officiel pour aller dans le détail :

\sphinxAtStartPar
\sphinxhref{https://docs.mapstore.geosolutionsgroup.com/en/v2024.01.02/user-guide/exploring-dashboards/}{Documentation Mapstore Dashboard}


\subsection{GeoStory}
\label{\detokenize{doc_user/application:geostory}}
\sphinxAtStartPar
Pour créer un dashboard, cliquez sur ce bouton \sphinxincludegraphics[width=30\sphinxpxdimen]{{geostory}.png}

\noindent{\hspace*{\fill}\sphinxincludegraphics[width=600\sphinxpxdimen]{{app_geostory}.png}\hspace*{\fill}}

\sphinxAtStartPar
 

\sphinxAtStartPar
Avec les GeoStories, vous pouvez créer des documents textes en y intégrant des cartes intéractives.
La gestion des composants se fait sur la gauche de l’interface qui sont : les titres, les bannières,
les paragraphes, les sections immersives, les geocarrousels, les sections multimedia et les pages web.
Un tutoriel vous guide directement lorsque vous créez une GeoStory.

\sphinxAtStartPar
Voici le lien de la documentation officiel pour aller dans le détail :

\sphinxAtStartPar
\sphinxhref{https://docs.mapstore.geosolutionsgroup.com/en/v2024.01.02/user-guide/exploring-stories/}{Documentation Mapstore GeoStory}

\sphinxstepscope


\section{L’intégration de données \sphinxhyphen{} page Import}
\label{\detokenize{doc_user/import:l-integration-de-donnees-page-import}}\label{\detokenize{doc_user/import::doc}}
\begin{sphinxShadowBox}
\sphinxstyletopictitle{Table des matières}
\begin{itemize}
\item {} 
\sphinxAtStartPar
\phantomsection\label{\detokenize{doc_user/import:id1}}{\hyperref[\detokenize{doc_user/import:introduction}]{\sphinxcrossref{Introduction}}}

\item {} 
\sphinxAtStartPar
\phantomsection\label{\detokenize{doc_user/import:id2}}{\hyperref[\detokenize{doc_user/import:integration-de-shapefile}]{\sphinxcrossref{Intégration de shapefile}}}

\item {} 
\sphinxAtStartPar
\phantomsection\label{\detokenize{doc_user/import:id3}}{\hyperref[\detokenize{doc_user/import:integration-de-csv}]{\sphinxcrossref{Intégration de CSV}}}

\item {} 
\sphinxAtStartPar
\phantomsection\label{\detokenize{doc_user/import:id4}}{\hyperref[\detokenize{doc_user/import:processus-d-integration}]{\sphinxcrossref{Processus d’intégration}}}

\end{itemize}
\end{sphinxShadowBox}


\subsection{Introduction}
\label{\detokenize{doc_user/import:introduction}}
\sphinxAtStartPar
La page Import permet d’intégrer des données de manière simplifiée dans le catalogue.

\noindent{\hspace*{\fill}\sphinxincludegraphics[width=600\sphinxpxdimen]{{import}.png}\hspace*{\fill}}

\sphinxAtStartPar
 

\sphinxAtStartPar
Deux format sont acceptés, le shapefile en zip et le CSV, à une limite de 500 Mo. Vous pouvez ajouter votre donnée, cliquer sur
le bouton « J’ai le droit de publier cette donnée » puis passer à l’étape suivante. Une bonne pratique est de ne pas mettre de caractères spéciaux
dans le nom des champs des couches shapefile, cela peut causer des problèmes par la suite.


\subsection{Intégration de shapefile}
\label{\detokenize{doc_user/import:integration-de-shapefile}}
\sphinxAtStartPar
Les particularités d’un shapefile est la projection et l’encodage:

\noindent{\hspace*{\fill}\sphinxincludegraphics[width=700\sphinxpxdimen]{{import_proj}.png}\hspace*{\fill}}

\sphinxAtStartPar
 

\sphinxAtStartPar
Pour bien renseigner la donnée, assurez vous que le carré orange qui represente l’emprise de votre donnée est au bon endroit et qu’une projection est renseignée.
De même pour l’encodage, si votre exemple d’objet possède des carractères illisibles, vous pouvez changer l’encodage.

\begin{sphinxadmonition}{note}{Note:}
\sphinxAtStartPar
Si pour une donnée, aucune projection n’est valide, veuillez le faire remonter au service informatique.
\end{sphinxadmonition}


\subsection{Intégration de CSV}
\label{\detokenize{doc_user/import:integration-de-csv}}
\sphinxAtStartPar
La particularité d’un CSV est la geométrie :

\noindent{\hspace*{\fill}\sphinxincludegraphics[width=700\sphinxpxdimen]{{import_csv}.png}\hspace*{\fill}}

\sphinxAtStartPar
 

\sphinxAtStartPar
Pour bien renseigner la donnée, vous pouvez choisir le séparateur de colonne, de texte et aussi renseigner une geométrie ou non. Pour ajouter une geométrie,
il faut obligatoirement un champ latitude et longitude dans le bon format comme sur la photo ci\sphinxhyphen{}dessus.


\subsection{Processus d’intégration}
\label{\detokenize{doc_user/import:processus-d-integration}}
\sphinxAtStartPar
Vous pouvez ensuite ajouter un titre et une description :

\noindent{\hspace*{\fill}\sphinxincludegraphics[width=600\sphinxpxdimen]{{import_shape_titre}.png}\hspace*{\fill}}

\sphinxAtStartPar
 

\sphinxAtStartPar
Pour l’ajout de mots clés, ils sont prédéfinis dans un catalogue de mots clés.
Pour faire apparaître la liste déroulante il faut cliquer sur le carré blanc, ou alors commencer à écrire un mot puis cliquer à la suite pour voir l’autocomplétion.

\noindent{\hspace*{\fill}\sphinxincludegraphics[width=600\sphinxpxdimen]{{import_shape_keyword}.png}\hspace*{\fill}}

\sphinxAtStartPar
 

\sphinxAtStartPar
Ensuite vient la date de création, elle se renseigne automatiquement mais vous pouvez la changer si la donnée est antérieur.

\noindent{\hspace*{\fill}\sphinxincludegraphics[width=600\sphinxpxdimen]{{import_shape_time}.png}\hspace*{\fill}}

\sphinxAtStartPar
 

\sphinxAtStartPar
En dernier, il faut décrire le processus de création de la donnée :

\noindent{\hspace*{\fill}\sphinxincludegraphics[width=600\sphinxpxdimen]{{import_shape_processus}.png}\hspace*{\fill}}

\sphinxAtStartPar
 

\sphinxAtStartPar
Et vous avez un récapitulatif de votre intégration, cliquez sur « publier » pour intégrer la donnée dans le catalogue.

\noindent{\hspace*{\fill}\sphinxincludegraphics[width=600\sphinxpxdimen]{{import_shape_pub}.png}\hspace*{\fill}}

\sphinxstepscope


\section{Le plugin QGIS}
\label{\detokenize{doc_user/qgis:le-plugin-qgis}}\label{\detokenize{doc_user/qgis::doc}}
\sphinxAtStartPar
Le module QGIS vous permet d’intéragir avec certaines données de la plateforme directement dans QGIS.
Le but est de mettre à disposition des données de référence stable et sans pouvoir les modifier.
Il suffit de double cliquer sur une couche pour l’afficher sur QGIS.

\noindent{\hspace*{\fill}\sphinxincludegraphics[width=600\sphinxpxdimen]{{plugin_base}.png}\hspace*{\fill}}

\sphinxAtStartPar
 

\sphinxAtStartPar
Les couches sont choisis par le service informatique, n’hésitez pas à faire des demandes pour que certaines couches se retrouve sur cette interface.

\sphinxAtStartPar
Pour installer le plugin, demandez au service informatique de vous fournir le document zip qui contient le code du plugin et le copier coller dans
le repertoire \sphinxcode{\sphinxupquote{C:\textbackslash{}Users\textbackslash{}votre\_nom\textbackslash{}AppData\textbackslash{}Roaming\textbackslash{}QGIS\textbackslash{}QGIS3\textbackslash{}profiles\textbackslash{}default\textbackslash{}python\textbackslash{}plugins}} et dézipper le. Puis relancer QGIS, vous verrez l’interface
à droite de l’écran, si il n’apparait pas directement ou si vous l’avez supprimé, vous pouvez l’afficher en cliquant sur « Extensions », « Office de l’eau »
puis « Afficher le panneau latéral » :

\noindent{\hspace*{\fill}\sphinxincludegraphics[width=600\sphinxpxdimen]{{plugin_aff}.png}\hspace*{\fill}}

\sphinxAtStartPar
 

\sphinxAtStartPar
Puis, dans les paramètres de l’extension (menu Extensions  Office de l’eau \textgreater{} Paramétrer le plugin), il faut cocher
« Télécharger le fichier à chaque lancement de QGiS » :

\noindent{\hspace*{\fill}\sphinxincludegraphics[width=600\sphinxpxdimen]{{plugin_down}.png}\hspace*{\fill}}

\sphinxAtStartPar
 

\begin{sphinxadmonition}{note}{Note:}
\sphinxAtStartPar
Les données afficher dans qgis sont séléctionnées par le service informatique, si vous voulez ajouter des données,
veillez demander au service informatique.
\end{sphinxadmonition}

\sphinxstepscope


\section{Les erreurs fréquentes}
\label{\detokenize{doc_user/error:les-erreurs-frequentes}}\label{\detokenize{doc_user/error::doc}}\phantomsection\label{\detokenize{doc_user/error:application}}
\sphinxAtStartPar
Les erreurs.

\sphinxstepscope


\chapter{Documentation administrateur}
\label{\detokenize{doc_admin:documentation-administrateur}}\label{\detokenize{doc_admin::doc}}
\sphinxAtStartPar
\sphinxhref{plateformecartographiqueole.pdf}{Documentation au format PDF}.

\sphinxstepscope


\section{Catalogue \sphinxhyphen{} GeoNetwork}
\label{\detokenize{doc_admin/catalogue:catalogue-geonetwork}}\label{\detokenize{doc_admin/catalogue::doc}}
\begin{sphinxShadowBox}
\sphinxstyletopictitle{Table des matières}
\begin{itemize}
\item {} 
\sphinxAtStartPar
\phantomsection\label{\detokenize{doc_admin/catalogue:id2}}{\hyperref[\detokenize{doc_admin/catalogue:introduction}]{\sphinxcrossref{Introduction}}}

\item {} 
\sphinxAtStartPar
\phantomsection\label{\detokenize{doc_admin/catalogue:id3}}{\hyperref[\detokenize{doc_admin/catalogue:gestion-des-fiches-de-metadonnees}]{\sphinxcrossref{Gestion des fiches de métadonnées}}}

\item {} 
\sphinxAtStartPar
\phantomsection\label{\detokenize{doc_admin/catalogue:id4}}{\hyperref[\detokenize{doc_admin/catalogue:gerer-les-droits-d-acces-aux-fiches-de-metadonnees}]{\sphinxcrossref{Gérer les droits d’accès aux fiches de métadonnées}}}

\item {} 
\sphinxAtStartPar
\phantomsection\label{\detokenize{doc_admin/catalogue:id5}}{\hyperref[\detokenize{doc_admin/catalogue:creer-une-fiche-a-la-main}]{\sphinxcrossref{Créer une fiche à la main}}}

\item {} 
\sphinxAtStartPar
\phantomsection\label{\detokenize{doc_admin/catalogue:id6}}{\hyperref[\detokenize{doc_admin/catalogue:administration}]{\sphinxcrossref{Administration}}}

\end{itemize}
\end{sphinxShadowBox}


\subsection{Introduction}
\label{\detokenize{doc_admin/catalogue:introduction}}
\sphinxAtStartPar
La technologie utilisée par le catalogue est GeoNetwork, cette documentation à pour but résumer rapidement les différentes options de l’interface mais n’a pas
pour vocation de remplacer la documentation officiel :
\sphinxurl{https://docs.geonetwork-opensource.org/4.2/user-guide/}

\sphinxAtStartPar
Le GeoNetwork est utilisé comme catalogue CSW (Catalogue Service for the Web) ce qui permet de référencer les métadonnées couplées aux flux de données.

\sphinxAtStartPar
La page principale se compose de 4 composants : la recherche de données, la visualisation, les fiches de métadonnées et l’administration :

\noindent{\hspace*{\fill}\sphinxincludegraphics[width=700\sphinxpxdimen]{{cat_barre}.png}\hspace*{\fill}}

\sphinxAtStartPar
 

\sphinxAtStartPar
La recherche de données est la même que dans le catalogue mais avec l’interface basique de GeoNetwork.
La visualisation renvoie sur le visualisateur qui est MapStore.


\subsection{Gestion des fiches de métadonnées}
\label{\detokenize{doc_admin/catalogue:gestion-des-fiches-de-metadonnees}}
\sphinxAtStartPar
Dans l’onglet « Contribuer » puis « Accueil édition » :

\noindent{\hspace*{\fill}\sphinxincludegraphics[width=700\sphinxpxdimen]{{cat_meta}.png}\hspace*{\fill}}

\sphinxAtStartPar
 

\sphinxAtStartPar
Cette section fournit une liste des fiches avec les fonctionnalités associées, vous pouvez éditer les fiches, les supprimer,
gérer les annuaires (inutile pour geOrchestra), faire de l’édition en série et gérer les droits d’accès.

\noindent{\hspace*{\fill}\sphinxincludegraphics[width=700\sphinxpxdimen]{{cat_fiche}.png}\hspace*{\fill}}

\sphinxAtStartPar
 

\sphinxAtStartPar
Dans l’interface d’édition d’une fiche, vous pouvez changer toutes les informations à gauche de l’écran, et ajouter des éléments à droite.
Les ajouts peuvent être des images, des liens ou des ressources qui correspondent à des liens de parentés, des flux OGC ou d’autres.


\subsection{Gérer les droits d’accès aux fiches de métadonnées}
\label{\detokenize{doc_admin/catalogue:gerer-les-droits-d-acces-aux-fiches-de-metadonnees}}\phantomsection\label{\detokenize{doc_admin/catalogue:privileges}}
\sphinxAtStartPar
Vous pouvez restraindre l’accès aux fiches de métadonnée, les fiches sont automatiquement visible pour toutes les organisations de l’infrastructure.
Si vou voulez modifier les différents droits en fonction des organisations, il faut aller dans la fiche de métadonnée que vous voulez modifier,
allez dans « Gérer la fiche » puis « Privilèges » et vous pourrez modifier les accès :

\noindent{\hspace*{\fill}\sphinxincludegraphics[width=700\sphinxpxdimen]{{cat_gerer}.png}\hspace*{\fill}}

\sphinxAtStartPar
 

\sphinxAtStartPar
Vous pouvez modifier l’accès à la consultation simple ou encore, la visualisation, le téléchargement, l’édition ou la notification en fonction des organismes.

\noindent{\hspace*{\fill}\sphinxincludegraphics[width=700\sphinxpxdimen]{{cat_privileges}.png}\hspace*{\fill}}

\sphinxAtStartPar
 


\subsection{Créer une fiche à la main}
\label{\detokenize{doc_admin/catalogue:creer-une-fiche-a-la-main}}
\sphinxAtStartPar
Pour créer une fiche à la main, vous pouvez cliquer sur « Contribuer » puis « Ajouter une fiche », choisir « Template for Vector data ISO19139 » :

\noindent{\hspace*{\fill}\sphinxincludegraphics[width=700\sphinxpxdimen]{{ajout_fiche}.png}\hspace*{\fill}}

\sphinxAtStartPar
 

\sphinxAtStartPar
Il faut modifier l’intitulé de la ressource et ajouter les flux WMS et WMTS en cliquant sur « Ajouter » et « Créer un lien vers une ressource « :

\noindent{\hspace*{\fill}\sphinxincludegraphics[width=700\sphinxpxdimen]{{ajout_lien}.png}\hspace*{\fill}}

\sphinxAtStartPar
 

\sphinxAtStartPar
Puis renseigner « OGC\sphinxhyphen{}WMS Web Map Service » et « OGC\sphinxhyphen{}WFS Web Features Service » dans « Protocol » pour avoir un flux WMS et WFS par donnée et le lien vers le geoserver.
Le lien se construit par le fqdn suivis de « geoserver » puis de l’organisation abrégé qui à intégré la donnée avec le datafeeder, par exemple « ole » et enfin « ows ».
Ce qui peux donner : \sphinxcode{\sphinxupquote{https://dev\sphinxhyphen{}carto.ole.re/geoserver/psc/ows}} et choisir la donnée.

\noindent{\hspace*{\fill}\sphinxincludegraphics[width=700\sphinxpxdimen]{{create_link}.png}\hspace*{\fill}}

\sphinxAtStartPar
 


\subsection{Administration}
\label{\detokenize{doc_admin/catalogue:administration}}
\begin{sphinxShadowBox}
\begin{itemize}
\item {} 
\sphinxAtStartPar
\phantomsection\label{\detokenize{doc_admin/catalogue:id7}}{\hyperref[\detokenize{doc_admin/catalogue:metadonnees-et-modeles}]{\sphinxcrossref{Métadonnées et modèles}}}

\item {} 
\sphinxAtStartPar
\phantomsection\label{\detokenize{doc_admin/catalogue:id8}}{\hyperref[\detokenize{doc_admin/catalogue:utilisateur-et-groupe}]{\sphinxcrossref{Utilisateur et groupe}}}

\item {} 
\sphinxAtStartPar
\phantomsection\label{\detokenize{doc_admin/catalogue:id9}}{\hyperref[\detokenize{doc_admin/catalogue:moissonnage}]{\sphinxcrossref{Moissonnage}}}

\item {} 
\sphinxAtStartPar
\phantomsection\label{\detokenize{doc_admin/catalogue:id10}}{\hyperref[\detokenize{doc_admin/catalogue:statistique-et-statut}]{\sphinxcrossref{Statistique et statut}}}

\item {} 
\sphinxAtStartPar
\phantomsection\label{\detokenize{doc_admin/catalogue:id11}}{\hyperref[\detokenize{doc_admin/catalogue:rapports}]{\sphinxcrossref{Rapports}}}

\item {} 
\sphinxAtStartPar
\phantomsection\label{\detokenize{doc_admin/catalogue:id12}}{\hyperref[\detokenize{doc_admin/catalogue:thesaurus}]{\sphinxcrossref{Thesaurus}}}

\item {} 
\sphinxAtStartPar
\phantomsection\label{\detokenize{doc_admin/catalogue:id13}}{\hyperref[\detokenize{doc_admin/catalogue:parametres}]{\sphinxcrossref{Paramètres}}}

\item {} 
\sphinxAtStartPar
\phantomsection\label{\detokenize{doc_admin/catalogue:id14}}{\hyperref[\detokenize{doc_admin/catalogue:outils}]{\sphinxcrossref{Outils}}}

\end{itemize}
\end{sphinxShadowBox}

\sphinxAtStartPar
Pour ce qui est de l’administration, elle est divisé en 8 catégories :

\noindent{\hspace*{\fill}\sphinxincludegraphics[width=700\sphinxpxdimen]{{cat_admin_parties}.png}\hspace*{\fill}}

\sphinxAtStartPar
 


\subsubsection{Métadonnées et modèles}
\label{\detokenize{doc_admin/catalogue:metadonnees-et-modeles}}\begin{quote}

\sphinxAtStartPar
La page « Métadonnées et modèles » sert à définir les modèles de fiches de métadonnées à utiliser :

\noindent{\hspace*{\fill}\sphinxincludegraphics[width=700\sphinxpxdimen]{{cat_modele}.png}\hspace*{\fill}}
\end{quote}

\sphinxAtStartPar
 

\sphinxAtStartPar
Les modèles de fiches de métadonnées sont gérées automatiquement par le module d’import de geOrchestra.


\subsubsection{Utilisateur et groupe}
\label{\detokenize{doc_admin/catalogue:utilisateur-et-groupe}}\begin{quote}

\noindent{\hspace*{\fill}\sphinxincludegraphics[width=700\sphinxpxdimen]{{cat_user}.png}\hspace*{\fill}}
\end{quote}

\sphinxAtStartPar
 

\sphinxAtStartPar
Les utilisateurs et les organisations sont gérés dans la page {\hyperref[\detokenize{doc_admin/utilisateurs:utilisateur}]{\sphinxcrossref{\DUrole{std,std-ref}{Utilisateur}}}}


\subsubsection{Moissonnage}
\label{\detokenize{doc_admin/catalogue:moissonnage}}\begin{quote}

\noindent{\hspace*{\fill}\sphinxincludegraphics[width=700\sphinxpxdimen]{{cat_moisson}.png}\hspace*{\fill}}
\end{quote}

\sphinxAtStartPar
 

\sphinxAtStartPar
Le moissonnage est très utile car il permet de référencer les fiches de métadonnées d’un autre catalogue sur le GeoNetwork interne.
Il faut connaître la technologie du catalogue que l’on veut référencer, renseigner l’url puis les différents filtres que l’on veut appliquer.
Il est aussi possible de plannifier le moissonnage.

\sphinxAtStartPar
Les moissonnages sont différents en fonction de la technologie du catalogue cible.
Voici la documentation officiel pour chaque technologie :

\sphinxAtStartPar
\sphinxurl{https://docs.geonetwork-opensource.org/4.2/user-guide/harvesting/}


\subsubsection{Statistique et statut}
\label{\detokenize{doc_admin/catalogue:statistique-et-statut}}\begin{quote}

\noindent{\hspace*{\fill}\sphinxincludegraphics[width=700\sphinxpxdimen]{{cat_stats}.png}\hspace*{\fill}}
\end{quote}

\sphinxAtStartPar
 

\sphinxAtStartPar
Cette section permet de connaître l’état du système très rapidement. L’analyse des liens scanne tous les liens des métadonnées,
le versionnement permet de connaître l’état d’une métadonnée précise.


\subsubsection{Rapports}
\label{\detokenize{doc_admin/catalogue:rapports}}\begin{quote}

\noindent{\hspace*{\fill}\sphinxincludegraphics[width=700\sphinxpxdimen]{{cat_rapport}.png}\hspace*{\fill}}
\end{quote}

\sphinxAtStartPar
 

\sphinxAtStartPar
La partie rapport permet de créer des rapports très rapidement :
\begin{itemize}
\item {} 
\sphinxAtStartPar
sur la mise à jour des fiches

\item {} 
\sphinxAtStartPar
sur les fiches stockées en interne

\item {} 
\sphinxAtStartPar
sur l’ajout de fichier dans les fiches

\item {} 
\sphinxAtStartPar
sur l’historique des fiches

\item {} 
\sphinxAtStartPar
sur les accès utilisateurs

\end{itemize}


\subsubsection{Thesaurus}
\label{\detokenize{doc_admin/catalogue:thesaurus}}\phantomsection\label{\detokenize{doc_admin/catalogue:id1}}\begin{quote}

\noindent{\hspace*{\fill}\sphinxincludegraphics[width=700\sphinxpxdimen]{{cat_thes}.png}\hspace*{\fill}}
\end{quote}

\sphinxAtStartPar
 

\sphinxAtStartPar
Le thesaurus est le dictionnaire à mots clés, il définit les mots clés que vous pouvez utiliser pour vos métadonnées. Il est utilisé dans le (module d’import de données)
lors du choix des mots clés. Par defaut dans geOrchestra, le thesaurus est définis sur les thèmes INSPIRE, vous pouvez le modifier en ajoutant un thesaurus à la main
dans cette interface puis modifier le code qui relie le thesurus au datafeeder.

\sphinxAtStartPar
Par exemple, pour ajouter le glossaire de l’Office internationale de l’eau, il faut télécharger le glossaire au format RDF\sphinxhyphen{}XML et cliquer sur « Ajouter un thesaurus » :
\begin{quote}

\noindent{\hspace*{\fill}\sphinxincludegraphics[width=700\sphinxpxdimen]{{thesaurus}.png}\hspace*{\fill}}
\end{quote}

\sphinxAtStartPar
 

\sphinxAtStartPar
Puis s’assurer que le thesaurus à bien chargé, il peut contenir des valeurs manquantes, le déplier au maximum pour voir les lignes. Pour le glossaire de l’Office
internationale de l’eau, lorsque l’on charge au maximum le thesaurus, on le voit en entier même si des « valeurs manquantes » apparaît.

\sphinxAtStartPar
Ensuite, pour l’utiliser dans le datafeeder il faut modifier la ligne dans le fichier \sphinxcode{\sphinxupquote{/etc/georchestra/datafeeder/frontend\sphinxhyphen{}config.json}}:

\begin{sphinxVerbatim}[commandchars=\\\{\}]
\PYG{l+s+s2}{\PYGZdq{}thesaurusUrl\PYGZdq{}}:\PYG{+w}{ }\PYG{l+s+s2}{\PYGZdq{}}\PYG{l+s+s2}{https://dev\PYGZhy{}carto.ole.re/geonetwork/srv/api/registries/vocabularies/search?type=CONTAINS\PYGZam{}thesaurus=local.theme.glossaire\PYGZus{}eau\PYGZus{}biodiv\PYGZus{}20241021\PYGZam{}rows=20000\PYGZam{}q=}\PYG{l+s+si}{\PYGZdl{}\PYGZob{}}\PYG{n+nv}{q}\PYG{l+s+si}{\PYGZcb{}}\PYG{l+s+s2}{\PYGZam{}uri=**\PYGZam{}lang=}\PYG{l+s+si}{\PYGZdl{}\PYGZob{}}\PYG{n+nv}{lang}\PYG{l+s+si}{\PYGZcb{}}\PYG{l+s+s2}{\PYGZdq{}}
\end{sphinxVerbatim}

\sphinxAtStartPar
En modifiant l’url en fonction du domaine, l’origine de thesaurus, « local » ou « externe », le type qui est ici « theme », le nom, ici
« glossaire\_eau\_biodiv\_20241021 » et ne pas hésiter à rajouter des lignes si le thesaurus est long comme celui de L’oieau : « rows=20000 ».

\sphinxAtStartPar
Puis relancer le datafeeder :

\begin{sphinxVerbatim}[commandchars=\\\{\}]
systemctl\PYG{+w}{ }restart\PYG{+w}{ }datafeeder.service
\end{sphinxVerbatim}


\subsubsection{Paramètres}
\label{\detokenize{doc_admin/catalogue:parametres}}\begin{quote}

\noindent{\hspace*{\fill}\sphinxincludegraphics[width=700\sphinxpxdimen]{{cat_param}.png}\hspace*{\fill}}
\end{quote}

\sphinxAtStartPar
 

\sphinxAtStartPar
Dans cet onglet se trouve les paramètres pour la configuration système dont voici la documentation en details :

\sphinxAtStartPar
\sphinxurl{https://docs.geonetwork-opensource.org/4.2/fr/administrator-guide/configuring-the-catalog/system-configuration/}

\sphinxAtStartPar
Sur cette partie se trouve aussi les paramètre pour changer l’interface utilisateur, changer le style, ajouter des logos, gérer les différents catalogues moissonnés,
gérer les différentes langues, activer et tester le CSW, ajouter des serveurs cartographiques type GeoServer et ajouter des pages statiques.


\subsubsection{Outils}
\label{\detokenize{doc_admin/catalogue:outils}}\begin{quote}

\noindent{\hspace*{\fill}\sphinxincludegraphics[width=700\sphinxpxdimen]{{cat_outil}.png}\hspace*{\fill}}
\end{quote}

\sphinxAtStartPar
 

\sphinxAtStartPar
Cette partie permet d’inéragir avec les indexs d’elasticsearch qui est le moteur de recherche derrière GeoNetwork. Cela permet de relancer l’indexation
des données. ll ne faut globalement pas cliquer sur ces boutons.

\sphinxstepscope


\section{MapStore}
\label{\detokenize{doc_admin/visualiseur:mapstore}}\label{\detokenize{doc_admin/visualiseur::doc}}
\begin{sphinxShadowBox}
\sphinxstyletopictitle{Table des matières}
\begin{itemize}
\item {} 
\sphinxAtStartPar
\phantomsection\label{\detokenize{doc_admin/visualiseur:id1}}{\hyperref[\detokenize{doc_admin/visualiseur:introduction}]{\sphinxcrossref{Introduction}}}

\item {} 
\sphinxAtStartPar
\phantomsection\label{\detokenize{doc_admin/visualiseur:id2}}{\hyperref[\detokenize{doc_admin/visualiseur:les-contextes}]{\sphinxcrossref{Les contextes}}}

\end{itemize}
\end{sphinxShadowBox}


\subsection{Introduction}
\label{\detokenize{doc_admin/visualiseur:introduction}}
\sphinxAtStartPar
La partie administrateur du module Mapstore ajoute peu de fonctionnalités, la seule fonction en plus est la création de contextes, et l’accès à la création d’un style
dans mapstore si l’on ajoute le rôle Administrateur de geoserver.


\subsection{Les contextes}
\label{\detokenize{doc_admin/visualiseur:les-contextes}}
\sphinxAtStartPar
Les contextes permettent de créer des cartes en choisissant l’interface finale pour par exemple rendre la carte plus abordable et moins technique.
Par exemple avec ce contexte qui ne presente que le bouton accueil, télécharger et importer :

\noindent{\hspace*{\fill}\sphinxincludegraphics[width=600\sphinxpxdimen]{{mapstore_contexte}.png}\hspace*{\fill}}

\sphinxAtStartPar
 

\sphinxAtStartPar
Un tutoriel est automatiquement lancé lorsque vous créer un contexte et vous guide pas à pas dans la création
Il faut commencer par choisir un titre, une description, ajouter les données que l’on veut afficher,
et choisir les fonctions :

\noindent{\hspace*{\fill}\sphinxincludegraphics[width=600\sphinxpxdimen]{{mapstore_plugins}.png}\hspace*{\fill}}

\sphinxAtStartPar
 

\sphinxAtStartPar
Les fonctions à choisir sont explicite et facile à comprendre.
Enfin il reste à enregister le contexte pour le rendre disponible aux groupes que l’on veut.

\sphinxstepscope


\section{Services \sphinxhyphen{} GeoServer}
\label{\detokenize{doc_admin/services:services-geoserver}}\label{\detokenize{doc_admin/services::doc}}
\begin{sphinxShadowBox}
\sphinxstyletopictitle{Table des matières}
\begin{itemize}
\item {} 
\sphinxAtStartPar
\phantomsection\label{\detokenize{doc_admin/services:id1}}{\hyperref[\detokenize{doc_admin/services:introduction}]{\sphinxcrossref{Introduction}}}

\item {} 
\sphinxAtStartPar
\phantomsection\label{\detokenize{doc_admin/services:id2}}{\hyperref[\detokenize{doc_admin/services:les-donnees-stockees-en-interne}]{\sphinxcrossref{Les données stockées en interne}}}

\item {} 
\sphinxAtStartPar
\phantomsection\label{\detokenize{doc_admin/services:id3}}{\hyperref[\detokenize{doc_admin/services:la-diffusion-des-donnees-avec-les-flux-ogc}]{\sphinxcrossref{La diffusion des données avec les flux OGC}}}

\item {} 
\sphinxAtStartPar
\phantomsection\label{\detokenize{doc_admin/services:id4}}{\hyperref[\detokenize{doc_admin/services:definir-des-styles-pour-les-flux-wms}]{\sphinxcrossref{Définir des styles pour les flux WMS}}}

\item {} 
\sphinxAtStartPar
\phantomsection\label{\detokenize{doc_admin/services:id5}}{\hyperref[\detokenize{doc_admin/services:la-restriction-d-acces-aux-donnees}]{\sphinxcrossref{La restriction d’accès aux données}}}

\item {} 
\sphinxAtStartPar
\phantomsection\label{\detokenize{doc_admin/services:id6}}{\hyperref[\detokenize{doc_admin/services:la-restriction-d-acces-aux-services}]{\sphinxcrossref{La restriction d’accès aux services}}}

\end{itemize}
\end{sphinxShadowBox}


\subsection{Introduction}
\label{\detokenize{doc_admin/services:introduction}}
\sphinxAtStartPar
Cette page est l’interface de GeoServer, le GeoServer est la technologie qui permet de diffuser les données stockées en interne via des webs services.
Voici la documentation officiel :
\sphinxurl{https://docs.geoserver.org/}

\sphinxAtStartPar
Vous n’avez normalement pas à intervenir dans cette page, à part pour des changements sur la configuration des différents flux.


\subsection{Les données stockées en interne}
\label{\detokenize{doc_admin/services:les-donnees-stockees-en-interne}}
\sphinxAtStartPar
GeoServer est directement connecté à une base de données PostGIS et diffuse les données internes :

\noindent{\hspace*{\fill}\sphinxincludegraphics[width=700\sphinxpxdimen]{{geos_interface}.png}\hspace*{\fill}}

\sphinxAtStartPar
 

\sphinxAtStartPar
Les données sont organisées en « espaces de travail » qui prend le nom de l’organisation à qui appartient la donnée, puis est reliée à un entrepot, qui est l’emplacement
dans la base de données de là où est stocké la donnée. Cette organisation et ce stockage se fait automatiquement avec le module d’importation de geOrchestra.


\subsection{La diffusion des données avec les flux OGC}
\label{\detokenize{doc_admin/services:la-diffusion-des-donnees-avec-les-flux-ogc}}
\sphinxAtStartPar
Lorsqu’une donnée est intégrée dans geOrchestra via le module d’intégration, deux types de services sont créés : un flux WMS (Web Map Service)
et un flux WFS (Web Feature Service).

\sphinxAtStartPar
WMS (Web Map Service) : Ce service permet de représenter la donnée sous forme de cartes raster (images générées à partir des données géospatiales).
Les couches WMS sont souvent utilisées pour l’affichage dans des visualiseurs cartographiques, car elles sont légères et rapides à charger.

\sphinxAtStartPar
WFS (Web Feature Service) : Ce service permet d’accéder aux données vectorielles, offrant la possibilité de requêter et de manipuler directement
les entités géospatiales (points, lignes, polygones). Le WFS est essentiel pour effectuer des requêtes sur les objets géospatiaux et obtenir des informations
précises sur ces entités.

\sphinxAtStartPar
Ces services sont conformes aux normes européennes et permettent une interopérabilité entre différents modules et systèmes. Lorsqu’une
donnée apparaît sur le visualiseur, elle est généralement issue du flux WMS pour des raisons de performance, car les données raster sont plus
rapides et moins gourmandes en ressources. Toutefois, le flux WFS est crucial pour permettre des interactions plus détaillées, telles que des requêtes sur les entités.
Vous pouvez configurer ces flux dans GeoServer en accédant aux paramètres du service, par exemple pour définir les autorisations ou activer/désactiver la
transformation du système de coordonnées de référence (CRS). Cela vous permet de contrôler précisément comment les données sont diffusées et utilisées au sein de la plateforme.


\subsection{Définir des styles pour les flux WMS}
\label{\detokenize{doc_admin/services:definir-des-styles-pour-les-flux-wms}}
\sphinxAtStartPar
Vous pouvez définir des styles dans l’onglet « Styles » de GeoServer puis les attribuer aux couches en allant dans « Couches », une fois une couche
sélectionnée, allez dans « Publication » et vous avez « Style par défaut » :

\noindent{\hspace*{\fill}\sphinxincludegraphics[width=700\sphinxpxdimen]{{styles}.png}\hspace*{\fill}}

\sphinxAtStartPar
 


\subsection{La restriction d’accès aux données}
\label{\detokenize{doc_admin/services:la-restriction-d-acces-aux-donnees}}
\sphinxAtStartPar
La manipulation des droits se fait normalement dans l’onglet {\hyperref[\detokenize{doc_admin/utilisateurs:utilisateur}]{\sphinxcrossref{\DUrole{std,std-ref}{Utilisateur}}}}. La seule chose qui ne peut pas se faire dans cette page
est la restriction d’accès aux données, qui se fait, pour les métadonnées dans l’onglet {\hyperref[\detokenize{doc_admin/catalogue:privileges}]{\sphinxcrossref{\DUrole{std,std-ref}{privilèges}}}}

\sphinxAtStartPar
Par défaut, toutes les données et les ressources dans GeoServer sont accessibles à tous les utilisateurs.
Pour gérer l’accès, des restrictions spécifiques peuvent être appliquées par la suite :

\noindent{\hspace*{\fill}\sphinxincludegraphics[width=700\sphinxpxdimen]{{geos_secu}.png}\hspace*{\fill}}

\sphinxAtStartPar
 

\sphinxAtStartPar
\sphinxstylestrong{Définir l’espace de travail} : Spécifiez l’espace de travail concerné. Dans cet exemple, nous utilisons l’espace de travail « ole »,
qui contient les données intégrées par l’Office de l’eau.

\sphinxAtStartPar
\sphinxstylestrong{Cibler les données} : Indiquez les données que vous souhaitez restreindre. Pour cibler toutes les données, vous pouvez utiliser le symbole « * ».

\sphinxAtStartPar
\sphinxstylestrong{Type d’accès} : Sélectionnez le type d’accès à restreindre. Dans cet exemple, nous choisissons l’accès en lecture.

\sphinxAtStartPar
\sphinxstylestrong{Définir les rôles} : Précisez les rôles qui auront accès à cette sécurité. Ici, nous incluons les rôles « SASPE » et « OREBA ».

\sphinxAtStartPar
Avec cet exemple, seulement les utilisateurs qui possèdent le rôles « SASPE » et/ou « OREBA » peuvent visualiser les flux des données de l’espace de travail « ole »
qui correspond aux données de l’Office de l’eau Réunion.

\sphinxAtStartPar
Ce qui en resulte par cette interface et les règles suivantes :
\begin{itemize}
\item {} 
\sphinxAtStartPar
toutes les données sont lisible par tous les groupes, mais par dessus vient s’ajouter :

\item {} 
\sphinxAtStartPar
les données de l’Office de l’eau Réunion ne sont lisible que par les utilisateurs qui sont dans les groupes OREBA et/ou SASPE

\end{itemize}

\noindent{\hspace*{\fill}\sphinxincludegraphics[width=700\sphinxpxdimen]{{geos_result}.png}\hspace*{\fill}}


\subsection{La restriction d’accès aux services}
\label{\detokenize{doc_admin/services:la-restriction-d-acces-aux-services}}
\sphinxAtStartPar
Pour restreindre les accès aux différents services, par exemple modifier les données directement via mapstore se fait via le service WFS et la fonction « Transaction ».

\noindent{\hspace*{\fill}\sphinxincludegraphics[width=700\sphinxpxdimen]{{services}.png}\hspace*{\fill}}

\sphinxAtStartPar
L’accès à la création de compteur via mapstore se fait via le service wps qui permet de réaliser des traitements géospatiaux directement via des requêtes HTTP.

\sphinxstepscope


\section{Utilsateur \sphinxhyphen{} console admin}
\label{\detokenize{doc_admin/utilisateurs:utilsateur-console-admin}}\label{\detokenize{doc_admin/utilisateurs::doc}}\phantomsection\label{\detokenize{doc_admin/utilisateurs:utilisateur}}
\begin{sphinxShadowBox}
\sphinxstyletopictitle{Table des matières}
\begin{itemize}
\item {} 
\sphinxAtStartPar
\phantomsection\label{\detokenize{doc_admin/utilisateurs:id1}}{\hyperref[\detokenize{doc_admin/utilisateurs:introduction}]{\sphinxcrossref{Introduction}}}

\item {} 
\sphinxAtStartPar
\phantomsection\label{\detokenize{doc_admin/utilisateurs:id2}}{\hyperref[\detokenize{doc_admin/utilisateurs:utilisateurs}]{\sphinxcrossref{Utilisateurs}}}

\item {} 
\sphinxAtStartPar
\phantomsection\label{\detokenize{doc_admin/utilisateurs:id3}}{\hyperref[\detokenize{doc_admin/utilisateurs:organismes}]{\sphinxcrossref{Organismes}}}

\item {} 
\sphinxAtStartPar
\phantomsection\label{\detokenize{doc_admin/utilisateurs:id4}}{\hyperref[\detokenize{doc_admin/utilisateurs:roles}]{\sphinxcrossref{Rôles}}}

\item {} 
\sphinxAtStartPar
\phantomsection\label{\detokenize{doc_admin/utilisateurs:id5}}{\hyperref[\detokenize{doc_admin/utilisateurs:autres}]{\sphinxcrossref{Autres :}}}

\end{itemize}
\end{sphinxShadowBox}


\subsection{Introduction}
\label{\detokenize{doc_admin/utilisateurs:introduction}}
\sphinxAtStartPar
La console d’admin sert à gérer les utilisateurs, les droits, et voir les activités des utilisateurs.
Chaque utilisateur est reliée à une organisation, et les accès sont organisés par des rôles qui sont prédéfinis.
Vous pouvez ajouter, modifier ou supprimer des rôles en fonction des utilisateurs.

\sphinxAtStartPar
La première page est le « Tableau de bord » avec le récapitulatif des actions passées, les utilisateurs en attente de validation,
et permet de voir qui s’est connecté sur la journée.

\noindent{\hspace*{\fill}\sphinxincludegraphics[width=700\sphinxpxdimen]{{user_dashboard}.png}\hspace*{\fill}}

\sphinxAtStartPar
 


\subsection{Utilisateurs}
\label{\detokenize{doc_admin/utilisateurs:utilisateurs}}
\sphinxAtStartPar
Cette section permet de voir la liste des utilisateurs et leurs informations :

\noindent{\hspace*{\fill}\sphinxincludegraphics[width=700\sphinxpxdimen]{{user_user}.png}\hspace*{\fill}}

\sphinxAtStartPar
 

\sphinxAtStartPar
En cliquant sur un utilisateur vous pourrez modifier ses caractéristiques :

\noindent{\hspace*{\fill}\sphinxincludegraphics[width=700\sphinxpxdimen]{{user_user_user}.png}\hspace*{\fill}}

\sphinxAtStartPar
 

\sphinxAtStartPar
C’est aussi dans cet onglet que vous pouvez accepter des nouveaux utilisateurs, changer les noms des utilisateurs et aussi renvoyer la modification de mot de passe
par mail.

\sphinxAtStartPar
Ne jamais supprimer l’utilisateur par défaut : Import DATAFEEDER, ce rôle est nécéssaire au fonctionnement du module datafeeder.


\subsection{Organismes}
\label{\detokenize{doc_admin/utilisateurs:organismes}}
\sphinxAtStartPar
Les utilisateurs sont obligatoirement rattachés à une organisation :

\noindent{\hspace*{\fill}\sphinxincludegraphics[width=700\sphinxpxdimen]{{user_orga}.png}\hspace*{\fill}}

\sphinxAtStartPar
 

\sphinxAtStartPar
Si vous cliquez sur une organisation, vous pouvez modifier ses informations ainsi que ses membres :

\noindent{\hspace*{\fill}\sphinxincludegraphics[width=700\sphinxpxdimen]{{user_orga_orga}.png}\hspace*{\fill}}

\sphinxAtStartPar
 


\subsection{Rôles}
\label{\detokenize{doc_admin/utilisateurs:roles}}
\sphinxAtStartPar
Les rôles permettent de regrouper les utilisateurs et de leur donner des accès et droits :

\noindent{\hspace*{\fill}\sphinxincludegraphics[width=700\sphinxpxdimen]{{user_role}.png}\hspace*{\fill}}

\sphinxAtStartPar
 

\sphinxAtStartPar
Certain rôles définissent des accès particulier et il est possible de créer des groupes en plus pour regrouper des utilisateurs entre eux.

\sphinxAtStartPar
Les rôles principaux sont :
\begin{itemize}
\item {} 
\sphinxAtStartPar
\sphinxstylestrong{SUPERUSER} : accès à la console d’admin

\item {} 
\sphinxAtStartPar
\sphinxstylestrong{ADMINISTRATOR} : permet d’accéder au module admin de GeoServer et permet de créer des styles dans Mapstore avec le rôle MAPSTORE\_ADMIN

\item {} 
\sphinxAtStartPar
\sphinxstylestrong{GN\_ADMIN} : permet d’accéder au GeoNetwork qui est le module admin du catalogue

\item {} 
\sphinxAtStartPar
\sphinxstylestrong{GN\_EDITOR} : permet d’éditer les fiches dans GeoNetwork

\item {} 
\sphinxAtStartPar
\sphinxstylestrong{GN\_REVIEWER} : permet de publier des données à la main dans GeoNetwork

\item {} 
\sphinxAtStartPar
\sphinxstylestrong{MAPSTORE\_ADMIN} : permet d’accéder au module admin de Mapstore, et permet aussi de modifier le style d’un WMS avec le rôle ADMINISTRATOR

\item {} 
\sphinxAtStartPar
\sphinxstylestrong{USER} : permet de se log dans geOrchestra et d’enregister des cartes, dashboards et GeoStories dans Mapstore

\item {} 
\sphinxAtStartPar
\sphinxstylestrong{REFERENT} : permet de modifier les informations de son organisme

\item {} 
\sphinxAtStartPar
\sphinxstylestrong{IMPORT} : donne accès au module d’import de données dans geOrchestra

\end{itemize}

\sphinxAtStartPar
On peut très bien ajouter des rôles, par exemple les rôles OREBA et SASPE, il faut ajouter le rôle OREBA et SASPE au utilisateurs qui appartiennent à ces services.
Puis si l’on veut partager des cartes dans Mapstore et ne les rendre visible ou éditable seulement par un service, il faudra spécifier le groupe en question.

\sphinxAtStartPar
Ou encore créer des groupe pour restraindre l’accès à certaines données avec GeoServer.


\subsection{Autres :}
\label{\detokenize{doc_admin/utilisateurs:autres}}\begin{itemize}
\item {} 
\sphinxAtStartPar
\sphinxstylestrong{Délégation} : sert à donner, à un utilisateur, le droit de promouvoir un autre utilisateur avec des rôles spécifiques

\item {} 
\sphinxAtStartPar
\sphinxstylestrong{Statistique} : permet de voir le nombre de requêtes par jour, et les couches les plus consultées

\item {} 
\sphinxAtStartPar
\sphinxstylestrong{Journaux} : permet d’accèder à l’historique des actions de la console d’admin

\end{itemize}

\sphinxstepscope


\section{Analytics}
\label{\detokenize{doc_admin/analytics:analytics}}\label{\detokenize{doc_admin/analytics::doc}}
\sphinxAtStartPar
Le module analytics permet d’analyser les flux OGC qui sont les données issues des différents services web OGC de GeoServer :

\noindent{\hspace*{\fill}\sphinxincludegraphics[width=700\sphinxpxdimen]{{ana}.png}\hspace*{\fill}}

\sphinxAtStartPar
 

\sphinxAtStartPar
L’interface permet de connaître :
\begin{itemize}
\item {} 
\sphinxAtStartPar
le service web, le titre de la couche et la requête

\item {} 
\sphinxAtStartPar
l’utilisateur et le nombre de requêtes

\item {} 
\sphinxAtStartPar
l’organisation et le nombre de requêtes

\end{itemize}

\sphinxstepscope


\section{Le plugin QGIS}
\label{\detokenize{doc_admin/qgis:le-plugin-qgis}}\label{\detokenize{doc_admin/qgis::doc}}
\sphinxAtStartPar
Le module QGIS est inspiré du plugin open source de GeoBretagne : \sphinxurl{https://github.com/geobretagne/qgis-plugin}

\noindent{\hspace*{\fill}\sphinxincludegraphics[width=600\sphinxpxdimen]{{plugin_base}.png}\hspace*{\fill}}

\sphinxAtStartPar
 

\sphinxAtStartPar
Pour renseigner des données il faut aller dans le dossier « config » puis éditer « config.json ».

\sphinxAtStartPar
Pour configurer l’emplacement du ce fichier, il faut se rendre dans le code « Office\_de\_leau/utils/plugin\_globals.py » et modifier la ligne 50 et 132 de ce fichier.

\sphinxAtStartPar
Ce fichier de configuration permet de définir quelles données seront accessible via le plugin ou non.
Ce plugin ne permet que de manipuler des flux WFS et WMS. Il faudra bien spécifier pour chaque données :
\begin{itemize}
\item {} 
\sphinxAtStartPar
« title » : qui sera le titre affiché dans qgis

\item {} 
\sphinxAtStartPar
« description » : qui sera la description de la donnée dans qgis

\item {} 
\sphinxAtStartPar
« type » : qui sera « wms\_layer » pour un flux WMS et « wfs\_feature\_type » pour un flux WFS

\item {} 
\sphinxAtStartPar
« params » : qui comprendra :

\item {} 
\sphinxAtStartPar
« url » : qui est l’url du flux

\item {} 
\sphinxAtStartPar
« name » : le nom de la donnée dans le geoserver

\item {} 
\sphinxAtStartPar
« outputFormat » : qui sera « image/jpeg » pour un flux WMS et  « outputFormat »: « application/json » pour un flux WFS

\item {} 
\sphinxAtStartPar
« srs » : qui est l’EPSG de la projection de la donnée

\end{itemize}

\sphinxstepscope


\chapter{Documentation d’installation \sphinxhyphen{} Développeur}
\label{\detokenize{doc_instal:documentation-d-installation-developpeur}}\label{\detokenize{doc_instal::doc}}
\sphinxAtStartPar
\sphinxhref{plateformecartographiqueole.pdf}{Documentation au format PDF}.

\sphinxstepscope


\section{Installation}
\label{\detokenize{doc_instal/installation:installation}}\label{\detokenize{doc_instal/installation::doc}}
\begin{sphinxShadowBox}
\sphinxstyletopictitle{Table des matières}
\begin{itemize}
\item {} 
\sphinxAtStartPar
\phantomsection\label{\detokenize{doc_instal/installation:id1}}{\hyperref[\detokenize{doc_instal/installation:introduction}]{\sphinxcrossref{Introduction}}}

\item {} 
\sphinxAtStartPar
\phantomsection\label{\detokenize{doc_instal/installation:id2}}{\hyperref[\detokenize{doc_instal/installation:ansible}]{\sphinxcrossref{Ansible}}}

\item {} 
\sphinxAtStartPar
\phantomsection\label{\detokenize{doc_instal/installation:id3}}{\hyperref[\detokenize{doc_instal/installation:erreurs-frequentes}]{\sphinxcrossref{Erreurs fréquentes}}}

\item {} 
\sphinxAtStartPar
\phantomsection\label{\detokenize{doc_instal/installation:id4}}{\hyperref[\detokenize{doc_instal/installation:serveur-mail}]{\sphinxcrossref{Serveur mail}}}

\item {} 
\sphinxAtStartPar
\phantomsection\label{\detokenize{doc_instal/installation:id5}}{\hyperref[\detokenize{doc_instal/installation:script-de-personnalisation}]{\sphinxcrossref{Script de personnalisation}}}

\item {} 
\sphinxAtStartPar
\phantomsection\label{\detokenize{doc_instal/installation:id6}}{\hyperref[\detokenize{doc_instal/installation:thesaurus}]{\sphinxcrossref{Thesaurus}}}

\item {} 
\sphinxAtStartPar
\phantomsection\label{\detokenize{doc_instal/installation:id7}}{\hyperref[\detokenize{doc_instal/installation:activer-le-module-analytics}]{\sphinxcrossref{Activer le module analytics}}}

\item {} 
\sphinxAtStartPar
\phantomsection\label{\detokenize{doc_instal/installation:id8}}{\hyperref[\detokenize{doc_instal/installation:certificat-ssl}]{\sphinxcrossref{Certificat ssl}}}

\item {} 
\sphinxAtStartPar
\phantomsection\label{\detokenize{doc_instal/installation:id9}}{\hyperref[\detokenize{doc_instal/installation:personnalisation-du-geoserver}]{\sphinxcrossref{Personnalisation du GeoServer}}}

\item {} 
\sphinxAtStartPar
\phantomsection\label{\detokenize{doc_instal/installation:id10}}{\hyperref[\detokenize{doc_instal/installation:relancer-l-infrastructure}]{\sphinxcrossref{Relancer l’infrastructure}}}

\item {} 
\sphinxAtStartPar
\phantomsection\label{\detokenize{doc_instal/installation:id11}}{\hyperref[\detokenize{doc_instal/installation:se-rendre-sur-l-application}]{\sphinxcrossref{Se rendre sur l’application}}}

\end{itemize}
\end{sphinxShadowBox}


\subsection{Introduction}
\label{\detokenize{doc_instal/installation:introduction}}
\sphinxAtStartPar
Georchestra est une IDG qui intègre plusieurs modules et donc plusieurs technologies, il y’a plusieurs façon d’installer cette infrastructure
\begin{itemize}
\item {} 
\sphinxAtStartPar
par docker

\item {} 
\sphinxAtStartPar
par Ansible

\item {} 
\sphinxAtStartPar
à la main

\end{itemize}

\sphinxAtStartPar
Le choix pour l’Office de l’eau Réunion à été Ansible qui permet d’installer des paquets Debians rapidement et automatiquement.

\sphinxAtStartPar
Le lien pour le github et la documentation dans son ensemble de georchestra est le suivant : \sphinxurl{https://github.com/georchestra}


\subsection{Ansible}
\label{\detokenize{doc_instal/installation:ansible}}
\sphinxAtStartPar
Prérequis :
\begin{itemize}
\item {} 
\sphinxAtStartPar
Debian Bookworm (12.x) VM

\item {} 
\sphinxAtStartPar
Mettre à jour les paquets :

\end{itemize}

\begin{sphinxVerbatim}[commandchars=\\\{\}]
apt\PYG{+w}{ }update
\end{sphinxVerbatim}
\begin{itemize}
\item {} 
\sphinxAtStartPar
Ansible : sudo apt install ansible

\end{itemize}

\begin{sphinxVerbatim}[commandchars=\\\{\}]
apt\PYG{+w}{ }install\PYG{+w}{ }ansible
\end{sphinxVerbatim}
\begin{itemize}
\item {} 
\sphinxAtStartPar
Java 17 :

\end{itemize}

\begin{sphinxVerbatim}[commandchars=\\\{\}]
apt\PYG{+w}{ }install\PYG{+w}{ }openjdk\PYGZhy{}17\PYGZhy{}jdk
\end{sphinxVerbatim}
\begin{itemize}
\item {} 
\sphinxAtStartPar
Si votre VM est nouvelle ou si vous avez apache qui tourne sur le port 80, veuillez l’enlever :

\end{itemize}

\begin{sphinxVerbatim}[commandchars=\\\{\}]
apt\PYG{+w}{ }remove\PYG{+w}{ }apache2
\end{sphinxVerbatim}
\begin{itemize}
\item {} 
\sphinxAtStartPar
Clone the source, le code est issue du repo « ansible » de georchestra :

\end{itemize}

\begin{sphinxVerbatim}[commandchars=\\\{\}]
apt\PYG{+w}{ }install\PYG{+w}{ }git
git\PYG{+w}{ }clone\PYG{+w}{ }https://github.com/ToffoluttiVittorio/ansible.git
\end{sphinxVerbatim}
\begin{itemize}
\item {} 
\sphinxAtStartPar
Aller dans le répertoire du repo git, toutes les commandes de cette partie se lance à partir de ce repertoire si non spécifié :

\end{itemize}

\begin{sphinxVerbatim}[commandchars=\\\{\}]
\PYG{n+nb}{cd}\PYG{+w}{ }ansible
\end{sphinxVerbatim}
\begin{itemize}
\item {} 
\sphinxAtStartPar
Changer le fqdn dans le fichier \sphinxcode{\sphinxupquote{playbooks/georchestra}} ligne 88 avec la variable \sphinxcode{\sphinxupquote{fqdn}} et modifier \sphinxcode{\sphinxupquote{georchestra.ole.re}} :

\end{itemize}

\begin{sphinxVerbatim}[commandchars=\\\{\}]
nano\PYG{+w}{ }playbooks/georchestra.yml
\end{sphinxVerbatim}

\sphinxAtStartPar
dans la ligne :

\begin{sphinxVerbatim}[commandchars=\\\{\}]
fqdn:\PYG{+w}{ }georchestra.ole.re
\end{sphinxVerbatim}
\begin{itemize}
\item {} 
\sphinxAtStartPar
et dans le fichier de personnalisation \sphinxcode{\sphinxupquote{Configuration/last.sh}}, remplacer \sphinxcode{\sphinxupquote{georchestra.ole.re}} par votre fqdn :

\end{itemize}

\begin{sphinxVerbatim}[commandchars=\\\{\}]
nano\PYG{+w}{ }Configuration/last.sh
\end{sphinxVerbatim}

\sphinxAtStartPar
dans la ligne :

\begin{sphinxVerbatim}[commandchars=\\\{\}]
\PYG{n+nb}{echo}\PYG{+w}{ }\PYG{l+s+s1}{\PYGZsq{}127.0.0.1 georchestra.ole.re\PYGZsq{}}\PYG{+w}{ }\PYG{p}{|}\PYG{+w}{ }sudo\PYG{+w}{ }tee\PYG{+w}{ }\PYGZhy{}a\PYG{+w}{ }/etc/hosts\PYG{+w}{ }\PYGZgt{}\PYG{+w}{ }/dev/null
\end{sphinxVerbatim}
\begin{itemize}
\item {} 
\sphinxAtStartPar
Installer les rôles de GeoNetwork :

\end{itemize}

\begin{sphinxVerbatim}[commandchars=\\\{\}]
ansible\PYGZhy{}galaxy\PYG{+w}{ }install\PYG{+w}{ }\PYGZhy{}r\PYG{+w}{ }requirements.yaml
chmod\PYG{+w}{ }\PYGZhy{}777\PYG{+w}{ }roles/
\end{sphinxVerbatim}
\begin{itemize}
\item {} 
\sphinxAtStartPar
Il faut run le playbooks qui est l’installation de tous les modules :

\end{itemize}

\begin{sphinxVerbatim}[commandchars=\\\{\}]
ansible\PYGZhy{}playbook\PYG{+w}{ }playbooks/georchestra.yml
\end{sphinxVerbatim}

\begin{sphinxadmonition}{note}{Note:}
\sphinxAtStartPar
Des erreurs peuvent apparaître lors de cette étape, veuillez consulter le chapitre juste en dessous « Erreurs fréquentes » si cela arrive.
\end{sphinxadmonition}

\sphinxAtStartPar
L’installation de l’infrastructure de geOrchestra est faite, il reste à installer un serveur de mail et les scripts de personnalisation pour avoir
l’application fonctionnel et complète pour l’Office de l’eau Réunion.


\subsection{Erreurs fréquentes}
\label{\detokenize{doc_instal/installation:erreurs-frequentes}}
\sphinxAtStartPar
Une erreur lors de la première installation mais n’est asbsolument pas blocante :

\begin{sphinxVerbatim}[commandchars=\\\{\}]
TASK\PYG{+w}{ }\PYG{o}{[}openldap\PYG{+w}{ }:\PYG{+w}{ }check\PYG{+w}{ }\PYG{k}{if}\PYG{+w}{ }the\PYG{+w}{ }root\PYG{+w}{ }already\PYG{+w}{ }exists\PYG{o}{]}\PYG{+w}{ }******************************************************************
fatal:\PYG{+w}{ }\PYG{o}{[}localhost\PYG{o}{]}:\PYG{+w}{ }FAILED!\PYG{+w}{ }\PYG{o}{=}\PYGZgt{}\PYG{+w}{ }\PYG{o}{\PYGZob{}}\PYG{l+s+s2}{\PYGZdq{}changed\PYGZdq{}}:\PYG{+w}{ }true,\PYG{+w}{ }\PYG{l+s+s2}{\PYGZdq{}cmd\PYGZdq{}}:\PYG{+w}{ }\PYG{o}{[}\PYG{l+s+s2}{\PYGZdq{}ldapsearch\PYGZdq{}},\PYG{+w}{ }\PYG{l+s+s2}{\PYGZdq{}\PYGZhy{}x\PYGZdq{}},\PYG{+w}{ }\PYG{l+s+s2}{\PYGZdq{}\PYGZhy{}b\PYGZdq{}},\PYG{+w}{ }\PYG{l+s+s2}{\PYGZdq{}dc=georchestra,dc=org\PYGZdq{}},\PYG{+w}{ }\PYG{l+s+s2}{\PYGZdq{}dc=georchestra\PYGZdq{}}\PYG{o}{]},\PYG{+w}{ }\PYG{l+s+s2}{\PYGZdq{}delta\PYGZdq{}}:\PYG{+w}{ }\PYG{l+s+s2}{\PYGZdq{}0:00:00.009190\PYGZdq{}},\PYG{+w}{ }\PYG{l+s+s2}{\PYGZdq{}end\PYGZdq{}}:\PYG{+w}{ }\PYG{l+s+s2}{\PYGZdq{}2024\PYGZhy{}10\PYGZhy{}09 09:19:33.368546\PYGZdq{}},\PYG{+w}{ }\PYG{l+s+s2}{\PYGZdq{}msg\PYGZdq{}}:\PYG{+w}{ }\PYG{l+s+s2}{\PYGZdq{}non\PYGZhy{}zero return code\PYGZdq{}},\PYG{+w}{ }\PYG{l+s+s2}{\PYGZdq{}rc\PYGZdq{}}:\PYG{+w}{ }\PYG{l+m}{32},\PYG{+w}{ }\PYG{l+s+s2}{\PYGZdq{}start\PYGZdq{}}:\PYG{+w}{ }\PYG{l+s+s2}{\PYGZdq{}2024\PYGZhy{}10\PYGZhy{}09 09:19:33.359356\PYGZdq{}},\PYG{+w}{ }\PYG{l+s+s2}{\PYGZdq{}stderr\PYGZdq{}}:\PYG{+w}{ }\PYG{l+s+s2}{\PYGZdq{}\PYGZdq{}},\PYG{+w}{ }\PYG{l+s+s2}{\PYGZdq{}stderr\PYGZus{}lines\PYGZdq{}}:\PYG{+w}{ }\PYG{o}{[}\PYG{o}{]},\PYG{+w}{ }\PYG{l+s+s2}{\PYGZdq{}stdout\PYGZdq{}}:\PYG{+w}{ }\PYG{l+s+s2}{\PYGZdq{}\PYGZsh{} extended LDIF\PYGZbs{}n\PYGZsh{}\PYGZbs{}n\PYGZsh{} LDAPv3\PYGZbs{}n\PYGZsh{} base \PYGZlt{}dc=georchestra,dc=org\PYGZgt{} with scope subtree\PYGZbs{}n\PYGZsh{} filter: dc=georchestra\PYGZbs{}n\PYGZsh{} requesting: ALL\PYGZbs{}n\PYGZsh{}\PYGZbs{}n\PYGZbs{}n\PYGZsh{} search result\PYGZbs{}nsearch: 2\PYGZbs{}nresult: 32 No such object\PYGZbs{}n\PYGZbs{}n\PYGZsh{} numResponses: 1\PYGZdq{}},\PYG{+w}{ }\PYG{l+s+s2}{\PYGZdq{}stdout\PYGZus{}lines\PYGZdq{}}:\PYG{+w}{ }\PYG{o}{[}\PYG{l+s+s2}{\PYGZdq{}\PYGZsh{} extended LDIF\PYGZdq{}},\PYG{+w}{ }\PYG{l+s+s2}{\PYGZdq{}\PYGZsh{}\PYGZdq{}},\PYG{+w}{ }\PYG{l+s+s2}{\PYGZdq{}\PYGZsh{} LDAPv3\PYGZdq{}},\PYG{+w}{ }\PYG{l+s+s2}{\PYGZdq{}\PYGZsh{} base \PYGZlt{}dc=georchestra,dc=org\PYGZgt{} with scope subtree\PYGZdq{}},\PYG{+w}{ }\PYG{l+s+s2}{\PYGZdq{}\PYGZsh{} filter: dc=georchestra\PYGZdq{}},\PYG{+w}{ }\PYG{l+s+s2}{\PYGZdq{}\PYGZsh{} requesting: ALL\PYGZdq{}},\PYG{+w}{ }\PYG{l+s+s2}{\PYGZdq{}\PYGZsh{}\PYGZdq{}},\PYG{+w}{ }\PYG{l+s+s2}{\PYGZdq{}\PYGZdq{}},\PYG{+w}{ }\PYG{l+s+s2}{\PYGZdq{}\PYGZsh{} search result\PYGZdq{}},\PYG{+w}{ }\PYG{l+s+s2}{\PYGZdq{}search: 2\PYGZdq{}},\PYG{+w}{ }\PYG{l+s+s2}{\PYGZdq{}result: 32 No such object\PYGZdq{}},\PYG{+w}{ }\PYG{l+s+s2}{\PYGZdq{}\PYGZdq{}},\PYG{+w}{ }\PYG{l+s+s2}{\PYGZdq{}\PYGZsh{} numResponses: 1\PYGZdq{}}\PYG{o}{]}\PYG{o}{\PYGZcb{}}
...ignoring
\end{sphinxVerbatim}

\sphinxAtStartPar
Si vous avez des erreurs sur \sphinxcode{\sphinxupquote{sviewer}} ou \sphinxcode{\sphinxupquote{htodcs}} de ce type :

\begin{sphinxVerbatim}[commandchars=\\\{\}]
TASK\PYG{+w}{ }\PYG{o}{[}georchestra\PYG{+w}{ }:\PYG{+w}{ }checkout\PYG{+w}{ }sviewer\PYG{o}{]}\PYG{+w}{ }*******************************************************************************************************************************************************************************************************
fatal:\PYG{+w}{ }\PYG{o}{[}localhost\PYG{o}{]}:\PYG{+w}{ }FAILED!\PYG{+w}{ }\PYG{o}{=}\PYGZgt{}\PYG{+w}{ }\PYG{o}{\PYGZob{}}\PYG{l+s+s2}{\PYGZdq{}changed\PYGZdq{}}:\PYG{+w}{ }false,\PYG{+w}{ }\PYG{l+s+s2}{\PYGZdq{}msg\PYGZdq{}}:\PYG{+w}{ }\PYG{l+s+s2}{\PYGZdq{}Unable to parse submodule hash line: Entrée dans \PYGZsq{}lib/ol3\PYGZsq{}\PYGZdq{}}\PYG{o}{\PYGZcb{}}
\end{sphinxVerbatim}

\sphinxAtStartPar
Il faut supprimer le repertoire htdocs, et relancer le run du playbook :

\begin{sphinxVerbatim}[commandchars=\\\{\}]
rm\PYG{+w}{ }\PYGZhy{}r\PYG{+w}{ }/var/www/georchestra/htdocs
\end{sphinxVerbatim}

\begin{sphinxadmonition}{note}{Note:}
\sphinxAtStartPar
Si vous avez encore une erreur lors de l’installation après avoir supprimer le repertoire htdocs, il faut souvent relancer encore le playbook sans rien toucher
\end{sphinxadmonition}

\sphinxAtStartPar
Si vous avez des erreurs de versions de paquets, il faut mettre les bonnes versions, conforme au fichier \sphinxcode{\sphinxupquote{ansible/playbooks/georchestra.yml}}.


\subsection{Serveur mail}
\label{\detokenize{doc_instal/installation:serveur-mail}}
\sphinxAtStartPar
Pour le serveur mail, un postfix est installé sur la vm et est réliée à carbonio, copier la configuration faite dans la vm dev\sphinxhyphen{}carto.ole.re,
le mail de l’administrateur se définit dans \sphinxcode{\sphinxupquote{ansible/playbooks/georchestra.yml}} et les templates des mails sont dans \sphinxcode{\sphinxupquote{/etc/georchestra/datafeeder}}
et \sphinxcode{\sphinxupquote{/etc/georchestra/}}


\subsection{Script de personnalisation}
\label{\detokenize{doc_instal/installation:script-de-personnalisation}}
\sphinxAtStartPar
Les scripts de personnalisation servent à ajouter les spécifications pour l’Office de l’eau Réunion sans directement changer le code d’installation.

\sphinxAtStartPar
Il y’a trois script bash qui modifient les logos, couleurs, référentiel de coordonée … dans le dossier « Configuration », les lancer depuis \sphinxcode{\sphinxupquote{ansible/Configuration}}
voici la commande pour les rendre executable et les lancer :

\begin{sphinxVerbatim}[commandchars=\\\{\}]
\PYG{n+nb}{cd}\PYG{+w}{ }Configuration
\end{sphinxVerbatim}

\begin{sphinxVerbatim}[commandchars=\\\{\}]
chmod\PYG{+w}{ }\PYG{l+m}{777}\PYG{+w}{ }script\PYGZus{}remplacement.sh
chmod\PYG{+w}{ }\PYG{l+m}{777}\PYG{+w}{ }other.sh
chmod\PYG{+w}{ }\PYG{l+m}{777}\PYG{+w}{ }last.sh
\end{sphinxVerbatim}

\begin{sphinxVerbatim}[commandchars=\\\{\}]
./script\PYGZus{}remplacement.sh
./other.sh
./last.sh
\end{sphinxVerbatim}


\subsection{Thesaurus}
\label{\detokenize{doc_instal/installation:thesaurus}}
\sphinxAtStartPar
Le thesaurus est le catalogue de mots clés utilisé lors de l’intégration de données par les agents.
Par defaut, georchestra utilise le catalogue INSPIRE, vous pouvez le modifier en allant sur {\hyperref[\detokenize{doc_admin/catalogue:id1}]{\sphinxcrossref{\DUrole{std,std-ref}{thesaurus}}}}.


\subsection{Activer le module analytics}
\label{\detokenize{doc_instal/installation:activer-le-module-analytics}}
\sphinxAtStartPar
Pour activer le module analytics, il faut changer les droits du schéma « ogcstatistics » de postgres à georchestra.
La base de donnée est accessible avec psql :

\begin{sphinxVerbatim}[commandchars=\\\{\}]
psql\PYG{+w}{ }\PYGZhy{}U\PYG{+w}{ }postgres\PYG{+w}{ }\PYGZhy{}h\PYG{+w}{ }localhost\PYG{+w}{ }\PYGZhy{}d\PYG{+w}{ }georchestra
\end{sphinxVerbatim}

\sphinxAtStartPar
Puis il faut lancer :

\begin{sphinxVerbatim}[commandchars=\\\{\}]
ALTER\PYG{+w}{ }SCHEMA\PYG{+w}{ }ogcstatistics\PYG{+w}{ }OWNER\PYG{+w}{ }TO\PYG{+w}{ }georchestra\PYG{p}{;}
\end{sphinxVerbatim}


\subsection{Certificat ssl}
\label{\detokenize{doc_instal/installation:certificat-ssl}}
\sphinxAtStartPar
Pour autoriser le geoserver et mapstore à communiquer entre eux pour la fonction print de mapstore, il est nécéssaire d’ajouter le certificat ssl à java,
cette documentation fonctionne parfaitement : \sphinxurl{https://stackoverflow.com/questions/14947517/pkix-path-building-failed-sun-security-provider-certpath-suncertpathbuilderexce}.

\sphinxAtStartPar
IL faut copier la valeur du certificat qui apparaît dans votre navigateur et l’enregistrer avec « .der », puis localiser votre \$JAVA\_HOME, et dans lib puis security se trouve
un fichier « cacerts », il faudra lancer :

\begin{sphinxVerbatim}[commandchars=\\\{\}]
sudo\PYG{+w}{ }keytool\PYG{+w}{ }\PYGZhy{}import\PYG{+w}{ }\PYGZhy{}alias\PYG{+w}{ }mysitedev\PYG{+w}{ }\PYGZhy{}keystore\PYG{+w}{  }\PYG{n+nv}{\PYGZdl{}JAVA\PYGZus{}HOME}/jre/lib/security/cacerts\PYG{+w}{ }\PYGZhy{}file\PYG{+w}{ }dev.der
\end{sphinxVerbatim}

\sphinxAtStartPar
où « mysitedev » est votre fqdn et dev.der le certificat, le mot de passer par défaut est : « changeit ».


\subsection{Personnalisation du GeoServer}
\label{\detokenize{doc_instal/installation:personnalisation-du-geoserver}}
\sphinxAtStartPar
Il faut changer à la main certaines configuration du GeoServer :
\begin{itemize}
\item {} 
\sphinxAtStartPar
modifier l’url du proxy en y rajoutant votre fqdn et décocher « Utiliser les entêtes pour l’url proxy » en allant dans la page « Services » puis dans « Global » :

\end{itemize}
\begin{quote}

\noindent{\hspace*{\fill}\sphinxincludegraphics[width=700\sphinxpxdimen]{{geoserver_global}.png}\hspace*{\fill}}
\end{quote}

\sphinxAtStartPar
 
\begin{itemize}
\item {} \begin{description}
\sphinxlineitem{modifier les services pour modifier les données depuis mapstore et faire des graphiques, enlever les 2 règels wfs.Transaction et wps.* en allant dans « Sécurité des services »}
\sphinxAtStartPar
si vous le souhaitez :

\end{description}

\end{itemize}
\begin{quote}

\noindent{\hspace*{\fill}\sphinxincludegraphics[width=700\sphinxpxdimen]{{geoserver_services}.png}\hspace*{\fill}}
\end{quote}

\sphinxAtStartPar
 

\sphinxAtStartPar
Une fois l’installation terminé, il faudra relancer le datafeeder et le reste de l’infrastructure:

\begin{sphinxVerbatim}[commandchars=\\\{\}]
systemctl\PYG{+w}{ }restart\PYG{+w}{ }datafeeder.service
\end{sphinxVerbatim}


\subsection{Relancer l’infrastructure}
\label{\detokenize{doc_instal/installation:relancer-l-infrastructure}}
\sphinxAtStartPar
Pour relancer l’infrastructure, il faut relancer les 3 tomcats et potentiellement nginx :

\begin{sphinxVerbatim}[commandchars=\\\{\}]
sudo\PYG{+w}{ }systemctl\PYG{+w}{ }restart\PYG{+w}{ }tomcat@georchestra.service
\end{sphinxVerbatim}

\begin{sphinxVerbatim}[commandchars=\\\{\}]
sudo\PYG{+w}{ }systemctl\PYG{+w}{ }restart\PYG{+w}{ }tomcat@geoserver.service
\end{sphinxVerbatim}

\begin{sphinxVerbatim}[commandchars=\\\{\}]
sudo\PYG{+w}{ }systemctl\PYG{+w}{ }restart\PYG{+w}{ }tomcat@proxycas.service
\end{sphinxVerbatim}

\begin{sphinxVerbatim}[commandchars=\\\{\}]
sudo\PYG{+w}{ }systemctl\PYG{+w}{ }restart\PYG{+w}{ }nginx
\end{sphinxVerbatim}


\subsection{Se rendre sur l’application}
\label{\detokenize{doc_instal/installation:se-rendre-sur-l-application}}
\sphinxAtStartPar
Pour se rendre sur l’application, aller à l’addresse :

\begin{sphinxVerbatim}[commandchars=\\\{\}]
https://le\PYGZus{}fqdn\PYGZus{}renseigné/
\end{sphinxVerbatim}

\sphinxstepscope


\section{Configuration}
\label{\detokenize{doc_instal/configuration:configuration}}\label{\detokenize{doc_instal/configuration::doc}}
\begin{sphinxShadowBox}
\sphinxstyletopictitle{Table des matières}
\begin{itemize}
\item {} 
\sphinxAtStartPar
\phantomsection\label{\detokenize{doc_instal/configuration:id1}}{\hyperref[\detokenize{doc_instal/configuration:introduction}]{\sphinxcrossref{Introduction}}}

\item {} 
\sphinxAtStartPar
\phantomsection\label{\detokenize{doc_instal/configuration:id2}}{\hyperref[\detokenize{doc_instal/configuration:localisation-des-differents-repertoires}]{\sphinxcrossref{Localisation des différents répertoires}}}

\item {} 
\sphinxAtStartPar
\phantomsection\label{\detokenize{doc_instal/configuration:id3}}{\hyperref[\detokenize{doc_instal/configuration:fichiers-de-configuration-du-datadir}]{\sphinxcrossref{Fichiers de configuration du datadir}}}

\item {} 
\sphinxAtStartPar
\phantomsection\label{\detokenize{doc_instal/configuration:id4}}{\hyperref[\detokenize{doc_instal/configuration:versionnement-des-modules}]{\sphinxcrossref{Versionnement des modules}}}

\item {} 
\sphinxAtStartPar
\phantomsection\label{\detokenize{doc_instal/configuration:id5}}{\hyperref[\detokenize{doc_instal/configuration:base-de-donnee}]{\sphinxcrossref{Base de donnée}}}

\item {} 
\sphinxAtStartPar
\phantomsection\label{\detokenize{doc_instal/configuration:id6}}{\hyperref[\detokenize{doc_instal/configuration:scripts-de-personnalisation}]{\sphinxcrossref{Scripts de personnalisation}}}

\end{itemize}
\end{sphinxShadowBox}


\subsection{Introduction}
\label{\detokenize{doc_instal/configuration:introduction}}
\sphinxAtStartPar
Le code étant très dense et compilé, il faudra comprendre la structure et les fichiers de configuration mis à dispostion plutôt que le code en profondeur.
La première définition des version, modules, port … se fait dans \sphinxcode{\sphinxupquote{ansible/playbooks/georchestra.yml}} et la modification des modules installés se fait dans
\sphinxcode{\sphinxupquote{/etc/georchestra/}}.


\subsection{Localisation des différents répertoires}
\label{\detokenize{doc_instal/configuration:localisation-des-differents-repertoires}}
\sphinxAtStartPar
Les dossiers de configuration se trouve dans : \sphinxcode{\sphinxupquote{/etc/georchestra/}}, ils sont documentés, facile à parcourir et modifier.

\sphinxAtStartPar
Le document d’installation se trouve dans \sphinxcode{\sphinxupquote{ansible/playbooks/georchestra.yml}}, c’est à partir de ce fichier que se fait l’installation
et spécifie les versions, les variables …

\sphinxAtStartPar
Les logs des différents modules sont dans : \sphinxcode{\sphinxupquote{/srv/log/}}

\sphinxAtStartPar
Les binaires et le code source sont divisée en trois :
\begin{itemize}
\item {} 
\sphinxAtStartPar
\sphinxcode{\sphinxupquote{/srv/tomcat/georchestra/webapps}} pour les modules analytics, console, geonetwork, geowebcache, header, import, mapstore

\item {} 
\sphinxAtStartPar
\sphinxcode{\sphinxupquote{/srv/tomcat/geoserver/webapps}} pour le module geoserver

\item {} 
\sphinxAtStartPar
\sphinxcode{\sphinxupquote{/srv/tomcat/proxycas/webapps}} pour les modules cas et ROOT

\end{itemize}

\sphinxAtStartPar
Les données de geonetwork et geoserver sont dans le repertoire : \sphinxcode{\sphinxupquote{/srv/data/}}

\sphinxAtStartPar
Les pages web statiques sont dans : \sphinxcode{\sphinxupquote{/var/www/georchestra/htdocs/}}

\sphinxAtStartPar
Le module nginx est lui dans : \sphinxcode{\sphinxupquote{/etc/nginx/}}


\subsection{Fichiers de configuration du datadir}
\label{\detokenize{doc_instal/configuration:fichiers-de-configuration-du-datadir}}
\sphinxAtStartPar
GeOrchestra possèdent un « datadir » qui est un repertoire de fichiers de configuration qui sert à modifier rapidement certaines configurations.
Il se situe dans : \sphinxcode{\sphinxupquote{/etc/georchestra}}
Il faut ensuite naviguer dans les différents répertoies pour modifier la configuration de chaque module.

\sphinxAtStartPar
Les paramètres généraux peuvent être modfiées dans le fichier \sphinxcode{\sphinxupquote{/etc/georchestra/default.properties}} où il est possible de modifier :
\begin{itemize}
\item {} 
\sphinxAtStartPar
le logo

\item {} 
\sphinxAtStartPar
le style du header

\item {} 
\sphinxAtStartPar
les paramètre de postgresql

\item {} 
\sphinxAtStartPar
les paramètre du ldap

\item {} 
\sphinxAtStartPar
les paramètres du rabittmq

\item {} 
\sphinxAtStartPar
les paramètres SMTP

\end{itemize}

\sphinxAtStartPar
Ensuite il faut naviguer dans les différents sous\sphinxhyphen{}répertoire pour modifier spécifiquement les configs, voici le lien
de la documentation qui explique cela plus en détails : \sphinxurl{https://github.com/georchestra/datadir}


\subsection{Versionnement des modules}
\label{\detokenize{doc_instal/configuration:versionnement-des-modules}}
\sphinxAtStartPar
Le versionnement s’effectue dans le fichier issue du clone du repo git \sphinxcode{\sphinxupquote{ansible/playbooks/georchestra.yml}} qui est le fichier qui va spécifier les versions et les modules à installer
lors du lancement de l’installation.

\sphinxAtStartPar
Ce fichier sert à configurer : les versions, les chemins, les ports, les modules …

\sphinxAtStartPar
Il est très simple à lire et comprendre et se trouve dans \sphinxcode{\sphinxupquote{ansible/playbooks/georchestra.yml}}.

\sphinxAtStartPar
Si vous voulez que les modifications dans ce fichiers s’execute il faut relancer l’installation dans le repo du git cloné:

\begin{sphinxVerbatim}[commandchars=\\\{\}]
sudo\PYG{+w}{ }ansible\PYGZhy{}playbook\PYG{+w}{ }playbooks/georchestra.yml
\end{sphinxVerbatim}


\subsection{Base de donnée}
\label{\detokenize{doc_instal/configuration:base-de-donnee}}
\sphinxAtStartPar
La base de donnée est accessible avec psql :

\begin{sphinxVerbatim}[commandchars=\\\{\}]
psql\PYG{+w}{ }\PYGZhy{}U\PYG{+w}{ }georchestra\PYG{+w}{ }\PYGZhy{}h\PYG{+w}{ }localhost
\end{sphinxVerbatim}

\sphinxAtStartPar
Elle stocke les données dans différents schémas.


\subsection{Scripts de personnalisation}
\label{\detokenize{doc_instal/configuration:scripts-de-personnalisation}}
\sphinxAtStartPar
Les scripts sont écris en shell, facile à comprendre et facilement modifiable, ils sont au nombre de 3 :
\begin{itemize}
\item {} 
\sphinxAtStartPar
script\_remplacement.sh

\item {} 
\sphinxAtStartPar
other.sh

\item {} 
\sphinxAtStartPar
last.sh

\end{itemize}

\sphinxAtStartPar
et sont dans \sphinxcode{\sphinxupquote{ansible/Configuration}}.

\sphinxstepscope


\section{Mise à jour}
\label{\detokenize{doc_instal/maj:mise-a-jour}}\label{\detokenize{doc_instal/maj::doc}}
\begin{sphinxShadowBox}
\sphinxstyletopictitle{Table des matières}
\begin{itemize}
\item {} 
\sphinxAtStartPar
\phantomsection\label{\detokenize{doc_instal/maj:id1}}{\hyperref[\detokenize{doc_instal/maj:introduction}]{\sphinxcrossref{Introduction}}}

\item {} 
\sphinxAtStartPar
\phantomsection\label{\detokenize{doc_instal/maj:id2}}{\hyperref[\detokenize{doc_instal/maj:paquets-debians}]{\sphinxcrossref{Paquets debians}}}

\end{itemize}
\end{sphinxShadowBox}


\subsection{Introduction}
\label{\detokenize{doc_instal/maj:introduction}}
\sphinxAtStartPar
La version actuelle de geOrchestra est la version 24, les versions sont supporté pendant 1 an avec des patchs mineurs qui ne demande pas de
configuration supplémentaire et peuvent être installées avec les paquets debians directement.

\sphinxAtStartPar
Pour ce qui est de l’installation de versions majeurs, elle se font en modifiant le fichier \sphinxcode{\sphinxupquote{georchestra.yml}},
il faudra relancer toute l’installation et potentiellement faire des ajustements.

\sphinxAtStartPar
Ne pas lancer la mise à jour de tous les paquets de georchestra d’un coup, des versions peuvent ne pas être compatible entre elles,
veuillez vous référer aux différentes releases et leurs compatibilités : \sphinxurl{https://github.com/georchestra/georchestra/releases}


\subsection{Paquets debians}
\label{\detokenize{doc_instal/maj:paquets-debians}}
\sphinxAtStartPar
Voici la liste des paquets debians installé par georchestra :

\noindent{\hspace*{\fill}\sphinxincludegraphics[width=700\sphinxpxdimen]{{debian_paquet}.png}\hspace*{\fill}}

\sphinxAtStartPar
 

\sphinxAtStartPar
Lancer la mise à jour des paquets debians si des patchs mineurs ont été apportés.

\sphinxAtStartPar
D’autres paquets sont aussi installé sur la machine :
\begin{itemize}
\item {} 
\sphinxAtStartPar
tomcat9

\item {} 
\sphinxAtStartPar
nginx

\item {} 
\sphinxAtStartPar
postgresql

\item {} 
\sphinxAtStartPar
elasticsearch

\item {} 
\sphinxAtStartPar
kibana

\end{itemize}



\renewcommand{\indexname}{Index}
\printindex
\end{document}