%% Generated by Sphinx.
\def\sphinxdocclass{report}
\documentclass[letterpaper,10pt,french]{sphinxmanual}
\ifdefined\pdfpxdimen
   \let\sphinxpxdimen\pdfpxdimen\else\newdimen\sphinxpxdimen
\fi \sphinxpxdimen=.75bp\relax
\ifdefined\pdfimageresolution
    \pdfimageresolution= \numexpr \dimexpr1in\relax/\sphinxpxdimen\relax
\fi
%% let collapsible pdf bookmarks panel have high depth per default
\PassOptionsToPackage{bookmarksdepth=5}{hyperref}

\PassOptionsToPackage{booktabs}{sphinx}
\PassOptionsToPackage{colorrows}{sphinx}

\PassOptionsToPackage{warn}{textcomp}
\usepackage[utf8]{inputenc}
\ifdefined\DeclareUnicodeCharacter
% support both utf8 and utf8x syntaxes
  \ifdefined\DeclareUnicodeCharacterAsOptional
    \def\sphinxDUC#1{\DeclareUnicodeCharacter{"#1}}
  \else
    \let\sphinxDUC\DeclareUnicodeCharacter
  \fi
  \sphinxDUC{00A0}{\nobreakspace}
  \sphinxDUC{2500}{\sphinxunichar{2500}}
  \sphinxDUC{2502}{\sphinxunichar{2502}}
  \sphinxDUC{2514}{\sphinxunichar{2514}}
  \sphinxDUC{251C}{\sphinxunichar{251C}}
  \sphinxDUC{2572}{\textbackslash}
\fi
\usepackage{cmap}
\usepackage[T1]{fontenc}
\usepackage{amsmath,amssymb,amstext}
\usepackage{babel}



\usepackage{tgtermes}
\usepackage{tgheros}
\renewcommand{\ttdefault}{txtt}



\usepackage[Sonny]{fncychap}
\ChNameVar{\Large\normalfont\sffamily}
\ChTitleVar{\Large\normalfont\sffamily}
\usepackage{sphinx}

\fvset{fontsize=auto}
\usepackage{geometry}


% Include hyperref last.
\usepackage{hyperref}
% Fix anchor placement for figures with captions.
\usepackage{hypcap}% it must be loaded after hyperref.
% Set up styles of URL: it should be placed after hyperref.
\urlstyle{same}

\addto\captionsfrench{\renewcommand{\contentsname}{Documentations:}}

\usepackage{sphinxmessages}
\setcounter{tocdepth}{1}



\title{Plateforme cartographique OLE}
\date{sept. 30, 2024}
\release{}
\author{Vittorio Toffolutti}
\newcommand{\sphinxlogo}{\vbox{}}
\renewcommand{\releasename}{}
\makeindex
\begin{document}

\ifdefined\shorthandoff
  \ifnum\catcode`\=\string=\active\shorthandoff{=}\fi
  \ifnum\catcode`\"=\active\shorthandoff{"}\fi
\fi

\pagestyle{empty}
\sphinxmaketitle
\pagestyle{plain}
\sphinxtableofcontents
\pagestyle{normal}
\phantomsection\label{\detokenize{index::doc}}


\sphinxAtStartPar
\sphinxhref{\_static/guide\_utilisateur.pdf}{Documentation au format PDF}.

\sphinxstepscope


\chapter{Documentation utilisateur}
\label{\detokenize{doc_user:documentation-utilisateur}}\label{\detokenize{doc_user::doc}}
\sphinxAtStartPar
\sphinxhref{\_static/guide\_utilisateur.pdf}{Documentation au format PDF}.

\sphinxstepscope


\section{Catalogue}
\label{\detokenize{doc_user/catalogue:catalogue}}\label{\detokenize{doc_user/catalogue::doc}}
\begin{sphinxShadowBox}
\sphinxstyletopictitle{Table des matières}
\begin{itemize}
\item {} 
\sphinxAtStartPar
\phantomsection\label{\detokenize{doc_user/catalogue:id1}}{\hyperref[\detokenize{doc_user/catalogue:introduction}]{\sphinxcrossref{Introduction}}}

\item {} 
\sphinxAtStartPar
\phantomsection\label{\detokenize{doc_user/catalogue:id2}}{\hyperref[\detokenize{doc_user/catalogue:la-recherche-des-donnees-dans-le-catalogue}]{\sphinxcrossref{La recherche des données dans le catalogue}}}

\item {} 
\sphinxAtStartPar
\phantomsection\label{\detokenize{doc_user/catalogue:id3}}{\hyperref[\detokenize{doc_user/catalogue:les-fonctionnalites-des-fiches-de-donnees}]{\sphinxcrossref{Les fonctionnalités des fiches de données}}}

\end{itemize}
\end{sphinxShadowBox}


\subsection{Introduction}
\label{\detokenize{doc_user/catalogue:introduction}}
\sphinxAtStartPar
Les données de ce catalogue proviennent de différents catalogue nationaux mais peuvent aussi être propre à l’Office de l’eau Réunion.
Chaque donnée est reliée à une organisation et possèdent une description détaillée.


\subsection{La recherche des données dans le catalogue}
\label{\detokenize{doc_user/catalogue:la-recherche-des-donnees-dans-le-catalogue}}
\sphinxAtStartPar
Il y a 3 onglets dans ce catalogue :
\begin{itemize}
\item {} 
\sphinxAtStartPar
\sphinxstylestrong{ACCUEIL} pour afficher les dernières données postées.

\end{itemize}

\noindent{\hspace*{\fill}\sphinxincludegraphics[width=600\sphinxpxdimen]{{catalogue_accueil}.png}\hspace*{\fill}}
\begin{itemize}
\item {} 
\sphinxAtStartPar
\sphinxstylestrong{DONNEES} pour afficher toutes les données.

\end{itemize}

\noindent{\hspace*{\fill}\sphinxincludegraphics[width=600\sphinxpxdimen]{{catalogue_donnees}.png}\hspace*{\fill}}
\begin{itemize}
\item {} 
\sphinxAtStartPar
\sphinxstylestrong{ORGANISATION} pour afficher toutes les organisations qui possèdent des données.

\end{itemize}

\noindent{\hspace*{\fill}\sphinxincludegraphics[width=600\sphinxpxdimen]{{catalogue_orga}.png}\hspace*{\fill}}

\sphinxAtStartPar
Vous pouvez filtrer vos recherches en fonction :
\begin{itemize}
\item {} 
\sphinxAtStartPar
de la date de publication

\item {} 
\sphinxAtStartPar
du type de données

\item {} 
\sphinxAtStartPar
du format

\item {} 
\sphinxAtStartPar
de l’organisation qui l’a publiée

\item {} 
\sphinxAtStartPar
des mots clés associés

\item {} 
\sphinxAtStartPar
ou encore du type de licence si elle est renseignée

\end{itemize}

\noindent{\hspace*{\fill}\sphinxincludegraphics[width=600\sphinxpxdimen]{{filter_options}.png}\hspace*{\fill}}


\subsection{Les fonctionnalités des fiches de données}
\label{\detokenize{doc_user/catalogue:les-fonctionnalites-des-fiches-de-donnees}}
\sphinxAtStartPar
Lorsque vous cliquez sur une donnée, la page de description de cette donnée s’affiche.


\subsubsection{Description de la donnée}
\label{\detokenize{doc_user/catalogue:description-de-la-donnee}}
\sphinxAtStartPar
Le haut de la page est dédiée à la description de cette donnée.
Il y’a :
\begin{itemize}
\item {} 
\sphinxAtStartPar
un titre

\item {} 
\sphinxAtStartPar
une description

\item {} 
\sphinxAtStartPar
la dernière date de la mise à jour,

\item {} 
\sphinxAtStartPar
son point de contact

\item {} 
\sphinxAtStartPar
le catalogue dont elle provient

\item {} 
\sphinxAtStartPar
les mots clés associées

\item {} 
\sphinxAtStartPar
un pourcentage à titre indicatif de la qualité de cette donnée

\item {} 
\sphinxAtStartPar
et d’autre informations plus technique

\end{itemize}

\noindent{\hspace*{\fill}\sphinxincludegraphics[width=600\sphinxpxdimen]{{fiche_info}.png}\hspace*{\fill}}


\subsubsection{Prévisualisation de la donnée}
\label{\detokenize{doc_user/catalogue:previsualisation-de-la-donnee}}
\sphinxAtStartPar
Une interface de prévisualisation est aussi accessible si vous descendez la page.
Cette interface permet de :
\begin{itemize}
\item {} 
\sphinxAtStartPar
prévisualiser la donnée

\end{itemize}

\noindent{\hspace*{\fill}\sphinxincludegraphics[width=600\sphinxpxdimen]{{fiche_previsu}.png}\hspace*{\fill}}
\begin{itemize}
\item {} 
\sphinxAtStartPar
visualiser le tableau attributaire

\end{itemize}

\noindent{\hspace*{\fill}\sphinxincludegraphics[width=600\sphinxpxdimen]{{fiche_table}.png}\hspace*{\fill}}
\begin{itemize}
\item {} 
\sphinxAtStartPar
faire différents graphiques en fonction des attributs

\end{itemize}

\noindent{\hspace*{\fill}\sphinxincludegraphics[width=600\sphinxpxdimen]{{fiche_graphe}.png}\hspace*{\fill}}


\subsubsection{Téléchargement de la donnée}
\label{\detokenize{doc_user/catalogue:telechargement-de-la-donnee}}
\sphinxAtStartPar
Vous pouvez aussi télécharger la donnée sous différents formats :

\noindent{\hspace*{\fill}\sphinxincludegraphics[width=600\sphinxpxdimen]{{fiche_tele}.png}\hspace*{\fill}}

\sphinxAtStartPar
Mais aussi avoir accées à d’autre liens et URL, ainsi qu’aux flux OGC disponibles :

\noindent{\hspace*{\fill}\sphinxincludegraphics[width=600\sphinxpxdimen]{{fiche_liens}.png}\hspace*{\fill}}

\sphinxAtStartPar
Vous pouvez aussi visualiser la donnée dans une interface cartographique en cliquant ici et cela vous fera apparaître le {\hyperref[\detokenize{doc_user/visualiseur:id1}]{\sphinxcrossref{\DUrole{std,std-ref}{visualiseur}}}}.

\noindent{\hspace*{\fill}\sphinxincludegraphics[width=600\sphinxpxdimen]{{fiche_carto}.png}\hspace*{\fill}}

\sphinxstepscope


\section{Visualiseur}
\label{\detokenize{doc_user/visualiseur:visualiseur}}\label{\detokenize{doc_user/visualiseur::doc}}\phantomsection\label{\detokenize{doc_user/visualiseur:id1}}
\begin{sphinxShadowBox}
\sphinxstyletopictitle{Table des matières}
\begin{itemize}
\item {} 
\sphinxAtStartPar
\phantomsection\label{\detokenize{doc_user/visualiseur:id2}}{\hyperref[\detokenize{doc_user/visualiseur:introduction}]{\sphinxcrossref{Introduction}}}

\item {} 
\sphinxAtStartPar
\phantomsection\label{\detokenize{doc_user/visualiseur:id3}}{\hyperref[\detokenize{doc_user/visualiseur:section-1-description}]{\sphinxcrossref{Section 1 : Description}}}

\item {} 
\sphinxAtStartPar
\phantomsection\label{\detokenize{doc_user/visualiseur:id4}}{\hyperref[\detokenize{doc_user/visualiseur:section-2-fonctionnalites}]{\sphinxcrossref{Section 2 : Fonctionnalités}}}

\item {} 
\sphinxAtStartPar
\phantomsection\label{\detokenize{doc_user/visualiseur:id5}}{\hyperref[\detokenize{doc_user/visualiseur:section-3-utilisation}]{\sphinxcrossref{Section 3 : Utilisation}}}

\item {} 
\sphinxAtStartPar
\phantomsection\label{\detokenize{doc_user/visualiseur:id6}}{\hyperref[\detokenize{doc_user/visualiseur:section-4-conseils-et-astuces}]{\sphinxcrossref{Section 4 : Conseils et astuces}}}

\end{itemize}
\end{sphinxShadowBox}


\subsection{Introduction}
\label{\detokenize{doc_user/visualiseur:introduction}}
\sphinxAtStartPar
Ceci est l’introduction de la partie « Visualiseur ».


\subsection{Section 1 : Description}
\label{\detokenize{doc_user/visualiseur:section-1-description}}
\sphinxAtStartPar
Cette section fournit une description détaillée du catalogue.


\subsection{Section 2 : Fonctionnalités}
\label{\detokenize{doc_user/visualiseur:section-2-fonctionnalites}}
\sphinxAtStartPar
Cette section détaille les fonctionnalités disponibles pour le catalogue.


\subsection{Section 3 : Utilisation}
\label{\detokenize{doc_user/visualiseur:section-3-utilisation}}
\sphinxAtStartPar
Cette section explique comment utiliser le catalogue.


\subsection{Section 4 : Conseils et astuces}
\label{\detokenize{doc_user/visualiseur:section-4-conseils-et-astuces}}
\sphinxAtStartPar
Des conseils pour tirer le meilleur parti du catalogue.

\sphinxstepscope


\section{Application}
\label{\detokenize{doc_user/application:application}}\label{\detokenize{doc_user/application::doc}}
\begin{sphinxShadowBox}
\sphinxstyletopictitle{Table des matières}
\begin{itemize}
\item {} 
\sphinxAtStartPar
\phantomsection\label{\detokenize{doc_user/application:id1}}{\hyperref[\detokenize{doc_user/application:introduction}]{\sphinxcrossref{Introduction}}}

\item {} 
\sphinxAtStartPar
\phantomsection\label{\detokenize{doc_user/application:id2}}{\hyperref[\detokenize{doc_user/application:section-1-description}]{\sphinxcrossref{Section 1 : Description}}}

\item {} 
\sphinxAtStartPar
\phantomsection\label{\detokenize{doc_user/application:id3}}{\hyperref[\detokenize{doc_user/application:section-2-fonctionnalites}]{\sphinxcrossref{Section 2 : Fonctionnalités}}}

\item {} 
\sphinxAtStartPar
\phantomsection\label{\detokenize{doc_user/application:id4}}{\hyperref[\detokenize{doc_user/application:section-3-utilisation}]{\sphinxcrossref{Section 3 : Utilisation}}}

\item {} 
\sphinxAtStartPar
\phantomsection\label{\detokenize{doc_user/application:id5}}{\hyperref[\detokenize{doc_user/application:section-4-conseils-et-astuces}]{\sphinxcrossref{Section 4 : Conseils et astuces}}}

\end{itemize}
\end{sphinxShadowBox}


\subsection{Introduction}
\label{\detokenize{doc_user/application:introduction}}
\sphinxAtStartPar
Ceci est l’introduction de la partie « Catalogue ».


\subsection{Section 1 : Description}
\label{\detokenize{doc_user/application:section-1-description}}
\sphinxAtStartPar
Cette section fournit une description détaillée du catalogue.


\subsection{Section 2 : Fonctionnalités}
\label{\detokenize{doc_user/application:section-2-fonctionnalites}}
\sphinxAtStartPar
Cette section détaille les fonctionnalités disponibles pour le catalogue.


\subsection{Section 3 : Utilisation}
\label{\detokenize{doc_user/application:section-3-utilisation}}
\sphinxAtStartPar
Cette section explique comment utiliser le catalogue.


\subsection{Section 4 : Conseils et astuces}
\label{\detokenize{doc_user/application:section-4-conseils-et-astuces}}
\sphinxAtStartPar
Des conseils pour tirer le meilleur parti du catalogue.

\sphinxstepscope


\section{Services}
\label{\detokenize{doc_user/services:services}}\label{\detokenize{doc_user/services::doc}}
\begin{sphinxShadowBox}
\sphinxstyletopictitle{Table des matières}
\begin{itemize}
\item {} 
\sphinxAtStartPar
\phantomsection\label{\detokenize{doc_user/services:id1}}{\hyperref[\detokenize{doc_user/services:introduction}]{\sphinxcrossref{Introduction}}}

\item {} 
\sphinxAtStartPar
\phantomsection\label{\detokenize{doc_user/services:id2}}{\hyperref[\detokenize{doc_user/services:section-1-description}]{\sphinxcrossref{Section 1 : Description}}}

\item {} 
\sphinxAtStartPar
\phantomsection\label{\detokenize{doc_user/services:id3}}{\hyperref[\detokenize{doc_user/services:section-2-fonctionnalites}]{\sphinxcrossref{Section 2 : Fonctionnalités}}}

\item {} 
\sphinxAtStartPar
\phantomsection\label{\detokenize{doc_user/services:id4}}{\hyperref[\detokenize{doc_user/services:section-3-utilisation}]{\sphinxcrossref{Section 3 : Utilisation}}}

\item {} 
\sphinxAtStartPar
\phantomsection\label{\detokenize{doc_user/services:id5}}{\hyperref[\detokenize{doc_user/services:section-4-conseils-et-astuces}]{\sphinxcrossref{Section 4 : Conseils et astuces}}}

\end{itemize}
\end{sphinxShadowBox}


\subsection{Introduction}
\label{\detokenize{doc_user/services:introduction}}
\sphinxAtStartPar
Ceci est l’introduction de la partie « Catalogue ».


\subsection{Section 1 : Description}
\label{\detokenize{doc_user/services:section-1-description}}
\sphinxAtStartPar
Cette section fournit une description détaillée du catalogue.


\subsection{Section 2 : Fonctionnalités}
\label{\detokenize{doc_user/services:section-2-fonctionnalites}}
\sphinxAtStartPar
Cette section détaille les fonctionnalités disponibles pour le catalogue.


\subsection{Section 3 : Utilisation}
\label{\detokenize{doc_user/services:section-3-utilisation}}
\sphinxAtStartPar
Cette section explique comment utiliser le catalogue.


\subsection{Section 4 : Conseils et astuces}
\label{\detokenize{doc_user/services:section-4-conseils-et-astuces}}
\sphinxAtStartPar
Des conseils pour tirer le meilleur parti du catalogue.

\sphinxstepscope


\section{Import}
\label{\detokenize{doc_user/import:import}}\label{\detokenize{doc_user/import::doc}}
\begin{sphinxShadowBox}
\sphinxstyletopictitle{Table des matières}
\begin{itemize}
\item {} 
\sphinxAtStartPar
\phantomsection\label{\detokenize{doc_user/import:id1}}{\hyperref[\detokenize{doc_user/import:introduction}]{\sphinxcrossref{Introduction}}}

\item {} 
\sphinxAtStartPar
\phantomsection\label{\detokenize{doc_user/import:id2}}{\hyperref[\detokenize{doc_user/import:section-1-description}]{\sphinxcrossref{Section 1 : Description}}}

\item {} 
\sphinxAtStartPar
\phantomsection\label{\detokenize{doc_user/import:id3}}{\hyperref[\detokenize{doc_user/import:section-2-fonctionnalites}]{\sphinxcrossref{Section 2 : Fonctionnalités}}}

\item {} 
\sphinxAtStartPar
\phantomsection\label{\detokenize{doc_user/import:id4}}{\hyperref[\detokenize{doc_user/import:section-3-utilisation}]{\sphinxcrossref{Section 3 : Utilisation}}}

\item {} 
\sphinxAtStartPar
\phantomsection\label{\detokenize{doc_user/import:id5}}{\hyperref[\detokenize{doc_user/import:section-4-conseils-et-astuces}]{\sphinxcrossref{Section 4 : Conseils et astuces}}}

\end{itemize}
\end{sphinxShadowBox}


\subsection{Introduction}
\label{\detokenize{doc_user/import:introduction}}
\sphinxAtStartPar
Ceci est l’introduction de la partie « Catalogue ».


\subsection{Section 1 : Description}
\label{\detokenize{doc_user/import:section-1-description}}
\sphinxAtStartPar
Cette section fournit une description détaillée du catalogue.


\subsection{Section 2 : Fonctionnalités}
\label{\detokenize{doc_user/import:section-2-fonctionnalites}}
\sphinxAtStartPar
Cette section détaille les fonctionnalités disponibles pour le catalogue.


\subsection{Section 3 : Utilisation}
\label{\detokenize{doc_user/import:section-3-utilisation}}
\sphinxAtStartPar
Cette section explique comment utiliser le catalogue.


\subsection{Section 4 : Conseils et astuces}
\label{\detokenize{doc_user/import:section-4-conseils-et-astuces}}
\sphinxAtStartPar
Des conseils pour tirer le meilleur parti du catalogue.

\sphinxstepscope


\chapter{Documentation administrateur}
\label{\detokenize{doc_admin:documentation-administrateur}}\label{\detokenize{doc_admin::doc}}
\sphinxAtStartPar
\sphinxhref{\_static/guide\_utilisateur.pdf}{Documentation au format PDF}.

\sphinxstepscope


\section{Application}
\label{\detokenize{doc_admin/catalogue:application}}\label{\detokenize{doc_admin/catalogue::doc}}
\begin{sphinxShadowBox}
\sphinxstyletopictitle{Table des matières}
\begin{itemize}
\item {} 
\sphinxAtStartPar
\phantomsection\label{\detokenize{doc_admin/catalogue:id1}}{\hyperref[\detokenize{doc_admin/catalogue:introduction}]{\sphinxcrossref{Introduction}}}

\item {} 
\sphinxAtStartPar
\phantomsection\label{\detokenize{doc_admin/catalogue:id2}}{\hyperref[\detokenize{doc_admin/catalogue:section-1-description}]{\sphinxcrossref{Section 1 : Description}}}

\item {} 
\sphinxAtStartPar
\phantomsection\label{\detokenize{doc_admin/catalogue:id3}}{\hyperref[\detokenize{doc_admin/catalogue:section-2-fonctionnalites}]{\sphinxcrossref{Section 2 : Fonctionnalités}}}

\item {} 
\sphinxAtStartPar
\phantomsection\label{\detokenize{doc_admin/catalogue:id4}}{\hyperref[\detokenize{doc_admin/catalogue:section-3-utilisation}]{\sphinxcrossref{Section 3 : Utilisation}}}

\item {} 
\sphinxAtStartPar
\phantomsection\label{\detokenize{doc_admin/catalogue:id5}}{\hyperref[\detokenize{doc_admin/catalogue:section-4-conseils-et-astuces}]{\sphinxcrossref{Section 4 : Conseils et astuces}}}

\end{itemize}
\end{sphinxShadowBox}


\subsection{Introduction}
\label{\detokenize{doc_admin/catalogue:introduction}}
\sphinxAtStartPar
Ceci est l’introduction de la partie « Catalogue ».


\subsection{Section 1 : Description}
\label{\detokenize{doc_admin/catalogue:section-1-description}}
\sphinxAtStartPar
Cette section fournit une description détaillée du catalogue.


\subsection{Section 2 : Fonctionnalités}
\label{\detokenize{doc_admin/catalogue:section-2-fonctionnalites}}
\sphinxAtStartPar
Cette section détaille les fonctionnalités disponibles pour le catalogue.


\subsection{Section 3 : Utilisation}
\label{\detokenize{doc_admin/catalogue:section-3-utilisation}}
\sphinxAtStartPar
Cette section explique comment utiliser le catalogue.


\subsection{Section 4 : Conseils et astuces}
\label{\detokenize{doc_admin/catalogue:section-4-conseils-et-astuces}}
\sphinxAtStartPar
Des conseils pour tirer le meilleur parti du catalogue.

\sphinxstepscope


\section{Application}
\label{\detokenize{doc_admin/visualiseur:application}}\label{\detokenize{doc_admin/visualiseur::doc}}
\begin{sphinxShadowBox}
\sphinxstyletopictitle{Table des matières}
\begin{itemize}
\item {} 
\sphinxAtStartPar
\phantomsection\label{\detokenize{doc_admin/visualiseur:id1}}{\hyperref[\detokenize{doc_admin/visualiseur:introduction}]{\sphinxcrossref{Introduction}}}

\item {} 
\sphinxAtStartPar
\phantomsection\label{\detokenize{doc_admin/visualiseur:id2}}{\hyperref[\detokenize{doc_admin/visualiseur:section-1-description}]{\sphinxcrossref{Section 1 : Description}}}

\item {} 
\sphinxAtStartPar
\phantomsection\label{\detokenize{doc_admin/visualiseur:id3}}{\hyperref[\detokenize{doc_admin/visualiseur:section-2-fonctionnalites}]{\sphinxcrossref{Section 2 : Fonctionnalités}}}

\item {} 
\sphinxAtStartPar
\phantomsection\label{\detokenize{doc_admin/visualiseur:id4}}{\hyperref[\detokenize{doc_admin/visualiseur:section-3-utilisation}]{\sphinxcrossref{Section 3 : Utilisation}}}

\item {} 
\sphinxAtStartPar
\phantomsection\label{\detokenize{doc_admin/visualiseur:id5}}{\hyperref[\detokenize{doc_admin/visualiseur:section-4-conseils-et-astuces}]{\sphinxcrossref{Section 4 : Conseils et astuces}}}

\end{itemize}
\end{sphinxShadowBox}


\subsection{Introduction}
\label{\detokenize{doc_admin/visualiseur:introduction}}
\sphinxAtStartPar
Ceci est l’introduction de la partie « Catalogue ».


\subsection{Section 1 : Description}
\label{\detokenize{doc_admin/visualiseur:section-1-description}}
\sphinxAtStartPar
Cette section fournit une description détaillée du catalogue.


\subsection{Section 2 : Fonctionnalités}
\label{\detokenize{doc_admin/visualiseur:section-2-fonctionnalites}}
\sphinxAtStartPar
Cette section détaille les fonctionnalités disponibles pour le catalogue.


\subsection{Section 3 : Utilisation}
\label{\detokenize{doc_admin/visualiseur:section-3-utilisation}}
\sphinxAtStartPar
Cette section explique comment utiliser le catalogue.


\subsection{Section 4 : Conseils et astuces}
\label{\detokenize{doc_admin/visualiseur:section-4-conseils-et-astuces}}
\sphinxAtStartPar
Des conseils pour tirer le meilleur parti du catalogue.

\sphinxstepscope


\section{Application}
\label{\detokenize{doc_admin/services:application}}\label{\detokenize{doc_admin/services::doc}}
\begin{sphinxShadowBox}
\sphinxstyletopictitle{Table des matières}
\begin{itemize}
\item {} 
\sphinxAtStartPar
\phantomsection\label{\detokenize{doc_admin/services:id1}}{\hyperref[\detokenize{doc_admin/services:introduction}]{\sphinxcrossref{Introduction}}}

\item {} 
\sphinxAtStartPar
\phantomsection\label{\detokenize{doc_admin/services:id2}}{\hyperref[\detokenize{doc_admin/services:section-1-description}]{\sphinxcrossref{Section 1 : Description}}}

\item {} 
\sphinxAtStartPar
\phantomsection\label{\detokenize{doc_admin/services:id3}}{\hyperref[\detokenize{doc_admin/services:section-2-fonctionnalites}]{\sphinxcrossref{Section 2 : Fonctionnalités}}}

\item {} 
\sphinxAtStartPar
\phantomsection\label{\detokenize{doc_admin/services:id4}}{\hyperref[\detokenize{doc_admin/services:section-3-utilisation}]{\sphinxcrossref{Section 3 : Utilisation}}}

\item {} 
\sphinxAtStartPar
\phantomsection\label{\detokenize{doc_admin/services:id5}}{\hyperref[\detokenize{doc_admin/services:section-4-conseils-et-astuces}]{\sphinxcrossref{Section 4 : Conseils et astuces}}}

\end{itemize}
\end{sphinxShadowBox}


\subsection{Introduction}
\label{\detokenize{doc_admin/services:introduction}}
\sphinxAtStartPar
Ceci est l’introduction de la partie « Catalogue ».


\subsection{Section 1 : Description}
\label{\detokenize{doc_admin/services:section-1-description}}
\sphinxAtStartPar
Cette section fournit une description détaillée du catalogue.


\subsection{Section 2 : Fonctionnalités}
\label{\detokenize{doc_admin/services:section-2-fonctionnalites}}
\sphinxAtStartPar
Cette section détaille les fonctionnalités disponibles pour le catalogue.


\subsection{Section 3 : Utilisation}
\label{\detokenize{doc_admin/services:section-3-utilisation}}
\sphinxAtStartPar
Cette section explique comment utiliser le catalogue.


\subsection{Section 4 : Conseils et astuces}
\label{\detokenize{doc_admin/services:section-4-conseils-et-astuces}}
\sphinxAtStartPar
Des conseils pour tirer le meilleur parti du catalogue.

\sphinxstepscope


\section{Application}
\label{\detokenize{doc_admin/utilisateurs:application}}\label{\detokenize{doc_admin/utilisateurs::doc}}
\begin{sphinxShadowBox}
\sphinxstyletopictitle{Table des matières}
\begin{itemize}
\item {} 
\sphinxAtStartPar
\phantomsection\label{\detokenize{doc_admin/utilisateurs:id1}}{\hyperref[\detokenize{doc_admin/utilisateurs:introduction}]{\sphinxcrossref{Introduction}}}

\item {} 
\sphinxAtStartPar
\phantomsection\label{\detokenize{doc_admin/utilisateurs:id2}}{\hyperref[\detokenize{doc_admin/utilisateurs:section-1-description}]{\sphinxcrossref{Section 1 : Description}}}

\item {} 
\sphinxAtStartPar
\phantomsection\label{\detokenize{doc_admin/utilisateurs:id3}}{\hyperref[\detokenize{doc_admin/utilisateurs:section-2-fonctionnalites}]{\sphinxcrossref{Section 2 : Fonctionnalités}}}

\item {} 
\sphinxAtStartPar
\phantomsection\label{\detokenize{doc_admin/utilisateurs:id4}}{\hyperref[\detokenize{doc_admin/utilisateurs:section-3-utilisation}]{\sphinxcrossref{Section 3 : Utilisation}}}

\item {} 
\sphinxAtStartPar
\phantomsection\label{\detokenize{doc_admin/utilisateurs:id5}}{\hyperref[\detokenize{doc_admin/utilisateurs:section-4-conseils-et-astuces}]{\sphinxcrossref{Section 4 : Conseils et astuces}}}

\end{itemize}
\end{sphinxShadowBox}


\subsection{Introduction}
\label{\detokenize{doc_admin/utilisateurs:introduction}}
\sphinxAtStartPar
Ceci est l’introduction de la partie « Catalogue ».


\subsection{Section 1 : Description}
\label{\detokenize{doc_admin/utilisateurs:section-1-description}}
\sphinxAtStartPar
Cette section fournit une description détaillée du catalogue.


\subsection{Section 2 : Fonctionnalités}
\label{\detokenize{doc_admin/utilisateurs:section-2-fonctionnalites}}
\sphinxAtStartPar
Cette section détaille les fonctionnalités disponibles pour le catalogue.


\subsection{Section 3 : Utilisation}
\label{\detokenize{doc_admin/utilisateurs:section-3-utilisation}}
\sphinxAtStartPar
Cette section explique comment utiliser le catalogue.


\subsection{Section 4 : Conseils et astuces}
\label{\detokenize{doc_admin/utilisateurs:section-4-conseils-et-astuces}}
\sphinxAtStartPar
Des conseils pour tirer le meilleur parti du catalogue.

\sphinxstepscope


\section{Application}
\label{\detokenize{doc_admin/analytics:application}}\label{\detokenize{doc_admin/analytics::doc}}
\begin{sphinxShadowBox}
\sphinxstyletopictitle{Table des matières}
\begin{itemize}
\item {} 
\sphinxAtStartPar
\phantomsection\label{\detokenize{doc_admin/analytics:id1}}{\hyperref[\detokenize{doc_admin/analytics:introduction}]{\sphinxcrossref{Introduction}}}

\item {} 
\sphinxAtStartPar
\phantomsection\label{\detokenize{doc_admin/analytics:id2}}{\hyperref[\detokenize{doc_admin/analytics:section-1-description}]{\sphinxcrossref{Section 1 : Description}}}

\item {} 
\sphinxAtStartPar
\phantomsection\label{\detokenize{doc_admin/analytics:id3}}{\hyperref[\detokenize{doc_admin/analytics:section-2-fonctionnalites}]{\sphinxcrossref{Section 2 : Fonctionnalités}}}

\item {} 
\sphinxAtStartPar
\phantomsection\label{\detokenize{doc_admin/analytics:id4}}{\hyperref[\detokenize{doc_admin/analytics:section-3-utilisation}]{\sphinxcrossref{Section 3 : Utilisation}}}

\item {} 
\sphinxAtStartPar
\phantomsection\label{\detokenize{doc_admin/analytics:id5}}{\hyperref[\detokenize{doc_admin/analytics:section-4-conseils-et-astuces}]{\sphinxcrossref{Section 4 : Conseils et astuces}}}

\end{itemize}
\end{sphinxShadowBox}


\subsection{Introduction}
\label{\detokenize{doc_admin/analytics:introduction}}
\sphinxAtStartPar
Ceci est l’introduction de la partie « Catalogue ».


\subsection{Section 1 : Description}
\label{\detokenize{doc_admin/analytics:section-1-description}}
\sphinxAtStartPar
Cette section fournit une description détaillée du catalogue.


\subsection{Section 2 : Fonctionnalités}
\label{\detokenize{doc_admin/analytics:section-2-fonctionnalites}}
\sphinxAtStartPar
Cette section détaille les fonctionnalités disponibles pour le catalogue.


\subsection{Section 3 : Utilisation}
\label{\detokenize{doc_admin/analytics:section-3-utilisation}}
\sphinxAtStartPar
Cette section explique comment utiliser le catalogue.


\subsection{Section 4 : Conseils et astuces}
\label{\detokenize{doc_admin/analytics:section-4-conseils-et-astuces}}
\sphinxAtStartPar
Des conseils pour tirer le meilleur parti du catalogue.

\sphinxstepscope


\chapter{Documentation d’installation}
\label{\detokenize{doc_instal:documentation-d-installation}}\label{\detokenize{doc_instal::doc}}
\sphinxAtStartPar
\sphinxhref{\_static/guide\_utilisateur.pdf}{Documentation au format PDF}.

\sphinxstepscope


\section{Installation}
\label{\detokenize{doc_instal/installation:installation}}\label{\detokenize{doc_instal/installation::doc}}
\begin{sphinxShadowBox}
\sphinxstyletopictitle{Table des matières}
\begin{itemize}
\item {} 
\sphinxAtStartPar
\phantomsection\label{\detokenize{doc_instal/installation:id1}}{\hyperref[\detokenize{doc_instal/installation:introduction}]{\sphinxcrossref{Introduction}}}

\item {} 
\sphinxAtStartPar
\phantomsection\label{\detokenize{doc_instal/installation:id2}}{\hyperref[\detokenize{doc_instal/installation:ansible}]{\sphinxcrossref{Ansible}}}

\item {} 
\sphinxAtStartPar
\phantomsection\label{\detokenize{doc_instal/installation:id3}}{\hyperref[\detokenize{doc_instal/installation:script-de-personnalisation}]{\sphinxcrossref{Script de personnalisation}}}

\end{itemize}
\end{sphinxShadowBox}


\subsection{Introduction}
\label{\detokenize{doc_instal/installation:introduction}}
\sphinxAtStartPar
Ceci est l’introduction de la partie « Catalogue ».


\subsection{Ansible}
\label{\detokenize{doc_instal/installation:ansible}}
\sphinxAtStartPar
Cette section fournit une description détaillée du catalogue.


\subsection{Script de personnalisation}
\label{\detokenize{doc_instal/installation:script-de-personnalisation}}
\sphinxAtStartPar
Cette section détaille les fonctionnalités disponibles pour le catalogue.

\sphinxstepscope


\section{Configuration}
\label{\detokenize{doc_instal/configuration:configuration}}\label{\detokenize{doc_instal/configuration::doc}}
\begin{sphinxShadowBox}
\sphinxstyletopictitle{Table des matières}
\begin{itemize}
\item {} 
\sphinxAtStartPar
\phantomsection\label{\detokenize{doc_instal/configuration:id1}}{\hyperref[\detokenize{doc_instal/configuration:introduction}]{\sphinxcrossref{Introduction}}}

\item {} 
\sphinxAtStartPar
\phantomsection\label{\detokenize{doc_instal/configuration:id2}}{\hyperref[\detokenize{doc_instal/configuration:chaque-module}]{\sphinxcrossref{Chaque module}}}

\item {} 
\sphinxAtStartPar
\phantomsection\label{\detokenize{doc_instal/configuration:id3}}{\hyperref[\detokenize{doc_instal/configuration:relancer-l-installation}]{\sphinxcrossref{relancer l’installation}}}

\end{itemize}
\end{sphinxShadowBox}


\subsection{Introduction}
\label{\detokenize{doc_instal/configuration:introduction}}
\sphinxAtStartPar
Ceci est l’introduction de la partie « Catalogue ».


\subsection{Chaque module}
\label{\detokenize{doc_instal/configuration:chaque-module}}
\sphinxAtStartPar
Cette section fournit une description détaillée du catalogue.


\subsection{relancer l’installation}
\label{\detokenize{doc_instal/configuration:relancer-l-installation}}
\sphinxAtStartPar
Cette section détaille les fonctionnalités disponibles pour le catalogue.

\sphinxstepscope


\section{Mises à jour}
\label{\detokenize{doc_instal/maj:mises-a-jour}}\label{\detokenize{doc_instal/maj::doc}}
\begin{sphinxShadowBox}
\sphinxstyletopictitle{Table des matières}
\begin{itemize}
\item {} 
\sphinxAtStartPar
\phantomsection\label{\detokenize{doc_instal/maj:id1}}{\hyperref[\detokenize{doc_instal/maj:introduction}]{\sphinxcrossref{Introduction}}}

\item {} 
\sphinxAtStartPar
\phantomsection\label{\detokenize{doc_instal/maj:id2}}{\hyperref[\detokenize{doc_instal/maj:paquets-debians}]{\sphinxcrossref{Paquets debians}}}

\item {} 
\sphinxAtStartPar
\phantomsection\label{\detokenize{doc_instal/maj:id3}}{\hyperref[\detokenize{doc_instal/maj:directements-avec-l-installation-donc-georchestra-yml}]{\sphinxcrossref{Directements avec l’installation donc georchestra.yml}}}

\end{itemize}
\end{sphinxShadowBox}


\subsection{Introduction}
\label{\detokenize{doc_instal/maj:introduction}}
\sphinxAtStartPar
Ceci est l’introduction de la partie « Catalogue ».


\subsection{Paquets debians}
\label{\detokenize{doc_instal/maj:paquets-debians}}
\sphinxAtStartPar
Cette section fournit une description détaillée du catalogue.


\subsection{Directements avec l’installation donc georchestra.yml}
\label{\detokenize{doc_instal/maj:directements-avec-l-installation-donc-georchestra-yml}}
\sphinxAtStartPar
Cette section détaille les fonctionnalités disponibles pour le catalogue.



\renewcommand{\indexname}{Index}
\printindex
\end{document}